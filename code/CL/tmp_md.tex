# Main note: ../main.md 

This the following notes are the over-arching goal of this project, which is, to understand the mathematical principles, at a mechnistic level, of how loss of plasticity emerges, how it persists, and how it could possibly be recovered from. 


# Recap and background
Here we recap the findings we have and review some background on  continual learning from the point of view of Ricahrd Sutton. 
``

*Broad objectives*: Through a series of works on Richard Sutton, he proposes that one of the most fundamental problems facing AI today is that we cannot let a model continually learn different tasks. This problem has two facets: 1) the catastrophic forgetting, which is forgetting what network has learned in the past, and 2) the issue of loss of plasticity, which is the network losing its ability to learn new concepts. We focus on the later issue of CL her, as it clearly relates to the theoretical result proven later. While there are numerous valuable observations and remedies in these works, there are numerous gaps in our understanding of CL, which makes us unable to build models and algorithms that are able to achieve CL in the same fashion as humans or animals in the real world achieve. Our attempt is to give the key ideas and notions in continual learning a theoretical footing, thereby consolidating the key root causes of loss plasticity. 

We first start by a review of loss of plasticity from Richard Sutton’s point of view, a review of hyper cloning. Then, we elaborate on our theories regarding the noise-less and noisy cases. Finally, we draw on parallels between the key results of our theory and loss of plasticity, where we highlight where it aligns and where it might offer a different perspective on fundamental causes of loss of plasticity.  Based on these theoretical results, we make several hypotheses on how these root causes of loss of plasticity emerge during prolonged training. We would like to understand the architectural and training dynamics that contribute to loss of plasticity in order to both avoid the loss of plasticity during training, and be able to recover from it if it is not catastrophic. Finally, we will also rely on empirical evidence to evaluate the hypotheses that were inspired by our theory. 

## Richard Sutton's view on Continual Learning (CL):
 Here we very briefly review the key notions and ideas that are proposed by Richard Sutton to diagnose, quantify and alleviate barriers to continual learning. Through a series of works, he observes that after having being trained one some tasks, the model’s ability to learn new tasks drops irrecoverably the level of a shallow or even linear model.  Here’s a brief recap of his key contributions to understanding and remedies for loss of plasticity. 

1. Standard Deep Learning and Continual Learning:
    * One‐Time vs. Continual Learning: Traditional deep‐learning methods (using backpropagation with gradient descent or variants such as Adam) are designed for “one‐time” training on a fixed dataset. In many real‐world applications—such as robotics, streaming data, or online reinforcement learning—the data distribution changes over time, requiring the network to continually learn.
    * Loss of Plasticity: Over time, as standard training continues in a non-stationary (continual) learning setting, deep networks lose their “plasticity” (i.e. the ability to quickly adapt to new data). This loss is manifested in several ways:
        * The weights tend to grow larger.
        * A growing fraction of neurons become “dead” (or saturated), meaning that they rarely change their output.
        * The internal representations (the “feature diversity”) become less rich, as measured by a decrease in the effective rank of the hidden layers.
    * This degradation means that—even if early performance on new tasks is good—the network eventually learns no better than a shallow (or even a linear) system when faced with many successive tasks.
2. Empirical Demonstrations:
    * Extensive experiments were conducted on supervised tasks (e.g., variations of ImageNet, class-incremental CIFAR‑100, Online Permuted MNIST, and a “Slowly Changing Regression” problem) and reinforcement learning tasks (such as controlling an “Ant” robot with changing friction).
    * In all these settings, standard backpropagation methods initially learn well but then gradually “forget how to learn” (i.e. they lose plasticity) over hundreds or thousands of tasks.
3. Maintaining Plasticity by Injecting Randomness:
    * The initial random weight initialization provides many advantages (diverse features, small weights, non-saturation) that enable rapid learning early on. However, because standard backprop only applies this “randomness” at the start, these beneficial properties fade with continued training.
    * The key idea is that continual learning requires a sustained injection of randomness or variability to maintain plasticity.
4. Continual Backpropagation (CBP):
    * To counteract the decay of plasticity, the authors propose an algorithm called Continual Backpropagation. CBP is almost identical to standard backpropagation except that, on every update, it selectively reinitializes a very small fraction of the network’s units.
    * Selective Reinitialization: Using a “utility measure” that assesses how useful a neuron (or feature) is for the current task (based on factors such as its activation, its outgoing weight magnitudes, and how much it is changing), the algorithm identifies neurons that are “underused” or “dead.” These neurons are then reinitialized (with the initial small random values), thereby reintroducing diversity and the benefits of a fresh start.
    * This process—sometimes called a “generate-and-test” mechanism—allows the network to continually inject new random features without having to completely reset or lose past learning.
5. Comparison with Other Methods:
    * Other techniques such as L2 regularization, Shrink and Perturb (which combines weight shrinkage with noise injection), dropout, and normalization were examined.
    * Although L2 regularization and Shrink and Perturb help slow the growth of weights and partially mitigate the loss of plasticity, they are generally less robust than CBP. In some experiments (both in supervised and reinforcement learning settings), popular methods like Adam (with standard parameters), dropout, and even batch normalization actually worsened the loss of plasticity over time.
6. Implications for Continual and Reinforcement Learning:
    * The findings imply that if deep neural networks are to be deployed in environments where continual adaptation is necessary, the training algorithms must be modified to continuously “refresh” the network’s ability to learn.
    * In reinforcement learning, where both the environment and the agent’s behavior can change over time, the loss of plasticity is especially problematic. The continual backpropagation approach (sometimes combined with a small amount of L2 regularization) was shown to significantly improve performance in nonstationary RL tasks (for example, in controlling an ant robot in environments with changing friction).
7. Broader Perspective:
    * The work challenges the assumption that gradient descent alone is sufficient for deep learning in dynamic, nonstationary settings.
    * It suggests that “sustained deep learning” (learning that continues to adapt over time) may require algorithms that combine traditional gradient-based methods with mechanisms for continual variability—in effect, a built-in “refresh” mechanism similar to how biological systems continually reorganize and adapt their neural circuitry.


## Hyper cloning recap and connection to loss of plasticity 

 In hyper cloning, authors showed various methods to “enlarge” a trained smaller model, such that its forward pass is perfectly preserved during cloning process, and showing that it can be trained much faster than a large model from scratch while achieving a similar accuracy. However, authors also observe that in the cloning strategies that were “noiseless” the training was not as effective, and had to introduce some types of noise to ensure the training diverged from a simple setting. 

*The theoretical result on cloning:* In this work, we first prove a series of result that are directly stated in terms of the 
We proved in the draft that if we duplicate a hyper-clone a model in a super-symmetric way (more accurately, forward and backward symmetry hold), then the forward and backward vectors of the network are cloned. More concretely, the forward and backward of the model are essentially the cloned (duplicated) versions of a smaller model from which they are cloned. This situation has a very dramatic consequence that, we can perfectly predict the training dynamics of the larger model with a smaller model, with the only caveat that the learning rate for different layers are set in a layer and module-dependent manner.  On the other hand, we show that the noisy cloning strategies can be modeled as the noiseless close plus some additive noise to the backprop gradients, where norm of the gradients per each layer may depend on their depth and network parameters in general. 

## Connection to loss of plasticity?

From a high level point of view, we can view the noiseless cloning strategies as analogues of normal backprop, in that this particular way of cloning may catastrophically limit the model's ability to learn, because it will always be as good as the smaller model, which highly resembles  notion of loss of plasticity phenomenon. 
More concretely, the fact that we can prove that the forward representation remain cloned throughout training, also implies indirectly that the rank of forward representations the larger model will always be equal to rank of the smaller model. This is arguably a very strong case of loss of rank (hard rank as opposed to empirical rank in Richard Sutton’s paper), which is provably unrecoverable! 


*Recovery of loss of plasticity*: Furthermore, our theory shows that  noisy cloning strategies as analogues of continual backprop in that they can be viewed as a normal backprop plus some injected noise.  Therefore, the ability to recover the large model capacity by having some type of noise is an analogue of resetting certain weights and units in continual backprop. 

*Key differences and remaining questions* Despite the similarity of our results on key aspects, there remains some important differences and open questions. 

Firstly, if we assume that in the smaller model all neurons are active and the weights are not saturated, the larger model will have similar properties. Therefore, we construct examples of a network that have catastrophic loss of plasticity, while having no dead neurons or overly large weights. In other words, our theory suggests that while dead and saturated neurons may be a sufficient condition for loss o plasticity, they are not necessary conditions. Conversely, the examples observed by Richard Sutton and other empirical evidence suggests that a network can experience loss of plasticity without having explicitly cloned neurons. In yet another words, while cloning or dead or saturated neurons may occur in some cases of loss of plasticity, but not all cases. In our quest to understand loss of plasticity, we will be searching for a set of “universal” conditions that are necessary and sufficient. Another key aspect of loss of plasticity is that without direct intervention, they will “persist” throughout training. Thus, we must be able to demonstrate  that once these conditions  are met, they will persist to be true throughout training, unless  something like a noise injection breaks these properties explicitly. We can refer to these properties as “canonical properties” of loss of plasticity. 

Secondly, even if we show that these conditions are universal and persistent, it is unclear how these conditions “emerge” during training? In other words, even if we have correctly identified these universal causes and can prove they are persistent, it’s still highly non-trivial why and how these conditions emerge as a result of prolonged training.  For example, while dead neurons or saturated neurons or cloned neurons , it is not guaranteed at all (theoretically or empirically) at all if such properties will emerge as a result of prolonged training. 
Since we know that certain choices of models and training will improve or deteriorate the emergence of these conditions, one can argue that that answer to this “emergence” question will be architecture and configuration-dependent  and not universal. Thus, once we can establish some clear links between training dynamics and model architecture, and loss of plasticity, we can also suggest ways to avoid loss of plasticity, or suggest methods to recover from it. 

Thus, these are the following key questions regarding loss of plasticity we are trying address in this work:
* *Canonical conditions*: What are they underlying universal conditions  that are necessary and sufficient, and they will persist throughout training?  
* *Emergence with training*: How do these conditions emerge as a result of prolonged training? “

### Canonical conditions
Let us try to capitalize on our earlier theories about cloning to arrive at a formalization of the question we are trying to address: 

*Definition*: In a neural network with parameters $\Theta$ , we define:
* Universal: we say $C(\Theta)$ a universal condition for loss of plasticity for network, if there some smaller network  with parameters $\theta$ ($|Theta| > |\theta|$), such that the larger model hidden units are exactly a copy of the smaller model up to some duplications. 
* Persistent:  having the we say that a $C(\Theta)$ is persistent
Let’s start by reviewing the key aspect of our theory on cloning and Richard Sutton’s views on loss of plasticity. Most notably, both views suggest that  a collapse of forward representations is a key indicator of loss of plasticity. However, as mentioned earlier, neither dead, saturated, nor neurons nor duplicated are necessary for loss of plasticity to occur. 




--- Processing: ../CL_RS.md ---

# Rich Sutton's Research Contributions to Continual Learning

## 1. Foundational Contributions
Rich Sutton has played a pivotal role in shaping reinforcement learning (RL), laying groundwork that also underpins continual learning. Many of his seminal contributions introduced algorithms and frameworks for agents to learn incrementally from ongoing experience – a core aspect of continual learning. Key foundational contributions include:

- **Temporal-Difference (TD) Learning**: Sutton’s 1988 work on TD learning introduced a method for an agent to update predictions by bootstrapping from newer estimates rather than waiting for final outcomes ([Reinforcement Learning: Temporal Difference (TD) Learning](https://www.lancaster.ac.uk/stor-i-student-sites/jordan-j-hood/2021/04/12/reinforcement-learning-temporal-difference-td-learning/#:~:text=Learning%20www,as%20the%20name%20suggests%2C)). TD learning merged strengths of dynamic programming and Monte Carlo methods, enabling effective incremental learning of value functions. This concept is *“likely the most core concept in Reinforcement Learning”* ([Reinforcement Learning: Temporal Difference (TD) Learning](https://www.lancaster.ac.uk/stor-i-student-sites/jordan-j-hood/2021/04/12/reinforcement-learning-temporal-difference-td-learning/#:~:text=Learning%20www,as%20the%20name%20suggests%2C)) and allowed agents to learn from a continuous stream of data, a prerequisite for continual adaptation.

- **Dyna Architecture**: In 1991, Sutton proposed the Dyna architecture, an integrated approach combining learning, planning, and reacting. In Dyna, an agent learns a world model online and uses it for simulated experience (planning) alongside real experience ([[PDF] Integrated Modeling and Control Based on Reinforcement Learning](https://papers.nips.cc/paper/1990/file/d9fc5b73a8d78fad3d6dffe419384e70-Paper.pdf#:~:text=,Dyna%20is%20based%20on)). *“Dyna architectures are those that learn a world model online while using approximations to [dynamic programming] to learn and plan optimal behavior”* ([[PDF] Integrated Modeling and Control Based on Reinforcement Learning](https://papers.nips.cc/paper/1990/file/d9fc5b73a8d78fad3d6dffe419384e70-Paper.pdf#:~:text=,Dyna%20is%20based%20on)). This framework was foundational for continual learning, as it showed how an agent could keep learning and improving by reusing past knowledge (via the learned model) in an ongoing way.

- **Options Framework (Hierarchical RL)**: Sutton (with Precup and Singh, 1999) introduced the options framework, which defines *“temporally extended ways of behaving”* (options) in RL ([[PDF] Temporal Abstraction in Temporal-difference Networks](http://papers.neurips.cc/paper/2826-temporal-abstraction-in-temporal-difference-networks.pdf#:~:text=%5BPDF%5D%20Temporal%20Abstraction%20in%20Temporal,and%20about%20predictions%20of)). Options are higher-level actions or skills that consist of lower-level primitives, with policies and termination conditions. This framework allows an agent to learn and reuse skills across tasks, effectively providing a form of knowledge transfer and memory over time. *“Generalization of one-step actions to option models… enables an agent to predict and reason at multiple time scales”* ([[PDF] Temporally Abstract Partial Models - OpenReview](https://openreview.net/pdf?id=LGvlCcMgWqb#:~:text=,reason%20at%20multiple%20time%20scales)), which is crucial for continual learning scenarios where the agent must build on prior skills.

- **General Value Functions and the Horde Architecture**: In more recent work, Sutton and colleagues developed the concept of General Value Functions (GVFs) and the Horde architecture (2011) for learning many predictions in parallel. Horde is a framework with a “democracy” of prediction-learning processes (termed “demons”) each learning a GVF about the agent’s sensorimotor stream. Sutton’s team demonstrated that an agent can scale to *“learn multiple pre-defined objectives in parallel”* and accumulate predictive knowledge continuously ([[1206.6262] Scaling Life-long Off-policy Learning - arXiv](https://arxiv.org/abs/1206.6262#:~:text=We%20build%20on%20our%20prior,to%20represent%20a%20wide)). In their words, *“GVFs have been shown able to represent a wide [range of predictions]”* in a lifelong learning setting ([[1206.6262] Scaling Life-long Off-policy Learning - arXiv](https://arxiv.org/abs/1206.6262#:~:text=We%20build%20on%20our%20prior,to%20represent%20a%20wide)). This idea of learning many predictions simultaneously without forgetting earlier ones directly informs continual learning research.

Sutton’s foundational work, including the widely used RL textbook (Sutton & Barto, 1998), established core algorithms and principles (e.g. incremental updates, bootstrapping, and exploration strategies) that enable an agent to learn continually. These contributions introduced formalisms and tools – such as TD error, experience replay (used later in deep RL and continual learning), and function approximation techniques – that remain central in modern continual learning research.

## 2. Relation to Reinforcement Learning
Continual learning and reinforcement learning are deeply intertwined, and Sutton’s work bridges them both conceptually and methodologically. Reinforcement learning deals with agents learning from an ongoing stream of interactions with an environment, which naturally aligns with the idea of *continual* learning (learning that never truly stops). Sutton himself has emphasized the importance of agents that keep learning over time. For example, continual reinforcement learning has been defined as the setting in which an agent *“never stop[s] learning”* ([A Definition of Continual Reinforcement Learning - arXiv](https://arxiv.org/html/2307.11046v2#:~:text=In%20contrast%2C%20continual%20reinforcement%20learning,the%20importance%20of%20continual)), highlighting that the best agents are those that can learn indefinitely. This ethos is a direct reflection of Sutton’s lifelong advocacy for incremental, online learning in RL.

Several RL principles introduced or popularized by Sutton have influenced continual learning algorithms:
- **Online Incremental Updates**: Methods like TD learning and gradient-descent updates allow learning to happen incrementally with each new observation, rather than in large batches. This is essential for continual learning, where data arrives sequentially. Sutton’s algorithms (e.g. TD(λ), SARSA, Q-learning refinements) showed how an agent can update knowledge on the fly and revisit old predictions efficiently, which is also how continual learning systems update without retraining from scratch.
- **Experience Replay and Off-Policy Learning**: While not invented solely by Sutton, the idea of reusing past experiences (experience replay) in RL (pioneered by Lin and later used in DQN) connects to rehearsal strategies in continual learning. Off-policy learning algorithms (such as Q-learning or off-policy TD) studied by Sutton enable learning from older data or from hypothetical trajectories (as in Dyna) ([[PDF] Integrated Modeling and Control Based on Reinforcement Learning](https://papers.nips.cc/paper/1990/file/d9fc5b73a8d78fad3d6dffe419384e70-Paper.pdf#:~:text=,Dyna%20is%20based%20on)), analogous to how rehearsal or memory replay methods mitigate forgetting in continual learning.
- **Exploration and Non-Stationarity**: RL deals with non-stationary data distributions when an agent’s policy changes or the environment changes. Sutton’s work on exploration strategies and non-stationary value functions (e.g. in continuing tasks) provides insight into continual learning, where the data distribution can shift over time (new tasks or contexts). Techniques ensuring stability in RL (like eligibility traces and stable function approximation) help inspire mechanisms to balance stability and plasticity in continual learning.

Importantly, Sutton has argued that the traditional ML focus on training static models (what he calls “non-continual learning”) is limiting. He suggests that solving AI requires agents that learn and adapt continually in the long run. In a recent interview, he *“argues the focus on non-continual learning over the past 40 years is now holding AI back”* ([Rich Sutton's new path for AI - Audacy](https://www.audacy.com/podcast/approximately-correct-an-ai-podcast-from-amii-d6257/episodes/rich-suttons-new-path-for-ai-4a1fa#:~:text=Rich%20Sutton%27s%20new%20path%20for,is%20now%20holding%20AI%20back)). In other words, many successes in ML (e.g. deep learning on fixed datasets) may plateau unless we embrace continual learning principles inherent in the RL paradigm. This perspective has encouraged researchers to apply RL-based thinking (like continual exploration, reward-driven adaptation, and lifelong skill acquisition) to broader continual learning problems.

The influence of Sutton’s RL work is also evident in how continual learning researchers design their algorithms. For example, the formal definition of continual reinforcement learning in recent literature ([A Definition of Continual Reinforcement Learning - arXiv](https://arxiv.org/html/2307.11046v2#:~:text=In%20contrast%2C%20continual%20reinforcement%20learning,the%20importance%20of%20continual)) echoes Sutton’s vision of an *always-learning* agent. Overall, Sutton’s reinforcement learning contributions provide both the theoretical foundation and practical algorithms that continual learning research builds upon, underscoring that an agent’s knowledge should *accumulate and adapt over its entire lifetime* rather than being learned once and for all.

## 3. Recent Advances
Continual learning has seen rapid progress in recent years, spurred in part by the deep learning revolution and by the principles established by Sutton and others. Researchers have proposed various strategies to enable neural networks to learn sequentially without forgetting past knowledge. Many of these advances can be seen as elaborations of ideas present in RL or directly influenced by Sutton’s insights (such as using regularization to protect learned knowledge or replaying experiences). Notable recent developments include:

- **Regularization-Based Methods**: These methods add constraints to the learning process to prevent catastrophic forgetting. A prime example is **Elastic Weight Consolidation (EWC)** by Kirkpatrick et al. (2017), which introduces a penalty term to slow down changes to weights important for old tasks. *“EWC allows knowledge of previous tasks to be protected during new learning, thereby avoiding catastrophic forgetting of old abilities”* ([Overcoming catastrophic forgetting in neural networks - ar5iv - arXiv](https://ar5iv.labs.arxiv.org/html/1612.00796#:~:text=arXiv%20ar5iv,It%20does%20so%20by)). This idea of selectively preserving important parameters connects to Sutton’s notion of valuing previously learned predictions – effectively treating certain learned weights as valuable predictions that shouldn’t be overwritten without penalty.

- **Replay and Rehearsal Methods**: Inspired by the replay buffers in RL (which themselves echo Sutton’s Dyna idea of learning from stored experiences), replay-based continual learning stores samples (or generative models of past data) to intermix old and new experiences. For instance, experience replay and **generative replay** (Shin et al., 2017) train the model on both new data and pseudo-data from previous tasks to refresh its memory. These methods operationalize the idea that reusing past experience (as in off-policy RL) can mitigate forgetting.

- **Dynamic Architectures and Expansion**: Some approaches dynamically grow or adjust the model’s architecture to accommodate new tasks, rather than forcing a single static network to handle everything. **Progressive Neural Networks** (Rusu et al., 2016) grow new columns for new tasks and leverage lateral connections to old knowledge, while other methods add neurons or modules on demand. The concept of **transferable features** and **soft gating** in these models resonates with hierarchical RL (options) – retaining modules (skills) learned before and choosing when to use or adapt them. Although Sutton’s work did not explicitly add neurons over time, his options framework and skill reuse ideas provide conceptual support for building systems that accumulate modules of knowledge.

- **Meta-Learning and Few-Shot Adaptation**: Another trend is applying meta-learning so that models can *learn how to learn* continually. Techniques like continual meta-learning adjust a model’s initialization or learning rules such that it can adapt quickly to new tasks without forgetting old ones. These approaches often draw on optimization-based meta-learning, which can be traced back to ideas in RL about tuning learning processes (for example, Sutton’s work on meta-gradient RL for adjusting parameters). The integration of meta-learning with continual learning reflects a convergence of ideas: using past experience to improve future learning efficiency – a principle that is central in reinforcement learning as well.

In addition to these methods, **recent work by Sutton’s own group has directly tackled continual learning challenges in deep networks**. Notably, Hernandez-Garcia, Sutton, and colleagues (2023) identified the “loss of plasticity” phenomenon: deep networks can become resistant to learning new information after prolonged training. They demonstrated this effect on image recognition and RL tasks and underscored its importance. The abstract of their work states that a learning system *“must continue to learn indefinitely. Unfortunately, our most advanced deep-learning networks gradually lose their ability to learn”* ([Maintaining Plasticity in Deep Continual Learning - Rich Sutton](https://www.youtube.com/watch?v=p_zknyfV9fY#:~:text=Abstract%3A%20Any%20learning%20system%20worthy,learning)). By highlighting this issue, they have spurred research into methods to maintain plasticity, such as resetting certain optimizer states, using regularizers to reinvigorate learning, or architectural tweaks (e.g. LayerNorm) to prevent saturation. This is a cutting-edge area building explicitly on Sutton’s legacy – ensuring agents remain adaptable over time.

The state-of-the-art in continual learning is a vibrant mix of these strategies. No single method has completely solved continual learning, but the community has made strides by combining ideas (for example, using both replay and regularization, or meta-learning with dynamic architectures). Researchers like James Kirkpatrick, David Lopez-Paz, Sylvain Lescouz, Joelle Pineau, and many others (often in collaboration with deep learning pioneers like Geoffrey Hinton or Yoshua Bengio) are actively contributing to the field. Ongoing research trends include applying continual learning to large-scale models (e.g., keeping large language models up-to-date), exploring unsupervised continual learning, and improving benchmarks and evaluation protocols for more realistic scenarios. The influence of Sutton’s foundational work is evident throughout these advances – from the incremental learning algorithms at their core to the broader vision of agents that accumulate knowledge over a lifetime.

## 4. Theoretical and Practical Challenges
Despite significant progress, continual learning still faces major theoretical and practical challenges. A foremost issue is **catastrophic forgetting**, the tendency of neural networks to forget previously learned tasks when trained on new ones. This problem was recognized in the 1980s and *“remains a core challenge in continual learning (CL), where models struggle to retain previous knowledge”* ([Mitigating Catastrophic Forgetting in Online Continual Learning by...](https://openreview.net/forum?id=olbTrkWo1D&referrer=%5Bthe%20profile%20of%20Peilin%20Zhao%5D(%2Fprofile%3Fid%3D~Peilin_Zhao2)#:~:text=Mitigating%20Catastrophic%20Forgetting%20in%20Online,to%20retain%20previous%20knowledge)). In other words, even with methods like EWC or replay, completely eliminating forgetting is an open problem ([A Study on Catastrophic Forgetting in Deep LSTM Networks](https://www.researchgate.net/publication/335698970_A_Study_on_Catastrophic_Forgetting_in_Deep_LSTM_Networks#:~:text=Networks%20www,forgetting%20remains%20an%20open%20problem)). Each class of solution so far comes with trade-offs – for example, regularization methods can constrain learning of new tasks, while replay methods require storage or generative models. *“Despite these advances, the problem of catastrophic forgetting remains unresolved. Each proposed solution comes with trade-offs”* ([Catastrophic Forgetting // is FT isn't the answer/solution? - sbagency](https://sbagency.medium.com/catastrophic-forgetting-is-ft-isnt-the-answer-84d251edd726#:~:text=sbagency%20sbagency,offs)).

One underlying difficulty is the **stability–plasticity dilemma**: a learning system must remain stable enough to preserve old knowledge (stability) yet plastic enough to integrate new knowledge (plasticity). Balancing this trade-off is non-trivial ([[PDF] New Insights for the Stability-Plasticity Dilemma in Online Continual ...](https://iclr.cc/media/iclr-2023/Slides/11634.pdf#:~:text=%E2%80%A2%20Stability,%E2%80%A2%20The)). Too much stability and the model becomes rigid (unable to learn new tasks); too much plasticity and it quickly overwrites old knowledge. Sutton’s observation of deep networks losing plasticity ([Maintaining Plasticity in Deep Continual Learning - Rich Sutton](https://www.youtube.com/watch?v=p_zknyfV9fY#:~:text=Abstract%3A%20Any%20learning%20system%20worthy,learning)) is one side of this coin – methods are needed to restore plasticity without causing forgetting. From a theoretical standpoint, there is not yet a unifying framework that explains how to optimally navigate this stability-plasticity balance in continually learning systems.

Another challenge is the **lack of formal theoretical guarantees** in continual learning. Unlike classical machine learning, which has well-developed theories for convergence and generalization (e.g., PAC learning or online learning regret bounds), continual learning scenarios (especially with non-i.i.d. data streams and task switching) are less understood. Researchers are beginning to address this by precisely defining the continual learning problem and its objectives. For instance, recent work has attempted to *“carefully defin[e] the continual reinforcement learning problem”* and formalize agents that learn indefinitely ([A Definition of Continual Reinforcement Learning - arXiv](https://arxiv.org/html/2307.11046v2#:~:text=In%20contrast%2C%20continual%20reinforcement%20learning,the%20importance%20of%20continual)). Such definitions are a first step toward theoretical analysis, but much remains to be done to derive performance guarantees or convergence proofs for continual learning algorithms.

On the practical side, **scalability and real-world deployment** pose challenges. Many continual learning methods are evaluated on relatively small-scale benchmarks or simplified tasks. There is concern about whether these methods will scale to more complex, real-world situations (e.g. robotics, continual learning in autonomous driving, or lifelong learning in interactive agents). A recent study noted a *“misalignment between the actual challenges of continual learning and the evaluation protocols in use”* ([Is Continual Learning Ready for Real-world Challenges? - arXiv](https://arxiv.org/abs/2402.10130#:~:text=This%20paper%20contends%20that%20this,the%20evaluation%20protocols%20in%20use)) – meaning that current benchmarks might not capture real-world complexity (such as continuous task blending, ambiguous task boundaries, or need for open-world learning where new classes emerge). Bridging this gap is essential for practical impact.

Additional practical challenges include:
- **Memory and Compute Constraints**: Some algorithms require storing data from all past tasks or training separate models for each task, which is impractical as tasks accumulate. Continual learners in the wild might be embedded in edge devices with limited resources, so efficiency is key.
- **Task Recognition and Transfer**: In many settings, the boundaries between tasks are not clearly given. The agent must detect distribution shifts or new tasks on its own (the **task-agnostic continual learning** scenario). The agent should also leverage commonalities between tasks (positive transfer) without interference. Achieving strong forward transfer (learning new tasks faster because of prior knowledge) while avoiding negative backward transfer (forgetting or degrading old task performance) is an open research frontier.
- **Theoretical Understanding of Neural Mechanisms**: Catastrophic forgetting is closely linked to how connectionist models distribute knowledge. A deeper theoretical understanding of why neural networks forget (e.g., weight interference, representational overlap) would inform better solutions. Similarly, understanding the “loss of plasticity” in optimization terms (such as plateaus in the loss landscape or saturation of activations) is an ongoing theoretical quest.

Looking forward, researchers identify several **future directions** to address these challenges. Developing a *unified theory of continual learning* is one such direction – possibly extending frameworks like Markov Decision Processes (MDPs) or online learning theory to encompass multiple tasks and non-stationary data. There is also interest in biologically inspired solutions: for example, taking inspiration from how humans and animals consolidate memories during sleep or through complementary learning systems (hippocampus and cortex). Such mechanisms could inform algorithms like experience rehearsal, generative replay, or dynamic reorganization of networks to protect important memories.

In summary, continual learning must overcome enduring challenges of forgetting and stability-plasticity, scale up to realistic problems, and gain stronger theoretical underpinnings. These challenges define an exciting research agenda: each limitation of current approaches points to an opportunity for innovation, where insights from reinforcement learning, neuroscience, and other fields can converge to advance our understanding and capabilities of lifelong learning systems.

## 5. Pathways for Contribution
For a researcher new to the field, there are rich opportunities to contribute to continual learning, especially on the theoretical side. Given the nascent state of a unifying theory, one promising pathway is to work on **formal frameworks and definitions** for continual learning. Clear definitions (such as recent attempts to formally define continual RL ([A Definition of Continual Reinforcement Learning - arXiv](https://arxiv.org/html/2307.11046v2#:~:text=In%20contrast%2C%20continual%20reinforcement%20learning,the%20importance%20of%20continual))) help in deriving analysis and comparing algorithms fairly. A newcomer could contribute by refining these definitions or proposing new metrics to evaluate continual learning (e.g., better measures of forgetting and knowledge transfer, or establishing theoretical bounds on performance degradation). Aligning evaluation protocols with real-world requirements is another impact area – for instance, defining benchmarks or challenge environments that capture the complexities of continual learning (as suggested by the misalignment noted in evaluations ([Is Continual Learning Ready for Real-world Challenges? - arXiv](https://arxiv.org/abs/2402.10130#:~:text=This%20paper%20contends%20that%20this,the%20evaluation%20protocols%20in%20use))).

On the theoretical front, one could delve into **analysis of learning dynamics** in neural networks under continual learning. This might involve studying the mathematical properties of loss landscapes when tasks change, or analyzing simplified models to understand catastrophic forgetting. For example, researching why certain regularization methods succeed or fail could lead to more principled algorithms. There is also room for developing **new algorithms with provable guarantees** – perhaps borrowing techniques from online convex optimization, game theory, or control theory to ensure stability. Bridging reinforcement learning theory (which deals with non-i.i.d. data and long-term credit assignment) with continual learning is fertile ground; ideas like regret minimization in non-stationary bandits or meta-learning guarantees could inspire continual learning theory.

Interdisciplinary intersections are especially promising. A new researcher might explore **neuroscience-inspired mechanisms** in a mathematically rigorous way. For instance, mechanisms of memory consolidation, neurogenesis (growing new neurons), or synaptic gating in the brain could translate into novel neural network architectures that dynamically grow or compartmentalize knowledge. Investigating such biologically motivated approaches could address the stability-plasticity dilemma in new ways (e.g., by creating separate fast and slow learning components, analogous to hippocampus and cortex). Collaboration with cognitive scientists or neuroscientists can provide insights into how natural systems achieve lifelong learning, which in turn can spark theoretical models for artificial systems.

Another pathway is to connect continual learning with **other areas of AI** that are currently booming. For example, continual learning for **large-scale models and lifelong knowledge** is a timely topic – how can we update large language models or vision models with new information continuously, without retraining from scratch or forgetting? This intersects with transfer learning and domain adaptation. A researcher could contribute by devising methods that allow pretrained models to absorb new data over time (important for keeping AI systems up-to-date in dynamic environments). There is also an intersection with **meta-learning and automated curriculum learning**: one can study how an agent might automatically generate or select experiences to maximally retain old knowledge while learning new things (essentially, self-curation of its training data stream).

From an applications standpoint, identifying real-world problems that benefit from continual learning and demonstrating solutions there can be highly impactful. Robotics is a clear example – an autonomous robot should learn from each experience throughout its life. A newcomer might work on a specific application (say, a household robot that learns new chores incrementally, or a recommendation system that adapts to user preference shifts) and contribute algorithms tailored to that context. Such applied work often reveals new theoretical challenges too, closing the loop between practice and theory.

In terms of community and resources, the continual learning field is very open and collaborative. Engaging with workshops and conferences dedicated to lifelong learning is a great way to contribute and get feedback. Notably, the **Conference on Lifelong Learning Agents (CoLLAs)** was launched in 2022 to bring together researchers focusing on continual learning agents ([Conference on Lifelong Learning Agents (CoLLAs)](https://lifelong-ml.cc/#:~:text=The%20Conference%20on%20Lifelong%20Learning,ideas%20on%20advancing%20machine%20learning)). Top machine learning venues (NeurIPS, ICML, ICLR) regularly feature continual learning papers, and journals like *IEEE TPAMI* and *JMLR* have published surveys and special issues on the topic ([A Comprehensive Survey of Continual Learning: Theory, Method ...](https://ieeexplore.ieee.org/document/10444954/#:~:text=A%20Comprehensive%20Survey%20of%20Continual,representative%20methods%2C%20and%20practical)). For a new researcher, contributing could mean publishing innovative findings at these venues, or even simply collaborating on open-source projects (the **ContinualAI** community, for instance, curates repositories and benchmarks for continual learning).

To summarize, a newcomer can contribute to continual learning by:
- **Developing Theory**: Work on formal definitions, stability-plasticity analysis, and deriving guarantees for algorithms.
- **Innovating Algorithms**: Propose new methods (regularization techniques, memory systems, meta-learning strategies) that address current limitations.
- **Cross-Pollination**: Bring ideas from other domains (neuroscience, RL, meta-learning, even evolutionary algorithms or federated learning) to continual learning.
- **Applications and Benchmarks**: Demonstrate continual learning in new applications or create more realistic benchmarks, guiding the field toward practical relevance.
- **Community Engagement**: Participate in continual learning workshops, share findings, and build upon the work of Sutton and others by keeping the conversation between theory and practice active.

Continual learning remains a frontier with many open questions. Rich Sutton’s contributions provide a strong foundation and inspiration – emphasizing that truly intelligent systems must learn *continually*. By building on this foundation and exploring the open problems, new researchers have the opportunity to make significant theoretical and practical advances in the quest for lifelong learning AI systems. 


Below is a concise summary of the key ideas that emerge from the talks and the two papers:

1. **Standard Deep Learning and Continual Learning:**
   - **One‐Time vs. Continual Learning:** Traditional deep‐learning methods (using backpropagation with gradient descent or variants such as Adam) are designed for “one‐time” training on a fixed dataset. In many real‐world applications—such as robotics, streaming data, or online reinforcement learning—the data distribution changes over time, requiring the network to continually learn.
   - **Loss of Plasticity:** Over time, as standard training continues in a nonstationary (continual) learning setting, deep networks lose their “plasticity” (i.e. the ability to quickly adapt to new data). This loss is manifested in several ways:
     - The weights tend to grow larger.
     - A growing fraction of neurons become “dead” (or saturated), meaning that they rarely change their output.
     - The internal representations (the “feature diversity”) become less rich, as measured by a decrease in the effective rank of the hidden layers.
   - This degradation means that—even if early performance on new tasks is good—the network eventually learns no better than a shallow (or even a linear) system when faced with many successive tasks.

2. **Empirical Demonstrations:**
   - Extensive experiments were conducted on supervised tasks (e.g., variations of ImageNet, class-incremental CIFAR‑100, Online Permuted MNIST, and a “Slowly Changing Regression” problem) and reinforcement learning tasks (such as controlling an “Ant” robot with changing friction).
   - In all these settings, standard backpropagation methods initially learn well but then gradually “forget how to learn” (i.e. they lose plasticity) over hundreds or thousands of tasks.

3. **Maintaining Plasticity by Injecting Randomness:**
   - The initial random weight initialization provides many advantages (diverse features, small weights, non-saturation) that enable rapid learning early on. However, because standard backprop only applies this “randomness” at the start, these beneficial properties fade with continued training.
   - The key idea is that **continual learning requires a sustained injection of randomness or variability** to maintain plasticity.

4. **Continual Backpropagation (CBP):**
   - To counteract the decay of plasticity, the authors propose an algorithm called **Continual Backpropagation**. CBP is almost identical to standard backpropagation except that, on every update, it selectively reinitializes a very small fraction of the network’s units.
   - **Selective Reinitialization:** Using a “utility measure” that assesses how useful a neuron (or feature) is for the current task (based on factors such as its activation, its outgoing weight magnitudes, and how much it is changing), the algorithm identifies neurons that are “underused” or “dead.” These neurons are then reinitialized (with the initial small random values), thereby reintroducing diversity and the benefits of a fresh start.
   - This process—sometimes called a “generate-and-test” mechanism—allows the network to continually inject new random features without having to completely reset or lose past learning.

5. **Comparison with Other Methods:**
   - Other techniques such as L2 regularization, Shrink and Perturb (which combines weight shrinkage with noise injection), dropout, and normalization were examined.
   - Although L2 regularization and Shrink and Perturb help slow the growth of weights and partially mitigate the loss of plasticity, they are generally less robust than CBP. In some experiments (both in supervised and reinforcement learning settings), popular methods like Adam (with standard parameters), dropout, and even batch normalization actually worsened the loss of plasticity over time.

6. **Implications for Continual and Reinforcement Learning:**
   - The findings imply that if deep neural networks are to be deployed in environments where continual adaptation is necessary, the training algorithms must be modified to continuously “refresh” the network’s ability to learn.
   - In reinforcement learning, where both the environment and the agent’s behavior can change over time, the loss of plasticity is especially problematic. The continual backpropagation approach (sometimes combined with a small amount of L2 regularization) was shown to significantly improve performance in nonstationary RL tasks (for example, in controlling an ant robot in environments with changing friction).

7. **Broader Perspective:**
   - The work challenges the assumption that gradient descent alone is sufficient for deep learning in dynamic, nonstationary settings.
   - It suggests that “sustained deep learning” (learning that continues to adapt over time) may require algorithms that combine traditional gradient-based methods with mechanisms for continual variability—in effect, a built-in “refresh” mechanism similar to how biological systems continually reorganize and adapt their neural circuitry.

In summary, the key idea is that standard deep learning methods gradually lose their ability to adapt (loss of plasticity) when faced with a continual stream of new tasks. The proposed solution is to modify backpropagation so that it continuously injects new random features (through selective reinitialization of low-utility units), thereby maintaining the network’s plasticity and enabling it to learn indefinitely in nonstationary environments.


Below is a mathematical‐level explanation of the key ideas behind loss of plasticity in continual learning and the “Continual Backpropagation” (CBP) solution.

---

### 1. Standard Backpropagation and Its Limitations

A deep neural network is parameterized by weights
$$
\mathbf{w} = \{w_{l,i,k}\}
$$
where
- $l$ indexes layers,
- $i$ indexes neurons (or “features”) in layer $l$,
- $k$ indexes neurons in layer $l+1$.

**Initialization:**  
Weights are initially drawn from a “small‐random” distribution, e.g.,
$$
w_{l,i,k}(0) \sim d \quad \text{with} \quad d = \mathcal{U}(-b,b),
$$
where $b$ is chosen (e.g., via Kaiming initialization) so that the activations do not saturate.

**Gradient Descent Update:**  
For each training example (or mini‐batch), the standard update is
$$
w_{l,i,k}(t+1) = w_{l,i,k}(t) - \alpha\, \nabla_{w_{l,i,k}} L(t),
$$
where
- $\alpha$ is the learning rate,
- $L(t)$ is the loss at time $t$.

**Loss of Plasticity:**  
When training continually on a nonstationary stream of data (or a long sequence of tasks), several phenomena occur:
- **Weight Growth:** The weights tend to grow larger over time.
- **Feature Saturation / “Dead” Units:** For activations like ReLU, if a neuron’s output $h_{l,i}(x)$ is zero (or nearly so) for almost all inputs, then
  $$
  \mathbb{P}\bigl[h_{l,i}(x)=0\bigr] \approx 1,
  $$
  the neuron is “dead” and its gradient becomes zero.
- **Representation Collapse (Low Effective Rank):**  
  For a given hidden layer, let $\Phi$ be the matrix of activations across examples. The *effective rank* of $\Phi$ is defined as
  $$
  \operatorname{erank}(\Phi) = \exp\left(-\sum_{k=1}^{q} p_k \log p_k\right),\quad p_k = \frac{\sigma_k}{\sum_{j=1}^{q}\sigma_j},
  $$
  where $\sigma_1,\dots,\sigma_q$ are the singular values of $\Phi$ (with $q = \max\{n, m\}$). A decrease in $\operatorname{erank}(\Phi)$ indicates that the network’s internal representation has lost diversity.

In continual learning, it is observed that after many tasks the network’s performance (say, measured by the error $E(t)$) deteriorates—often approaching or even falling below the performance of a shallow or linear model. In symbols, one finds for standard backpropagation that
$$
\lim_{T\to\infty} E_{\text{BP}}(T) \gtrsim E_{\text{linear}},
$$
indicating a loss of the “plasticity” needed to learn new tasks.

---

### 2. A Utility Measure for Neurons

The intuition is that the “good” properties of the network—diverse, non‐saturated features with small weights—arise from the initial random distribution. To maintain these advantages over time, one can track the “utility” of each neuron and selectively refresh those that are under‐used.

**Contribution Utility:**  
For neuron $i$ in layer $l$ at time $t$, define an instantaneous measure of its contribution as:
$$
c_{l,i}(t) = \; |h_{l,i}(t)| \; \sum_{k=1}^{n_{l+1}} |w_{l,i,k}(t)|,
$$
where
- $h_{l,i}(t)$ is the neuron's output,
- $\sum_{k}|w_{l,i,k}(t)|$ measures the total “influence” of neuron $i$ on the next layer.

To smooth this over time, one can maintain a running average:
$$
c_{l,i,t} = (1-\eta)\, c_{l,i}(t) + \eta\, c_{l,i,t-1},
$$
with decay rate $\eta \in (0,1)$.

**Adaptation Utility:**  
Because the speed at which a neuron can change is also important, one may consider an “adaptation utility” inversely related to the magnitude of its incoming weights:
$$
a_{l,i}(t) = \frac{1}{\sum_{j=1}^{n_{l-1}} |w_{l-1,j,i}(t)|},
$$
or a running average thereof.

**Overall Utility:**  
A combined measure might then be given by (after bias‐correction)
$$
y_{l,i}(t) = \frac{|h_{l,i}(t) - \hat{f}_{l,i}(t)|\;\sum_{k=1}^{n_{l+1}} |w_{l,i,k}(t)|}{\sum_{j=1}^{n_{l-1}} |w_{l-1,j,i}(t)|},
$$
and its running average
$$
u_{l,i,t} = (1-\eta)\, y_{l,i}(t) + \eta\, u_{l,i,t-1}.
$$
Finally, a bias-corrected utility can be computed as
$$
\hat{u}_{l,i,t} = \frac{u_{l,i,t}}{1-\eta^{a_{l,i}}},
$$
where $a_{l,i}$ may also serve as the “age” of the neuron (i.e. the number of updates since its last reinitialization).

Low values of $\hat{u}_{l,i,t}$ indicate that the neuron is “underperforming” or has become “stale.”

---

### 3. Continual Backpropagation (CBP) Algorithm

The CBP algorithm augments standard gradient descent by periodically “refreshing” low-utility neurons. Mathematically, the CBP procedure for each layer $l$ is as follows:

1. **Standard Update:**  
   For each weight, perform the gradient update:
   $$
   w_{l,i,k}(t+1) = w_{l,i,k}(t) - \alpha\, \nabla_{w_{l,i,k}} L(t).
   $$

2. **Age Update:**  
   For each neuron $i$ in layer $l$, update its age:
   $$
   a_{l,i} \leftarrow a_{l,i} + 1.
   $$

3. **Utility Update:**  
   Update the running utility $u_{l,i,t}$ as described above.

4. **Selective Reinitialization:**  
   For each layer $l$, define a replacement fraction $\rho$ (a small number, e.g. such that on average one neuron is reinitialized every few hundred updates). Then for neurons $i$ that satisfy:
   - $a_{l,i} \ge m$ (i.e. they are “mature” enough), and
   - $\hat{u}_{l,i,t}$ is among the lowest $\rho n_l$ values,
   
   perform the following:
   - **Reset Incoming Weights:**  
     $$
     w_{l-1}[:, i] \sim d,
     $$
     i.e. re-sample the incoming weights from the original distribution.
   - **Reset Outgoing Weights:**  
     $$
     w_{l}[i, :] = 0,
     $$
     so that the new neuron does not perturb the current function.
   - **Reset Utility and Age:**  
     $$
     u_{l,i,t} \leftarrow 0,\quad a_{l,i} \leftarrow 0.
     $$

This additional “generate-and-test” step keeps a small fraction of neurons “fresh” so that the benefits of the initial randomness (small weights, diverse activations) persist indefinitely.

---

### 4. Mathematical Effects and Comparisons

**Under Standard Backpropagation:**  
- The weight magnitudes $W(t) = \frac{1}{N}\sum_{l,i,k} |w_{l,i,k}(t)|$ tend to increase with time.
- The effective rank $\operatorname{erank}(\Phi(t))$ of hidden layer activations decreases.
- The fraction of dead neurons (those for which $h_{l,i}(x)=0$ for almost all $x$) increases.

As a consequence, if $E_{\text{BP}}(T)$ denotes the error after $T$ tasks, then
$$
\lim_{T\to\infty} E_{\text{BP}}(T) \approx E_{\text{linear}},
$$
meaning the network’s performance degrades to that of a shallow model.

**Under Continual Backpropagation (CBP):**  
The periodic reinitialization maintains:
- **Bounded Weight Magnitudes:** $W(t)$ remains low.
- **High Effective Rank:** $\operatorname{erank}(\Phi(t))$ stays high, indicating diverse representations.
- **Low Fraction of Dead Units:** Most neurons remain active.
  
Thus, the error $E_{\text{CBP}}(T)$ remains low (and often improves) over many tasks:
$$
\lim_{T\to\infty} E_{\text{CBP}}(T) \ll E_{\text{BP}}(T).
$$

In reinforcement learning, where both the environment and the agent’s actions continually change the data distribution, similar mathematical effects are observed. For example, the agent’s cumulative reward $R(t)$ under standard methods may plateau or even decrease, whereas with CBP (often combined with a modest amount of L2 regularization), the reward remains high.

---

### 5. Summary

Mathematically, the key innovations are:
- Recognizing that the properties of the initial random weight distribution—small magnitude, diversity, non-saturation—are crucial for rapid adaptation.
- Defining a utility measure $u_{l,i,t}$ for each neuron that combines its contribution (via activations and outgoing weights) and its capacity to adapt (via the inverse of the incoming weights).
- Implementing a selective reinitialization rule that, when $a_{l,i} \ge m$ and $u_{l,i,t}$ is low (specifically, among the lowest $\rho$ fraction), resets the neuron's weights to reintroduce the beneficial properties of the initial state.
- This procedure mathematically maintains a low overall weight magnitude, high effective rank, and low incidence of dead neurons, thereby preserving the network’s plasticity and ensuring continued learning in nonstationary or continual learning settings.

This approach—combining standard gradient descent with a continual, selective “refresh” of low‐utility neurons—provides a mathematically grounded mechanism to overcome the loss of plasticity that plagues standard deep learning when faced with a long sequence of tasks.


================================================================================

--- Processing: ../nn_interpret.md ---
Thanks for the detailed clarifications! I will conduct research on existing academic and open-source efforts related to analyzing neural network activations, weights, and interpretability, particularly with an emphasis on:
- Tracking activations and weight changes both during and after training.
- Extensible architectures (starting from MLP but scalable to CNNs, Transformers, etc.).
- Methods for displaying raw activations and weight matrices while considering cases where the data is too large.
- Interactive vs. static approaches for understanding neural network behavior.
- Potential for an "agentic" framework, where the tool actively interacts with the model, making decisions on what to analyze next based on intermediate results.

I'll return with findings on relevant existing projects, techniques, and potential inspiration for refining the conceptual design.

# Analyzing Neural Network Activations, Weights, and Interpretability

## Tools for Tracking Activations and Weights During Training
- **TensorBoard & Logging Libraries:** Basic tools like TensorBoard can log metrics, weights, and activations at each training step, providing simple visualizations (e.g. histograms of weight magnitudes) over time. However, these often focus on summary statistics and may miss instance-specific behaviors ([Comgra: A Tool for Analyzing and Debugging Neural Networks](https://arxiv.org/html/2407.21656v1#:~:text=What%20Parts%20of%20the%20Network,these%20correlate%20with%20other%20inputs)). More advanced logging frameworks (Weights & Biases, Visdom, etc.) offer custom dashboards for tracking arbitrary values during training ([Comgra: A Tool for Analyzing and Debugging Neural Networks](https://arxiv.org/html/2407.21656v1#:~:text=Visdom,for%20different%20types%20of%20data)), but still require deciding **a priori** what to record.  
- **Comgra (2024):** An open-source library designed to flexibly inspect network parameters and intermediate tensors throughout training ([Comgra: A Tool for Analyzing and Debugging Neural Networks](https://arxiv.org/html/2407.21656v1#:~:text=Comgra.%20Comgra%20,details%20for%20a%20specific%20input)). Comgra addresses the “combinatorial explosion” of possible things to monitor by letting users interactively choose which activations or weights to explore after training, avoiding the need to rerun experiments ([Comgra: A Tool for Analyzing and Debugging Neural Networks](https://arxiv.org/html/2407.21656v1#:~:text=The%20Need%20for%20Flexibility,At)). It records selected tensors during training and provides a GUI to navigate them, balancing aggregate statistics with per-example details in one interface ([Comgra: A Tool for Analyzing and Debugging Neural Networks](https://arxiv.org/html/2407.21656v1#:~:text=layer,a%20noticeable%20loss%20in%20performance)). This ensures one can examine both *overall trends* and *specific cases* without a huge logging overhead. Usage is analogous to TensorBoard: instrument the training loop to save data, then open a browser UI for interactive exploration ([Comgra: A Tool for Analyzing and Debugging Neural Networks](https://arxiv.org/html/2407.21656v1#:~:text=Usage,as%20well%20as%20code%20examples)).  
- **TorchLens (2023):** A Python package for extracting and visualizing hidden-layer activations in PyTorch models ([TorchLens: A Python package for extracting and visualizing hidden ...](https://www.biorxiv.org/content/10.1101/2023.03.16.532916v1#:~:text=,layer%20activations%20in%20PyTorch%20models)). It makes it easy to tap into any layer of an MLP, CNN, or Transformer to record activations and even the computational graph. TorchLens provides detailed visualizations of model architectures (like an improved computation graph over TensorBoard’s basic graph) ([Comgra: A Tool for Analyzing and Debugging Neural Networks](https://arxiv.org/html/2407.21656v1#:~:text=Visualizing%20the%20Computation%20Graph,a%20toolkit%20for%20visualizing%20models)), and lets researchers **probe activations** without writing boilerplate code. This facilitates comparisons of activation patterns across layers or between two training epochs.  
- **Tracking Weight Dynamics:** Beyond activations, some tools focus on how **weight matrices** evolve. For example, WeightWatcher is a research-inspired tool that analyzes trained weight matrices using heavy-tailed spectrum metrics ([WeightWatcher: Data-Free Diagnostics for Deep Learning](https://weightwatcher.ai/#:~:text=WeightWatcher%20%28w%7Cw%29%20is%20an%20open,JMLR%2C%20Nature%20Communications%2C%20and%20NeurIPS2023)) ([WeightWatcher: Data-Free Diagnostics for Deep Learning](https://weightwatcher.ai/#:~:text=%23%20Weightwatcher%20computes%20unique%20layer,quality%20metrics)). It can flag layers that are *under-trained* or *over-regularized* via an alpha metric (ideal range ~2–6 for well-trained layers) ([WeightWatcher: Data-Free Diagnostics for Deep Learning](https://weightwatcher.ai/#:~:text=weightwatcher%20,be%20between%202%20and%206)). Such analysis can be done at checkpoints during training to see if certain layers have converged or not. Academic studies of weight dynamics (e.g. how initialization or optimization affects weight trajectories) provide theoretical insight ([New insights into training dynamics of deep classifiers | MIT News | Massachusetts Institute of Technology](https://news.mit.edu/2023/training-dynamics-deep-classifiers-0308#:~:text=The%20paper%2C%20%E2%80%9CDynamics%20in%20Deep,of%20the%20layers%20are%20intertwined)) ([New insights into training dynamics of deep classifiers | MIT News | Massachusetts Institute of Technology](https://news.mit.edu/2023/training-dynamics-deep-classifiers-0308#:~:text=fit%20a%20training%20dataset%20will,to%20accurately%20classify%20new%20examples)), but practical tools for live weight tracking tend to reduce information to summaries (means, variances, spectral norms, etc.). A well-designed framework might log weight distribution histograms over time or compute metrics like WeightWatcher’s *alpha* at each epoch to observe training progress per layer.

## Interpretability Techniques from MLPs to CNNs to Transformers
- **Attribution Libraries:** For trained models, general interpretability frameworks like **Captum** (PyTorch’s interpretability library) provide a suite of attribution methods (saliency maps, integrated gradients, etc.) to explain predictions ([Comgra: A Tool for Analyzing and Debugging Neural Networks](https://arxiv.org/html/2407.21656v1#:~:text=Attribution%20and%20Interpretability,to%20interpret%20your%20model%E2%80%99s%20behavior)). Such tools apply to MLPs, CNNs, or Transformers in a model-agnostic way, treating the network as a black box to attribute importance to inputs or neurons. This helps answer “why did this input yield that output?” by tracking influence through the network.  
- **Mechanistic Interpretability Toolkits:** A growing set of libraries focus on opening the black box of **internal mechanisms**, often starting with simple models and scaling up. For example, **TransformerLens** (Nanda & Bloom 2022) allows loading pretrained transformers and inspecting or even modifying their internals (attention patterns, layer outputs) easily ([Comgra: A Tool for Analyzing and Debugging Neural Networks](https://arxiv.org/html/2407.21656v1#:~:text=Mechanistic%20Interpretability,toolkit%20for%20sequence%20generation%20models)). Similarly, **Pyrene** and **NNSight** provide interfaces to intervene on activations or weights during runs ([Comgra: A Tool for Analyzing and Debugging Neural Networks](https://arxiv.org/html/2407.21656v1#:~:text=Mechanistic%20Interpretability,toolkit%20for%20sequence%20generation%20models)). These emerged from research in **mechanistic interpretability**, where investigators often begin with small MLPs or toy models to understand learned algorithms, then extend methods to larger CNNs and Transformers. Techniques like **network surgery** (ablating or re-weighting neurons) and **counterfactual inputs** (designed to target specific neurons) are supported in such frameworks to test hypotheses about model behavior.  
- **Architecture-General vs. Specialized Approaches:** Simpler multi-layer perceptrons (MLPs) can be analyzed with generic approaches (e.g. recording activation values, visualizing weights as heatmaps) that carry over to deeper architectures. CNNs introduce structured weights (filters) that can be visualized as images, and activations that can be seen as feature maps; accordingly, tools like **ActiVis** and **Deep Visualization Toolbox** were created to explore CNN internals in real time by showing each layer’s feature maps for a given input ([Understanding Neural Networks Through Deep Visualization](https://anhnguyen.me/project/understanding-neural-networks-through-deep-visualization/#:~:text=Understanding%20Neural%20Networks%20Through%20Deep,files%20or%20read%20video)) ([Deep Visualization Toolbox Open-source software...](https://prostheticknowledge.tumblr.com/post/123726938701/deep-visualization-toolbox-open-source-software#:~:text=software,the%20reaction%20of%20every%20neuron)). Transformers have specialized components (self-attention matrices, multi-head attention, etc.), spurring custom visualization tools like **BertViz** (for attention patterns) and libraries like **Inseq** for sequence-to-sequence models ([Comgra: A Tool for Analyzing and Debugging Neural Networks](https://arxiv.org/html/2407.21656v1#:~:text=Mechanistic%20Interpretability,toolkit%20for%20sequence%20generation%20models)). The key trend is that methods are increasingly *extensible*: an insight gained from a simple MLP (say, tracking neuron activation distributions) can be scaled to thousands of neurons in a CNN, or millions in a Transformer, with the aid of automation and visualization techniques.

## Visualizing and Summarizing Large Activations and Weights
Interpreting a network often means dealing with **high-dimensional data** – e.g. millions of activations across a dataset, or weight matrices with thousands of parameters. Researchers have developed strategies to summarize and visualize this information:

- **Activation Atlases:** Google Brain’s *Activation Atlas* technique (2019) is a prime example of summarizing large activation spaces. By applying dimensionality reduction and feature visualization to millions of intermediate activations, they create an *“explorable activation atlas”* that maps out the prominent features a network has learned ([Activation Atlas](https://distill.pub/2019/activation-atlas#:~:text=By%20using%20feature%20inversion%20to,network%20typically%20represents%20some%20concepts)). Instead of examining one input at a time, an atlas provides a global view of concepts (for instance, clusters of neurons responding to “electronics” or “animal faces” appear as regions in the map) ([Activation Atlas](https://distill.pub/2019/activation-atlas#:~:text=Above%2C%20an%20activation%20atlas%20of,screen%20version)). This kind of visualization helps show *which concepts are represented and where*, giving a big-picture understanding of a CNN’s feature space. It directly addresses the limitation that inspecting single inputs “doesn’t give us a big picture view… when what we want is a map of an entire forest, inspecting one tree at a time will not suffice” ([Activation Atlas](https://distill.pub/2019/activation-atlas#:~:text=Unfortunately%2C%20visualizing%20activations%20has%20a,a%20time%20will%20not%20suffice)).
- **Network Dissection:** *Network Dissection* (Bau et al. 2017) offers a way to compress a complex CNN’s behavior into human-readable summaries by automatically labeling neurons with semantic concepts ([[1704.05796] Network Dissection: Quantifying Interpretability of Deep Visual Representations](https://arxiv.org/abs/1704.05796#:~:text=,our%20method%20to%20compare%20the)). It uses a broad set of visual concepts (like textures, objects, parts) and checks which neurons strongly respond to those concepts in a dataset. The result is a dictionary of neurons and their likely semantic roles (e.g. “neuron 123 = detects cats”). Importantly, this framework quantifies interpretability (what fraction of neurons have clear meanings) and can be applied at different training stages ([[1704.05796] Network Dissection: Quantifying Interpretability of Deep Visual Representations](https://arxiv.org/abs/1704.05796#:~:text=latent%20representations%20of%20various%20networks,interpretability%20of%20deep%20visual%20representations)). For instance, one can see neurons gradually specialize as training proceeds ([[1704.05796] Network Dissection: Quantifying Interpretability of Deep Visual Representations](https://arxiv.org/abs/1704.05796#:~:text=latent%20representations%20of%20various%20networks,interpretability%20of%20deep%20visual%20representations)), helping track *how representational structure emerges*. This is a form of summarization: instead of showing all weights, it highlights a few salient ones with descriptions.
- **Weight Visualizations & Metrics:** For weight matrices, straightforward visualizations include heatmaps of weight values or their distributions. For CNNs, visualizing the first-layer filters gives an intuition of learned edges or color detectors; for deeper layers, tools like **Netron** and **Penzai** focus on visualizing model structures (shapes of weight tensors, connectivity) to manage complexity ([Comgra: A Tool for Analyzing and Debugging Neural Networks](https://arxiv.org/html/2407.21656v1#:~:text=Visualizing%20the%20Computation%20Graph,a%20toolkit%20for%20visualizing%20models)). Summarization metrics can distill large weight sets into numbers: e.g., the **WeightWatcher** tool computes metrics like the *alpha* exponent of layer weight spectra, condensing each layer’s quality into a single number ([WeightWatcher: Data-Free Diagnostics for Deep Learning](https://weightwatcher.ai/#:~:text=weightwatcher%20,be%20between%202%20and%206)). These numbers can be tracked across layers or over training time to find outliers (e.g. a layer whose weights are degenerate or poorly trained). Such quantitative summaries are easier to visualize (as bar charts or trend lines) than the raw weight matrices themselves.
- **Dimensionality Reduction and Embeddings:** Another approach is to embed high-dimensional activations or neurons into lower dimensions for visualization. Techniques like t-SNE or UMAP can project activation vectors (for many data points) into 2D, revealing clustering of neurons or data by similarity. For example, plotting the activations of a certain layer for thousands of inputs might show distinct clusters corresponding to classes. Similarly, one can treat each neuron as a point in a high-dimensional space (defined by its responses across many inputs) and use embedding to find groups of neurons that behave similarly. These visualizations, while not as directly interpretable as Activation Atlases, help **spot structure** in otherwise unwieldy tensors – an important step before feeding results to an interpretability assistant like ChatGPT for summarization.

## Interactive vs. Static Approaches
Interpretability tools vary in how users engage with them:

- **Interactive Tools and Frameworks:** Interactive systems allow users to pose new queries, adjust inputs, and immediately see results, which is invaluable for exploratory analysis. **ActiVis** (Facebook, 2017) is an early example: an interactive visualization system for large-scale models that integrates multiple coordinated views, such as an architecture graph and a neuron activation heatmap, enabling pattern discovery at both instance-level and subset-level ([[1704.01942] ActiVis: Visual Exploration of Industry-Scale Deep Neural Network Models](https://arxiv.org/abs/1704.01942#:~:text=participatory%20design%20sessions%20with%20over,of%20how%20ActiVis%20may%20work)). Users could select subsets of data (say, all images of “cats”) and see which neurons fire strongly, or click on a neuron to see what inputs activate it – all in a fluid UI. Similarly, Google’s open-source **Language Interpretability Tool (LIT)** provides an interactive dashboard for NLP models ([The Language Interpretability Tool (LIT): Interactive Exploration and Analysis o](https://research.google/blog/the-language-interpretability-tool-lit-interactive-exploration-and-analysis-of-nlp-models/#:~:text=With%20these%20challenges%20in%20mind%2C,extensible%20visualizations%20and%20model%20analysis)). It supports *local explanations* (e.g. salience maps on a specific sentence) and *aggregate analysis* (e.g. embedding projections of an entire dataset) side by side ([The Language Interpretability Tool (LIT): Interactive Exploration and Analysis o](https://research.google/blog/the-language-interpretability-tool-lit-interactive-exploration-and-analysis-of-nlp-models/#:~:text=LIT%20supports%20local%20explanations%2C%20including,in%20our%20system%20demonstration%20paper)). Users can generate counterfactual inputs on the fly and see how the model’s predictions and internal activations change, facilitating a tight human-in-the-loop investigation cycle ([The Language Interpretability Tool (LIT): Interactive Exploration and Analysis o](https://research.google/blog/the-language-interpretability-tool-lit-interactive-exploration-and-analysis-of-nlp-models/#:~:text=LIT%20supports%20local%20explanations%2C%20including,in%20our%20system%20demonstration%20paper)). Interactive tools typically emphasize **flexibility and drill-down**: one can start with a broad overview and then zoom into particular cases, or vice versa, to test hypotheses about model behavior in real time.  
- **Static Analysis and Reports:** In contrast, many interpretability techniques yield static outputs – think of a research paper figure showing a set of maximally activating images for several neurons, or a plot of weight distributions at epoch end. Static approaches include saliency maps or Grad-CAM heatmaps produced for a fixed set of inputs, or feature visualizations of neurons (e.g. the synthesized images that strongly activate a neuron). These are often insightful but are inherently limited to the scenarios the researcher anticipated. They don’t easily allow asking new questions of the model without going back to code. For example, a static *feature visualization* shows what one neuron likes, but if you suddenly wonder how that neuron behaves for a specific real input, you’d need to run an experiment outside of the static report. Static results are great for communication (e.g. illustrating a learned feature in a publication) and for documenting known behaviors, but they lack the ability to **adapt** to the analyst’s curiosity in the moment. Modern tools aim to bridge this gap: even Distill.pub articles often embed interactive widgets so that what starts as a “static” article becomes a playground for the reader. This trend recognizes that interpretability is often an iterative process of discovery, benefitting from interactive exploration rather than one-shot analysis.

## Agentic and Automated Analysis Approaches
A recent and exciting development is the idea of an **“agentic” interpretability tool** – one that actively decides what to inspect next, rather than just passively visualizing predetermined data. Instead of a human manually probing the network step by step, an *AI agent* can leverage intermediate findings to guide further analysis. Two notable examples:

- **MIT’s Multimodal Automated Interpretability Agent (MAIA, 2024):** This system uses a pretrained language model equipped with a suite of tools to conduct interpretability research on neural nets ([A Multimodal Automated Interpretability Agent](https://arxiv.org/html/2404.14394v1#:~:text=This%20paper%20describes%20maia%2C%20a,of%20maia%20to%20computer%20vision)) ([A Multimodal Automated Interpretability Agent](https://arxiv.org/html/2404.14394v1#:~:text=trained%20vision,Across%20several%20trained%20models%20and)). In essence, MAIA behaves like a research scientist: it can synthesize new inputs (for example, generate images or edit text) to test what causes a particular neuron or sub-network to activate, find real dataset examples that maximally activate a component, and then *summarize its observations in natural language*. The agent iteratively forms hypotheses (“Neuron X seems to respond to *cars*”), designs experiments to verify them (e.g. feed images of cars, planes, boats to see if it fires only on cars), and refines its understanding based on results ([AI agents help explain other AI systems | MIT News | Massachusetts Institute of Technology](https://news.mit.edu/2024/ai-agents-help-explain-other-ai-systems-0103#:~:text=For%20example%2C%20FIND%20contains%20synthetic,The)) ([AI agents help explain other AI systems | MIT News | Massachusetts Institute of Technology](https://news.mit.edu/2024/ai-agents-help-explain-other-ai-systems-0103#:~:text=Sarah%20Schwettmann%20PhD%20%2721%2C%20co,%E2%80%9D)). Notably, MAIA was able to produce neuron descriptions for vision models that were comparable to human experts’ descriptions ([A Multimodal Automated Interpretability Agent](https://arxiv.org/html/2404.14394v1#:~:text=experimentation%20on%20subcomponents%20of%20other,truth)) ([A Multimodal Automated Interpretability Agent](https://arxiv.org/html/2404.14394v1#:~:text=Interpretability%20experiments%20proposed%20by%20maia,classified.%E2%80%A1)). It also tackled higher-level tasks like identifying biases or spurious features by actively searching for inputs that trigger those behaviors. This agentic approach demonstrates that language models (with the right tooling) can go beyond static summarization – they can *actively explore* a neural network, which is a promising direction for complex models that are too large for exhaustive manual probing ([AI agents help explain other AI systems | MIT News | Massachusetts Institute of Technology](https://news.mit.edu/2024/ai-agents-help-explain-other-ai-systems-0103#:~:text=Central%20to%20this%20strategy%20is,other%20systems%20in%20real%20time)) ([AI agents help explain other AI systems | MIT News | Massachusetts Institute of Technology](https://news.mit.edu/2024/ai-agents-help-explain-other-ai-systems-0103#:~:text=examples%2C%20the%20AIA%20actively%20participates,other%20systems%20in%20real%20time)).
- **OpenAI’s Automated Neuron Explanations (2023):** OpenAI researchers recently explored using GPT-4 to explain neurons in GPT-2 ([OpenAI is Using GPT-4 to Explain Neurons' Behavior in GPT-2 - InfoQ](https://www.infoq.com/news/2023/05/openai-gpt4-explains-gpt2/#:~:text=A%20recent%20paper%20by%20a,step%20towards%20automating%20DNN%20interpretability)). Their method is a three-step loop: (1) **Explain** – prompt GPT-4 with examples of a neuron’s top activations and ask it to hypothesize in plain English what the neuron looks for; (2) **Simulate** – have GPT-4 (or another model) predict the neuron’s activation on a wide range of inputs based on that hypothesis; (3) **Score** – compare the simulated activations against the actual neuron activations to see how well the explanation holds up ([OpenAI is Using GPT-4 to Explain Neurons' Behavior in GPT-2 - InfoQ](https://www.infoq.com/news/2023/05/openai-gpt4-explains-gpt2/#:~:text=A%20recent%20paper%20by%20a,step%20towards%20automating%20DNN%20interpretability)) ([OpenAI is Using GPT-4 to Explain Neurons' Behavior in GPT-2 - InfoQ](https://www.infoq.com/news/2023/05/openai-gpt4-explains-gpt2/#:~:text=In%20the%20first%20step%2C%20a,find%20phrases%20related%20to%20community)). For example, GPT-4 might guess *“Neuron 245 activates for phrases about community or gatherings”* ([OpenAI is Using GPT-4 to Explain Neurons' Behavior in GPT-2 - InfoQ](https://www.infoq.com/news/2023/05/openai-gpt4-explains-gpt2/#:~:text=In%20the%20first%20step%2C%20a,find%20phrases%20related%20to%20community)). The system then checks this by seeing if Neuron 245 indeed fires on words like *“team, group,”* etc., and not on unrelated words. If the match is good, the explanation is validated; if not, the process can iterate with a refined prompt. This approach effectively uses an LLM as an *analyst* that both proposes and evaluates interpretability hypotheses. It’s “agentic” in the sense that the AI is taking on tasks a human analyst would do – generating candidate explanations and testing them – all in an automated loop. While currently focused on individual neurons in language models, the method could extend to analyzing entire circuits or interactions between neurons. It highlights how a ChatGPT-like model can be harnessed as a powerful interpretability assistant, leveraging its world knowledge to articulate what a pattern in activations might represent, and its generation capabilities to design experiments.

These agent-driven methods are at the frontier of interpretability research. They marry the strengths of deep learning (pattern recognition and generation) with the investigative process of science. Crucially, they can scale analysis in ways humans alone might struggle with, by quickly sifting through thousands of neurons and zeroing in on the interesting ones with proposed meanings ([AI agents help explain other AI systems | MIT News | Massachusetts Institute of Technology](https://news.mit.edu/2024/ai-agents-help-explain-other-ai-systems-0103#:~:text=For%20example%2C%20FIND%20contains%20synthetic,and%20not%20air%20or%20sea)) ([AI agents help explain other AI systems | MIT News | Massachusetts Institute of Technology](https://news.mit.edu/2024/ai-agents-help-explain-other-ai-systems-0103#:~:text=Sarah%20Schwettmann%20PhD%20%2721%2C%20co,%E2%80%9D)). The “FIND” benchmark introduced alongside the AIA work ([AI agents help explain other AI systems | MIT News | Massachusetts Institute of Technology](https://news.mit.edu/2024/ai-agents-help-explain-other-ai-systems-0103#:~:text=Complementing%20the%20AIA%20method%20is,g)) even provides ground-truth functions to systematically evaluate how well such agents explain known computations – a sign that this approach is maturing.

## Inspirations for a ChatGPT-Based Neural Network Analysis Tool
Bringing these findings together, we can envision a new neural network analysis tool powered by ChatGPT (or similar LLMs) that leverages the best of both worlds: human-friendly dialogue and powerful automated analysis. Key design inspirations include:

- **Logging and UI from Training to Inference:** Like Comgra and TorchLens, the tool should allow tracking of any activation or weight of interest during training and afterward. The **flexibility** emphasized by Comgra – to choose between individual activations vs. summary stats, different time points, and different inputs on the fly ([Comgra: A Tool for Analyzing and Debugging Neural Networks](https://arxiv.org/html/2407.21656v1#:~:text=What%20Parts%20of%20the%20Network,these%20correlate%20with%20other%20inputs)) ([Comgra: A Tool for Analyzing and Debugging Neural Networks](https://arxiv.org/html/2407.21656v1#:~:text=The%20Need%20for%20Flexibility,At)) – suggests our ChatGPT-based tool must be able to fetch *both* granular data (e.g. “What were the layer-2 weights at epoch 5 vs epoch 50?”) and high-level summaries (“How did the weight distribution change?”) on demand. A possible implementation is a back-end logging system that records extensive data (perhaps guided by heuristics to keep it manageable), which ChatGPT can query via an API. The ChatGPT interface can then present insights in natural language, supplemented by small charts or tables, much like a conversational TensorBoard. This marries the interactive exploration of training dynamics with an LLM’s ability to summarize and explain those dynamics in plain English.
- **Model-Agnostic Analysis Modules:** To handle MLPs, CNNs, and Transformers uniformly, the tool can draw on ideas from Captum and mechanistic interpretability libraries. For instance, it could have a **“saliency probe”** that ChatGPT can invoke to compute attributions for a given input and model, or a **“activation extractor”** for any layer. By wrapping these techniques in an API, ChatGPT could say, *“I will compute which features most influenced this output”*, call an attribution method, and then explain the results. The tool’s architecture might thus be an *agent* (ChatGPT) orchestrating various modules (for attribution, activation visualization, etc.), similar to how MAIA uses a library of interpretability tools ([A Multimodal Automated Interpretability Agent](https://arxiv.org/html/2404.14394v1#:~:text=rather%20than%20labeling%20features%20in,sweeps%20over%20entire%20networks%2C%20or)) ([A Multimodal Automated Interpretability Agent](https://arxiv.org/html/2404.14394v1#:~:text=modular%20design%20of%20the%20tool,we%20use%20the%20following%20set)). Starting with simple tests on an MLP (e.g. *“Does neuron 4 fire for positive numbers?”*), the same agent could seamlessly scale up to a ResNet or Transformer, because it can query appropriate modules (e.g. attention pattern analyzer for Transformers, filter visualizer for CNNs). The **extensible design** of LIT – where new components can be added for new model types ([The Language Interpretability Tool (LIT): Interactive Exploration and Analysis o](https://research.google/blog/the-language-interpretability-tool-lit-interactive-exploration-and-analysis-of-nlp-models/#:~:text=Customizability)) – is a good blueprint for keeping the tool relevant as architectures evolve.
- **Handling Large Data via Summarization:** The challenge of large activations and weight matrices can be tackled by combining visualization techniques with ChatGPT’s summarization capabilities. For example, the tool might internally generate an Activation Atlas for a particular layer  ([Activation Atlas](https://distill.pub/2019/activation-atlas#:~:text=By%20using%20feature%20inversion%20to,network%20typically%20represents%20some%20concepts)), then have ChatGPT *interpret the atlas*: *“Layer 5 appears to have clusters for ‘building structures’ and ‘foliage textures’, indicating specialized feature detectors.”* By automating techniques like dimensionality reduction or concept labeling (à la Network Dissection), the tool can feed ChatGPT higher-level descriptors instead of raw numbers. ChatGPT’s strength in language means it could take a set of neuron labels or a graph of neuron clusters and produce a coherent narrative: *“Early layers differentiate mainly simple edges, while later layers in the CNN have neurons grouped into semantically rich concepts (faces, text, etc.) ([Activation Atlas](https://distill.pub/2019/activation-atlas#:~:text=Above%2C%20an%20activation%20atlas%20of,screen%20version)). Many neurons are devoted to texture in layer 3, which might explain why the model is texture-biased.”* Similarly, for weights, ChatGPT could report: *“Layer 10’s weight matrix has a heavy-tailed distribution (alpha ~7), which WeightWatcher suggests is a sign of under-training ([WeightWatcher: Data-Free Diagnostics for Deep Learning](https://weightwatcher.ai/#:~:text=weightwatcher%20,be%20between%202%20and%206)). This might be a weak link in the network’s performance.”* These kinds of interpretations directly draw from research insights and make them accessible.
- **Interactive Conversational Interface:** Inspired by interactive tools like ActiVis and LIT, the new tool’s interface is conversational but could also include rich media. A user might ask, *“Show me how the activations of layer 2 changed during training”*, and ChatGPT could present a small trend plot of activation means or a description: *“Layer 2’s activation variance increased and then stabilized after epoch 10, suggesting it learned a diversified set of features early on.”* The user could then ask, *“Which neurons in layer 2 are most active for class ‘cat’ images?”*, and the agent would fetch that info (perhaps by scanning the dataset) and reply with an answer and possibly an embedded image of those neurons’ feature visualizations. This *interactive Q&A* style makes analysis accessible – you don’t need to write code or dig through logs; you can simply ask questions about the network’s internals. It’s essentially ChatGPT acting as a knowledgeable guide through the model, powered by real data.
- **Automated “Agentic” Investigations:** Taking a cue from MAIA and OpenAI’s neuron explainer, the tool could have an *autonomous mode* where it performs a series of analyses by itself and reports findings. For example, upon loading a new model, the ChatGPT agent could systematically: scan for neurons with high variance, generate hypotheses about their roles (perhaps by retrieving the top activating inputs and asking itself what they have in common), and then present the user with a summary: *“I noticed neuron 87 in layer 5 consistently activates for images with text – it might be an ‘OCR/text detector’ unit. Neuron 21 in layer 7 seems to pick up on dog faces. There are 10 neurons in layer 9 that together respond to different colors, suggesting color-sensitive features.”* Each of these findings would be backed by evidence the agent gathered (which the user could ask to see, e.g. “show me examples”). The agent can also identify problematic behaviors: *“I tested the model on counterfactual inputs and found that flipping gendered words changes the output significantly, indicating a bias.”* By proactively looking for such patterns, the tool can surface critical insights without the user needing to know exactly what to ask. This *exploratory analysis mode* is directly inspired by the success of automated interpretability agents ([AI agents help explain other AI systems | MIT News | Massachusetts Institute of Technology](https://news.mit.edu/2024/ai-agents-help-explain-other-ai-systems-0103#:~:text=Central%20to%20this%20strategy%20is,other%20systems%20in%20real%20time)) ([AI agents help explain other AI systems | MIT News | Massachusetts Institute of Technology](https://news.mit.edu/2024/ai-agents-help-explain-other-ai-systems-0103#:~:text=Sarah%20Schwettmann%20PhD%20%2721%2C%20co,%E2%80%9D)) and would be a standout feature of a ChatGPT-powered framework.

In summary, recent academic research and open-source projects chart a clear path toward a more intelligent neural network analysis tool. From **tracking training dynamics** (Comgra, TorchLens) to **visualizing internals** (ActiVis, LIT) and **automating interpretability** (MAIA, OpenAI’s GPT-4 explainer), each provides pieces of the puzzle. A new tool based on ChatGPT APIs can combine these pieces: it would log and visualize like existing frameworks, but crucially, also **converse and reason** about the network’s behavior. By doing so, it directly leverages the strengths identified in the literature – flexibility in data inspection ([Comgra: A Tool for Analyzing and Debugging Neural Networks](https://arxiv.org/html/2407.21656v1#:~:text=The%20Need%20for%20Flexibility,At)), multi-scale visualization ([[1704.01942] ActiVis: Visual Exploration of Industry-Scale Deep Neural Network Models](https://arxiv.org/abs/1704.01942#:~:text=participatory%20design%20sessions%20with%20over,of%20how%20ActiVis%20may%20work)), conceptual summarization ([[1704.05796] Network Dissection: Quantifying Interpretability of Deep Visual Representations](https://arxiv.org/abs/1704.05796#:~:text=,our%20method%20to%20compare%20the)), and autonomous hypothesis generation ([AI agents help explain other AI systems | MIT News | Massachusetts Institute of Technology](https://news.mit.edu/2024/ai-agents-help-explain-other-ai-systems-0103#:~:text=examples%2C%20the%20AIA%20actively%20participates,other%20systems%20in%20real%20time)) – to help users deeply understand their neural networks in an intuitive, human-centered way. The result would be an AI-powered “copilot” for neural network interpretability, turning state-of-the-art research ideas into practical tooling for model developers. 

**Sources:**

1. Dietz, F. *et al.* (2023). *Comgra: A Tool for Analyzing and Debugging Neural Networks* ([Comgra: A Tool for Analyzing and Debugging Neural Networks](https://arxiv.org/html/2407.21656v1#:~:text=The%20Need%20for%20Flexibility,At)) ([Comgra: A Tool for Analyzing and Debugging Neural Networks](https://arxiv.org/html/2407.21656v1#:~:text=Comgra.%20Comgra%20,details%20for%20a%20specific%20input))  
2. Taylor, J.M. & Kriegeskorte, N. (2023). *TorchLens: A Python package for extracting and visualizing hidden activations* ([Comgra: A Tool for Analyzing and Debugging Neural Networks](https://arxiv.org/html/2407.21656v1#:~:text=Visualizing%20the%20Computation%20Graph,a%20toolkit%20for%20visualizing%20models))  
3. Kokhlikyan, N. *et al.* (2020). *Captum: A unified model interpretability library for PyTorch* ([Comgra: A Tool for Analyzing and Debugging Neural Networks](https://arxiv.org/html/2407.21656v1#:~:text=Attribution%20and%20Interpretability,to%20interpret%20your%20model%E2%80%99s%20behavior))  
4. Carter, S. *et al.* (2019). *Activation Atlas (Distill)* ([Activation Atlas](https://distill.pub/2019/activation-atlas#:~:text=By%20using%20feature%20inversion%20to,network%20typically%20represents%20some%20concepts)) ([Activation Atlas](https://distill.pub/2019/activation-atlas#:~:text=Unfortunately%2C%20visualizing%20activations%20has%20a,a%20time%20will%20not%20suffice))  
5. Bau, D. *et al.* (2017). *Network Dissection: Quantifying Interpretability of Deep Visual Representations* ([[1704.05796] Network Dissection: Quantifying Interpretability of Deep Visual Representations](https://arxiv.org/abs/1704.05796#:~:text=,our%20method%20to%20compare%20the)) ([[1704.05796] Network Dissection: Quantifying Interpretability of Deep Visual Representations](https://arxiv.org/abs/1704.05796#:~:text=latent%20representations%20of%20various%20networks,interpretability%20of%20deep%20visual%20representations))  
6. Kahng, M. *et al.* (2018). *ActiVis: Visual Exploration of Industry-Scale DNNs* ([[1704.01942] ActiVis: Visual Exploration of Industry-Scale Deep Neural Network Models](https://arxiv.org/abs/1704.01942#:~:text=participatory%20design%20sessions%20with%20over,of%20how%20ActiVis%20may%20work))  
7. Tenney, I. *et al.* (2020). *Language Interpretability Tool (LIT)* ([The Language Interpretability Tool (LIT): Interactive Exploration and Analysis o](https://research.google/blog/the-language-interpretability-tool-lit-interactive-exploration-and-analysis-of-nlp-models/#:~:text=LIT%20supports%20local%20explanations%2C%20including,in%20our%20system%20demonstration%20paper))  
8. Rott Shaham, T. *et al.* (2024). *MAIA: A Multimodal Automated Interpretability Agent* ([AI agents help explain other AI systems | MIT News | Massachusetts Institute of Technology](https://news.mit.edu/2024/ai-agents-help-explain-other-ai-systems-0103#:~:text=Central%20to%20this%20strategy%20is,other%20systems%20in%20real%20time)) ([AI agents help explain other AI systems | MIT News | Massachusetts Institute of Technology](https://news.mit.edu/2024/ai-agents-help-explain-other-ai-systems-0103#:~:text=For%20example%2C%20FIND%20contains%20synthetic,and%20not%20air%20or%20sea))  
9. OpenAI (2023). *Language Models can Explain Neurons in Language Models* ([OpenAI is Using GPT-4 to Explain Neurons' Behavior in GPT-2 - InfoQ](https://www.infoq.com/news/2023/05/openai-gpt4-explains-gpt2/#:~:text=A%20recent%20paper%20by%20a,step%20towards%20automating%20DNN%20interpretability)) ([OpenAI is Using GPT-4 to Explain Neurons' Behavior in GPT-2 - InfoQ](https://www.infoq.com/news/2023/05/openai-gpt4-explains-gpt2/#:~:text=In%20the%20first%20step%2C%20a,find%20phrases%20related%20to%20community))  
10. Martin, C. *et al.* (2023). *WeightWatcher: Data-Free Diagnostics for Deep Neural Networks*

================================================================================

--- Processing: ../rank_deficiency.md ---
Thanks for the clarification! I will look into theoretical and semi-theoretical results on conditions that lead to stable rank deficiencies in neural networks, particularly in the middle or penultimate layers. The focus will be on:

- Rank collapse or stable low-rank representations under training.
- Loss functions like cross-entropy and MSE.
- The role of nonlinear activations (especially ReLU, sigmoid).
- The effect of standard stochastic gradient descent (SGD).
- Scenarios relevant to overparameterized MLPs.

I will get back to you with findings on theoretical guarantees or conditions that enforce such stable rank deficiencies.

# Stable Rank Deficiencies in Hidden Layers of Neural Networks

**Overview:** Deep neural networks often learn **low-rank representations** in their hidden layers, especially in later layers. Theory and empirical evidence suggest that as training progresses (and with sufficient overparameterization), the **effective rank** of hidden-layer activations or weight matrices can remain low or even *collapse*, rather than utilizing the full potential rank. Below, we outline key factors and findings that explain **when and why hidden layers maintain a stable, low rank** during training, focusing on theoretical conditions, loss functions, activation types, SGD dynamics, and overparameterization. 

## Monotonic Rank Reduction in Deep Networks (Theoretical Conditions)

**Composition Limits Rank:** A fundamental theoretical insight is that the **rank of a composed function (multiple layers)** cannot increase with depth – it tends to *monotonically decrease* or stay the same ([](https://proceedings.neurips.cc/paper_files/paper/2022/file/d5cd70b708f726737e2ebace18c3f71b-Paper-Conference.pdf#:~:text=rank%2C%20focusing%20particularly%20on%20the,MLPs%2C%20and%20Transformers%20on%20ImageNet)). In other words, as inputs propagate through layers, the dimension of the learned feature manifold is non-increasing. Any layer’s output is constrained by the information in previous layers, so once a representation loses rank (becomes low-dimensional), subsequent layers cannot recover the lost dimensions. Feng *et al.* (2022) formally prove a **“universal monotonic decreasing property”** of network rank based on differential and algebraic composition rules ([](https://proceedings.neurips.cc/paper_files/paper/2022/file/d5cd70b708f726737e2ebace18c3f71b-Paper-Conference.pdf#:~:text=rank%2C%20focusing%20particularly%20on%20the,MLPs%2C%20and%20Transformers%20on%20ImageNet)). This means **if a hidden layer becomes rank-deficient, that deficiency tends to persist or worsen in deeper layers**. For example, common operations like pooling, downsampling, or even standard fully-connected layers can significantly **drop the rank** of their outputs ([](https://proceedings.neurips.cc/paper_files/paper/2022/file/d5cd70b708f726737e2ebace18c3f71b-Paper-Conference.pdf#:~:text=decreasing%20comes%20from%20the%20structural,layer%20can%20loose%20ranks%20considerably)). This structural constraint helps explain why certain middle or penultimate layers might **lock in a low rank** — once features collapse onto a low-dimensional subspace, later layers can only use those dimensions (barring injection of new information).

**Intrinsic Data Constraints:** Theoretical conditions for low rank also relate to the data and function being learned. If the target function or data manifold is inherently low-dimensional, a network might **find a low-rank representation that suffices**. In fact, the rank of a layer’s output (viewed as a function) measures the “volume of independent information” it carries ([](https://proceedings.neurips.cc/paper_files/paper/2022/file/d5cd70b708f726737e2ebace18c3f71b-Paper-Conference.pdf#:~:text=In%20mathematics%2C%20the%20rank%20of,learning%20that%20underlies%20many%20tasks)). Under mild assumptions, an optimal network will not inflate rank beyond what’s needed to represent the underlying data structure. This connects to ideas like the **Information Bottleneck**, where networks compress intermediate representations. Patel & Shwartz-Ziv (2024) define a “local rank” measure of feature-manifold dimensionality and show that hidden-layer rank tends to **decrease in later training, forming an emergent bottleneck** ([Learning to Compress: Local Rank and Information Compression in Deep Neural Networks](https://arxiv.org/html/2410.07687v1#:~:text=Deep%20neural%20networks%20tend%20to,This%20work%20bridges%20the%20gap)) ([Learning to Compress: Local Rank and Information Compression in Deep Neural Networks](https://arxiv.org/html/2410.07687v1#:~:text=particularly%20multilayer%20perceptrons%20,2023a)). In summary, **deep networks naturally compress information**, and theory indicates that **once a layer’s rank becomes low (e.g. due to a bottleneck or saturating behavior), it remains stably low or further collapses as training continues** ([](https://proceedings.neurips.cc/paper_files/paper/2022/file/d5cd70b708f726737e2ebace18c3f71b-Paper-Conference.pdf#:~:text=rank%2C%20focusing%20particularly%20on%20the,MLPs%2C%20and%20Transformers%20on%20ImageNet)) ([](https://proceedings.neurips.cc/paper_files/paper/2022/file/d5cd70b708f726737e2ebace18c3f71b-Paper-Conference.pdf#:~:text=decreasing%20comes%20from%20the%20structural,layer%20can%20loose%20ranks%20considerably)).

## Implicit Low-Rank Bias in Overparameterized Models

**Gradient Descent Favors Low Rank:** When neural networks are *overparameterized* (more parameters than data or than strictly needed), there are infinitely many solutions that fit the training data. Theory shows that standard training algorithms have an **implicit bias toward “simpler” (lower-complexity) solutions**, often reflected in low-rank structure. In particular, for deep *linear* networks (a simplified case with no nonlinear activations), it’s proven that gradient descent (or flow) on the squared loss converges to solutions with minimal **effective rank**. Gunasekar *et al.* (2017) demonstrated that a depth-2 linear network trained on a matrix factorization task converges to the minimum nuclear-norm solution ([](https://minyoungg.github.io/overparam/resources/overparam-v2.pdf#:~:text=Notable%20work%20from%20Arora%20et,%282020%3B%202021%29%20argues)) – essentially the **lowest-rank weight matrix** that fits the data. More generally, later works showed that gradient descent tends to act as a *greedy rank minimizer* in linear matrix factorization problems ([](https://minyoungg.github.io/overparam/resources/overparam-v2.pdf#:~:text=Notable%20work%20from%20Arora%20et,%282020%3B%202021%29%20argues)). In deep linear networks, this means if the training data can be fit by a low-rank mapping, gradient-based optimization will converge to that low-rank solution, leaving the weight matrices **rank-deficient** (many singular values driven to zero). This implicit bias arises *without* any explicit rank regularization – it’s a property of the dynamics of overparameterized models.

**Depth Amplifies Rank Bias:** Depth itself contributes to the bias. Adding more hidden layers (even linear ones) can strengthen the tendency toward low-rank solutions. Arora *et al.* (2019) found that in deep linear networks, **singular values of the effective mapping decay faster with increased depth**, indicating that deeper architectures induce a stronger preference for concentrating information in a few directions ([](https://minyoungg.github.io/overparam/resources/overparam-v2.pdf#:~:text=solutions,rank%20solutions.%20Pennington)). In practice, **deeper networks often end up using only a subset of their neurons or directions effectively**, yielding low-rank feature matrices in intermediate and penultimate layers. Empirically, researchers observed that simply increasing depth (without changing the training objective) biases the network toward learning embeddings with lower rank ([](https://minyoungg.github.io/overparam/resources/overparam-v2.pdf#:~:text=work%20on%20over%02parameterization%20and%20highlights,matrix%20has%20a%20low%201)). In fact, random initialized deep models already tend to map data into a low-rank feature space (as measured by the Gram matrix of features) and this bias remains after training ([](https://minyoungg.github.io/overparam/resources/overparam-v2.pdf#:~:text=work%20on%20over%02parameterization%20and%20highlights,matrix%20has%20a%20low%201)). This “**low-rank simplicity bias**” means that among the many possible solutions in an overparameterized setting, deep networks prefer ones where features live in a smaller subspace ([](https://minyoungg.github.io/overparam/resources/overparam-v2.pdf#:~:text=hypothesis%20that%20deeper%20networks%20are,wide%20variety%20of%20commonly%20used)) ([](https://minyoungg.github.io/overparam/resources/overparam-v2.pdf#:~:text=gelu%20relu%20leaky%20relu%20tanh,matrix%20on%20a%20variety%20of)). Notably, this bias appears *robust*: it occurs across different initializations, hyperparameters, and even different optimizers ([](https://minyoungg.github.io/overparam/resources/overparam-v2.pdf#:~:text=simplicity%20bias%20exists%20even%20after,Practically%2C%20we)). In short, **overparameterization (especially increased depth) combined with gradient-based training often leads to persistent rank deficiencies** in hidden layers – the network finds a solution that uses far fewer independent dimensions than the layer width, and stays there.

**Permanent Rank Collapse:** Once a network converges to such a low-rank solution, those rank deficiencies tend to be “permanent” in the sense that continued training doesn’t reintroduce dropped dimensions. Instead, extended training can further reinforce the collapsed structure. For example, in overparameterized classifiers, one often observes **Neural Collapse** at the penultimate layer: features for each class collapse to their class mean, and those means become maximally spaced in a $C$-dimensional simplex (for $C$ classes). This implies the penultimate layer’s output has rank at most $C$ (much lower than its potential dimension). Importantly, this collapsed configuration emerges *in the late stage of training* and then remains stable ([Are All Losses Created Equal: A Neural Collapse Perspective | OpenReview](https://openreview.net/forum?id=8rfYWE3nyXl#:~:text=a%20recent%20line%20work%20showing,produce%20equivalent)). Any extra degrees of freedom (e.g. additional neurons) simply align redundantly with this low-dimensional structure rather than expanding it. Thus, in highly overparameterized Multi-Layer Perceptrons (MLPs), it’s common to see entire directions in weight space or neuron activations effectively unused – **the network has more capacity than it needs, and gradient descent naturally finds a solution that leaves a “rank gap”**. The theoretical and empirical works above provide conditions for this: **if a low-rank solution exists (e.g. data lies on a low-dimensional manifold, or fewer than $N$ independent features are needed to classify $N$ classes), an overparameterized network will often converge to that solution, making the hidden-layer rank deficit permanent** ([](https://minyoungg.github.io/overparam/resources/overparam-v2.pdf#:~:text=Notable%20work%20from%20Arora%20et,%282020%3B%202021%29%20argues)) ([Are All Losses Created Equal: A Neural Collapse Perspective | OpenReview](https://openreview.net/forum?id=8rfYWE3nyXl#:~:text=a%20recent%20line%20work%20showing,produce%20equivalent)). In summary, **overparameterization doesn’t increase the utilized rank – paradoxically, it often ensures some neurons or directions are redundant, locking in a low-rank representation** for middle and penultimate layers.

## Loss Functions (Cross-Entropy vs. MSE) and Rank Stability

**Cross-Entropy and Neural Collapse:** The choice of loss function can influence training dynamics and feature geometry, but common losses in classification (cross-entropy) and regression (mean squared error) ultimately can lead to similar low-rank feature outcomes. Cross-entropy (with softmax output) is known to drive networks into the **“neural collapse”** regime during the terminal phase of training, especially in classification tasks. Under cross-entropy loss, as training error approaches zero, the network continues to sharpen the separations between classes: penultimate layer features for each class become nearly identical (collapsed to their mean), and different class means maximize their mutual distances in feature space. This is a highly symmetric, low-rank configuration (features span roughly a $C-1$ dimensional subspace for $C$ classes). Papyan *et al.* (2020) observed this phenomenon empirically, and it has since been analyzed theoretically. In particular, for sufficiently large networks trained to convergence on cross-entropy, **the only global minimizers are those exhibiting neural collapse** ([Are All Losses Created Equal: A Neural Collapse Perspective | OpenReview](https://openreview.net/forum?id=8rfYWE3nyXl#:~:text=a%20recent%20line%20work%20showing,produce%20equivalent)). That means the **optimal solution inherently has low-rank feature structure** (the rank of the class-feature matrix equals the number of classes, not the number of features). Any features beyond that subspace are essentially unused.

**MSE Loss and Other Losses:** One might suspect that mean squared error (MSE) loss, which doesn’t push outputs to extremes the way cross-entropy does, might behave differently. However, recent theoretical work shows that **MSE loss can also lead to neural collapse at optimality** for overparameterized models. Zhou *et al.* (2022) compare cross-entropy vs. MSE and find that **both losses (and even variants like label smoothing or focal loss) yield the same neural collapse structure in the learned features**, given a large enough network trained to minimal loss ([Are All Losses Created Equal: A Neural Collapse Perspective | OpenReview](https://openreview.net/forum?id=8rfYWE3nyXl#:~:text=a%20recent%20line%20work%20showing,produce%20equivalent)) ([Are All Losses Created Equal: A Neural Collapse Perspective | OpenReview](https://openreview.net/forum?id=8rfYWE3nyXl#:~:text=global%20solution%20and%20landscape%20analyses,FL%20loss%20near%20the%20optimal)). In other words, the global optimum features under MSE classification loss still have all samples of a class coincident at the class mean, and class means maximally separated (simplex configuration). This result implies that **the low-rank collapse of penultimate-layer features is not specific to cross-entropy** – it is a property of the *classification problem* itself at the optimum, rather than the particular loss formula ([Are All Losses Created Equal: A Neural Collapse Perspective | OpenReview](https://openreview.net/forum?id=8rfYWE3nyXl#:~:text=a%20recent%20line%20work%20showing,produce%20equivalent)). Empirically, networks trained with either cross-entropy or MSE (on the same classification task) tend to end up with very similar penultimate-layer geometry and test performance, as long as they are sufficiently overparameterized and trained long enough ([Are All Losses Created Equal: A Neural Collapse Perspective | OpenReview](https://openreview.net/forum?id=8rfYWE3nyXl#:~:text=model%20assumption%2C%20we%20provide%20either,large%20and%20trained%20until%20convergence)).

**Dynamics Differences:** While the *final outcomes* under cross-entropy and MSE can be similar (both can collapse features to a low-rank configuration), the **training dynamics** may differ. Cross-entropy is an **exponential-type loss** that continues to penalize even tiny classification errors, which often drives weights to grow in norm and features to become extremely pure (one-hot like probabilities). This can encourage faster or more pronounced collapse of features during the later stages of training. In contrast, MSE (for classification) treats the problem more like regression to one-hot targets; once the network fits the training points, there’s no additional push to exaggerate the features. As a result, some studies noted that **neural collapse emerges more clearly or earlier with cross-entropy** (which implicitly maximizes class separation margin), whereas with MSE, collapse may still occur but perhaps requires more epochs or stronger overparameterization to mirror the same effect ([Are All Losses Created Equal: A Neural Collapse Perspective | OpenReview](https://openreview.net/forum?id=8rfYWE3nyXl#:~:text=a%20recent%20line%20work%20showing,produce%20equivalent)). Nonetheless, in either case the penultimate layer ends up **low-rank (approximately rank = number of classes)** at convergence. For regression tasks, MSE can also lead to low-rank internal representations if the target function is low-dimensional. Overall, common loss functions like cross-entropy and MSE *do not prevent* rank deficiency in hidden layers; at optimum they often **demand it**, by driving the network toward a simplified, structured solution (neural collapse being a prime example) ([Are All Losses Created Equal: A Neural Collapse Perspective | OpenReview](https://openreview.net/forum?id=8rfYWE3nyXl#:~:text=a%20recent%20line%20work%20showing,produce%20equivalent)).

## Activation Functions (ReLU vs. Sigmoid) and Rank Collapse

**ReLU Networks:** The choice of activation affects how information flows and whether it’s preserved or squashed, which in turn influences rank. **ReLU (Rectified Linear Unit)** activations are piecewise-linear and can introduce **dead neurons or linear dependencies** that reduce rank. For example, if a ReLU neuron’s input is negative for all training samples, that neuron outputs all zeros – effectively removing one dimension from that layer’s output (a rank drop). Even when active, ReLUs output either a linear scaling of the input or zero, so large groups of neurons can end up encoding redundant directions (especially if their weight vectors are correlated). From a theoretical perspective, understanding rank in nonlinear networks is harder than in linear ones. Recent work by Timor *et al.* (2023) shows that in contrast to linear networks, **gradient flow on ReLU networks doesn’t always minimize rank** – in fact, they construct scenarios where a shallow ReLU network does *not* find the lowest-rank solution ([](https://proceedings.mlr.press/v201/timor23a/timor23a.pdf#:~:text=what%20extent%20this%20gen%02eralizes%20to,solutions%20in%20several%20reasonable%20settings)). This is a “negative result” indicating that ReLU’s piecewise linearity can sometimes preserve or create just enough variation to avoid trivial rank collapse in small cases. **However, on the positive side, deeper ReLU networks **are** biased toward low-rank solutions in many settings ([](https://proceedings.mlr.press/v201/timor23a/timor23a.pdf#:~:text=datasets%20of%20size%202%29,solutions%20in%20several%20reasonable%20settings)).** In other words, depth appears to restore the implicit rank minimization tendency even with ReLUs. Sufficiently deep ReLU MLPs have been proven to favor low-rank function mappings under certain assumptions ([](https://proceedings.mlr.press/v201/timor23a/timor23a.pdf#:~:text=datasets%20of%20size%202%29,solutions%20in%20several%20reasonable%20settings)), aligning with the empirical findings discussed earlier (depth-driven low-rank bias). Moreover, experiments show that the phenomenon of rank reduction with depth holds **across a variety of activation functions** – ReLU included. Rosenfeld *et al.* (2021) observed that increasing the number of layers **decreases the effective rank of the feature matrix for ReLU networks**, just as it does for smooth activations ([](https://minyoungg.github.io/overparam/resources/overparam-v2.pdf#:~:text=gelu%20relu%20leaky%20relu%20tanh,matrix%20on%20a%20variety%20of)). In summary, ReLU’s on-off behavior can cause **rank collapse by deactivating neurons or duplicating features**, and while a shallow ReLU net might not automatically minimize rank, a deep ReLU net trained with SGD still tends to learn a compressed (low-rank) representation in later layers.

**Sigmoid and Saturating Activations:** **Sigmoid or tanh activations** (smooth, squashing non-linearities) have their own influence on rank. These functions saturate at extreme values (output approaching 0 or 1 for sigmoid, -1 or 1 for tanh), which can effectively **flatten variations** in input. During training, it is often observed that later in training (or in early layers of very deep networks), many sigmoid/tanh neurons enter saturation for a wide range of inputs. A neuron stuck near 0 or 1 for all inputs contributes almost no meaningful variability – it’s nearly a constant output, reducing the rank of that layer’s activation matrix (similar to a dead or saturated unit). This behavior ties in with the **Information Bottleneck (IB) theory**: Tishby and colleagues (2017) reported that networks with saturating activations show phases of training where **mutual information between the layer and the input drops**, implying the layer is discarding information and compressing its representation. This compression often corresponds to many neurons saturating, hence fewer effective degrees of freedom (a lower-dimensional manifold of activations). In practice, sigmoid networks were found to undergo an initial fitting phase followed by a **compression phase**, where the hidden-layer information (and empirically, the variance or entropy of activations) collapses significantly. This suggests that **sigmoid/tanh networks may exhibit an even stronger rank-collapse tendency in later layers** compared to ReLU, since saturation can make large portions of the layer’s output almost constant. Indeed, Patel & Shwartz-Ziv (2024) note that networks compress their feature manifolds in later training, and this was originally observed in saturating networks consistent with IB predictions ([Learning to Compress: Local Rank and Information Compression in Deep Neural Networks](https://arxiv.org/html/2410.07687v1#:~:text=Deep%20neural%20networks%20tend%20to,This%20work%20bridges%20the%20gap)).

**Activation-Type Comparisons:** Empirical studies directly comparing activation functions find that **the trend of low-rank representations is quite general**. For instance, one study computed the “effective rank” of the feature Gram matrix for networks with ReLU, leaky ReLU, tanh, GELU, and even sinusoidal activations, across various depths. The result was universal: **adding more layers consistently lowered the effective rank of the penultimate-layer features for all activation types tested** ([](https://minyoungg.github.io/overparam/resources/overparam-v2.pdf#:~:text=gelu%20relu%20leaky%20relu%20tanh,matrix%20on%20a%20variety%20of)). While the absolute level of rank may differ (e.g. some activations might retain slightly more information in shallow layers), the *qualitative behavior* – a bias toward low-rank, structured representations – appears across the board ([](https://minyoungg.github.io/overparam/resources/overparam-v2.pdf#:~:text=gelu%20relu%20leaky%20relu%20tanh,matrix%20on%20a%20variety%20of)) ([](https://minyoungg.github.io/overparam/resources/overparam-v2.pdf#:~:text=depth%20of%20the%20model,By%20hierarchically)). Therefore, common nonlinearities like ReLU and sigmoid both permit rank-deficient solutions. ReLU may allow networks to retain more piecewise-linear information in some cases (not *always* minimizing rank), but in deep and overparameterized regimes it still converges to low-rank feature mappings. Sigmoidal activations naturally encourage compression via saturation, often leading to **“rank collapse” as training goes on** (neurons saturate and outputs cluster). In both cases, once neurons either saturate or become redundant, those dimensions effectively drop out of the model’s representation. No matter the activation, a deep network that has more capacity than needed will tend to **only use a few dominant directions** in each layer’s output – yielding a stable, low-rank representation by the penultimate layer.

## Role of SGD and Training Dynamics in Rank Deficiency

**SGD as an Implicit Regularizer:** Interestingly, the stochastic nature of training (stochastic gradient descent, SGD) itself plays a role in enforcing low-rank structures. Recent theoretical analyses point out that **mini-batch SGD injects noise that biases the solution toward low-complexity (low-rank) weights**. Galanti & Poggio (2022) proved that when training deep networks with small-batch SGD (especially with common tweaks like weight decay), the noise in the gradients creates an *implicit constraint* that favors low-rank solutions ([](https://cbmm.mit.edu/sites/default/files/publications/Implicit%20Rank%20Minimization.pdf#:~:text=We%20analyze%20deep%20ReLU%20neural,Our%20analysis)). In fact, they show the very source of SGD noise can be viewed as a form of *rank regularization*: all else equal, SGD tends to drive the weight matrices to smaller effective rank over training ([](https://cbmm.mit.edu/sites/default/files/publications/Implicit%20Rank%20Minimization.pdf#:~:text=We%20analyze%20deep%20ReLU%20neural,Our%20analysis)). Their theory connects batch size, learning rate, and weight decay to the rank of the learned weight matrices ([](https://cbmm.mit.edu/sites/default/files/publications/Implicit%20Rank%20Minimization.pdf#:~:text=%E2%80%A2%20In%20Thm,weight%20decay%2C%20optimization%20and%20rank)). Specifically, **smaller batch sizes and the use of weight decay act as strong low-rank regularizers** ([](https://cbmm.mit.edu/sites/default/files/publications/Implicit%20Rank%20Minimization.pdf#:~:text=%E2%80%A2%20In%20Thm,weight%20decay%2C%20optimization%20and%20rank)). Intuitively, gradient noise continually perturbs the solution within the space of zero training error solutions, and it preferentially guides the weights toward configurations that are “simpler” (analogous to how noise in linear regression can favor solutions with smaller norm). In deep networks, that simplicity manifests as weight matrices with many singular values effectively zero – i.e. **SGD implicitly pushes toward rank deficiency** in each layer’s weights ([](https://cbmm.mit.edu/sites/default/files/publications/Implicit%20Rank%20Minimization.pdf#:~:text=We%20analyze%20deep%20ReLU%20neural,Our%20analysis)). This complements the implicit bias of gradient descent itself; even without noise, gradient descent finds a low-rank solution in many cases, and **with noise (SGD) this bias is even more pronounced**.

**Empirical Evidence in SGD Dynamics:** Empirically, one finds that SGD-trained networks often learn features in a stage-wise fashion, capturing the most significant structure first. Early in training, networks latch onto the largest variance or easiest features in the data (sometimes called *spectral bias* or *dominant feature first* learning). As Pezeshki *et al.* (2020) observed, **SGD tends to learn statistically dominant features first, which leads to learning low-rank solutions** for the data ([](https://minyoungg.github.io/overparam/resources/overparam-v2.pdf#:~:text=compression%2C%20Guo%20et%20al,an%20implicit%20regularizer%20during%20training)). In practical terms, this means the network might initially increase the rank of representations to fit the data variation, but once it has fit the major patterns, additional training **compresses the representation**, aligning with those dominant patterns and ignoring minor ones. This is consistent with a reduction in rank as training continues. Some works also note that different optimization algorithms (SGD vs. adaptive methods like Adam) yield similar low-rank phenomena in deep models ([](https://minyoungg.github.io/overparam/resources/overparam-v2.pdf#:~:text=simplicity%20bias%20exists%20even%20after,Practically%2C%20we)), suggesting the effect is not solely due to SGD’s noise but also due to the parameterization. Nevertheless, **SGD’s stochastic noise reinforces symmetry-breaking and flat minima selection** that often coincides with low-rank weight configurations. The *theorem* by Galanti & Poggio even implies that as long as there is some SGD noise (e.g. mini-batches) and weight decay, the training will *never exactly converge* but instead keep hovering around a solution, effectively preventing the network from utilizing extra degrees of freedom that aren’t needed ([](https://cbmm.mit.edu/sites/default/files/publications/Implicit%20Rank%20Minimization.pdf#:~:text=only%20assumed%20to%20be%20differentiable,rank)) ([](https://cbmm.mit.edu/sites/default/files/publications/Implicit%20Rank%20Minimization.pdf#:~:text=%E2%80%A2%20In%20Sec,4)). In other words, the noise keeps the model near a minimal-rank solution. 

**Breaking vs. Enforcing Rank Deficiency:** One might wonder if SGD noise could ever *break* a rank deficiency (for example, jostle the network out of a bad symmetric solution where two neurons are identical, thereby increasing rank). In practice, SGD **does break exact symmetric degeneracies** (it’s rare for two neurons to remain perfectly identical during SGD training because tiny gradient differences will separate them). However, the net effect of SGD’s randomness is not to maximize rank, but rather to explore solutions of similar performance and favor the ones with smoother or simpler structure. If a certain low-rank configuration suffices to fit the data, SGD is unlikely to kick the network into a higher-rank regime without a clear benefit. In fact, small-batch SGD will add noise that, on average, drives the weights toward the flat region of the loss landscape that often corresponds to compressive solutions ([](https://cbmm.mit.edu/sites/default/files/publications/Implicit%20Rank%20Minimization.pdf#:~:text=%E2%80%A2%20In%20Thm,weight%20decay%2C%20optimization%20and%20rank)). Thus, **SGD more often enforces rank deficiency than alleviates it**. The overall training dynamic typically sees the rank of hidden-layer activations **initially high (when learning diverse features), then stabilizing or decreasing as training converges**, especially under SGD. This aligns with the “compression phase” idea and has been measured directly: for example, Patel (2024) found that the *local rank* of features in each layer dropped during the final phase of training, indicating SGD was fine-tuning the model by compressing representations further ([Learning to Compress: Local Rank and Information Compression in Deep Neural Networks](https://arxiv.org/html/2410.07687v1#:~:text=Deep%20neural%20networks%20tend%20to,This%20work%20bridges%20the%20gap)) ([Learning to Compress: Local Rank and Information Compression in Deep Neural Networks](https://arxiv.org/html/2410.07687v1#:~:text=dimensionality%20and%20demonstrate%2C%20both%20theoretically,information%20bottlenecks%20and%20representation%20learning)). In summary, **SGD (with typical settings) implicitly regularizes the network toward low-rank solutions**, making rank-deficient hidden layers a persistent outcome of the training process ([](https://cbmm.mit.edu/sites/default/files/publications/Implicit%20Rank%20Minimization.pdf#:~:text=We%20analyze%20deep%20ReLU%20neural,Our%20analysis)) ([](https://minyoungg.github.io/overparam/resources/overparam-v2.pdf#:~:text=compression%2C%20Guo%20et%20al,an%20implicit%20regularizer%20during%20training)).

## Overparameterized MLPs and Conditions for Rank Collapse

**Excess Capacity Leads to Redundancy:** In multi-layer perceptrons with more neurons or layers than necessary, certain conditions practically guarantee rank deficiencies. If an MLP has layers much wider than the intrinsic dimension of the problem, it has the freedom to realize a solution where many neurons are simply not needed. **Gradient descent will often utilize only a subset of neurons (or a subset of independent directions in those neurons)** to solve the task, leaving the rest effectively redundant. For instance, if an MLP could solve a task with an internal representation of dimension $d$, giving it $d \times 2$ neurons in that layer doesn’t force it to use all $2d$ directions – it may well use only $d$ of them (making the layer’s output rank $d$). The other neurons might end up as linear combinations of the first $d$ or stuck at zero weights. **This is a common scenario for “permanent” rank deficiency**: the network finds a low-rank configuration early (or by midpoint of training) and never needs to activate the extra capacity. Once those extra neurons settle into redundancy (e.g., duplicating another neuron’s behavior or outputting near-constant), the rank of that layer stays low. Any slight perturbation (like SGD noise) doesn’t overcome the bias to keep them redundant, because deviating would not improve the loss.

**Bottleneck Layers and Architecture:** Some networks explicitly include bottleneck layers (fewer neurons) in the middle by design; those obviously enforce low rank at that point. But even in **uniformly wide networks, an “effective bottleneck” can emerge**. Rangamani *et al.* (2023) empirically found that as one goes deeper into a trained classifier, the **within-class variability of features shrinks and class means become the dominant components** ([Feature Learning in Deep Classifiers through Intermediate Neural Collapse](https://proceedings.mlr.press/v202/rangamani23a/rangamani23a.pdf#:~:text=these%20properties%20extend%20to%20intermediate,class%20means%20aligns%20with%20the)). In effect, by the penultimate layer, most of the variation in features is between classes rather than within – which means the representation has roughly one degree of freedom per class (a very low rank structure) ([Feature Learning in Deep Classifiers through Intermediate Neural Collapse](https://proceedings.mlr.press/v202/rangamani23a/rangamani23a.pdf#:~:text=these%20properties%20extend%20to%20intermediate,class%20means%20aligns%20with%20the)). This happens even if every hidden layer had the same width; the *network learned* to create an information bottleneck near the end. Such **self-induced bottlenecks** are a hallmark of overparameterized models: they have enough layers/neurons to first separate the classes and then compress each class cluster tightly. The penultimate layer in these cases is severely rank-deficient (often close to rank $C$ for $C$ classes). Notably, this low-rank state is maintained – it doesn’t revert or expand – even if training continues longer (the clusters just tighten further). Conditions that encourage this include having **much more model capacity than the minimum required**, and optimizing to near-zero training loss (so the network can afford to project data onto a structured low-dimensional subspace that cleanly separates classes).

**Explicit Regularization and Rank:** It’s worth mentioning that some explicit regularization techniques can also encourage low-rank solutions (e.g. weight decay, which is commonly used, biases toward smaller weights that often imply fewer independent components). However, what we’ve discussed are *implicit* phenomena – even in the absence of explicit rank penalties, overparameterized MLPs tend to end up with stable rank deficiencies. In fact, adding too strong an explicit rank penalty is often unnecessary or even harmful; simply relying on the network’s inductive biases and SGD tends to find a good low-rank solution on its own ([](https://minyoungg.github.io/overparam/resources/overparam-v2.pdf#:~:text=maps%20to%20low,parameters%2C%20and)) ([](https://minyoungg.github.io/overparam/resources/overparam-v2.pdf#:~:text=shows%20that%20linear%20models%20with,demonstrate%20the%20practical%20applications%20of)). Researchers have demonstrated that **deliberately overparameterizing a model (especially in depth) can improve generalization by leveraging this implicit low-rank bias**, rather than explicitly constraining rank ([](https://minyoungg.github.io/overparam/resources/overparam-v2.pdf#:~:text=optimizers,parameterizing%20standard%20models%20at%20training)). For example, inserting extra linear layers (which add depth but no new nonlinearity) during training was shown to yield lower-rank features and better generalization, even though the model’s theoretical capacity didn’t change ([](https://minyoungg.github.io/overparam/resources/overparam-v2.pdf#:~:text=optimizers,parameterizing%20standard%20models%20at%20training)). This underscores that **overparameterization itself (when coupled with SGD) is a form of regularization** that often realizes as low-rank internal representations.

**Summary of Conditions:** In an overparameterized MLP, you are likely to see **permanent rank collapse** in hidden layers when: (1) the model has significantly more parameters than needed to fit the data, (2) the training is run to near convergence (zero or negligible training error), and (3) standard losses (like cross-entropy or MSE) and optimizers (SGD or similar) are used. Under these conditions, theoretical and empirical studies indicate the network will converge to a solution where hidden layers (especially the penultimate layer) have **stable, low rank** – often determined by the problem’s inherent dimensionality (such as number of classes or principal components of the data). The **rank remains low throughout the end of training** because neither the architecture nor the training dynamics provide an incentive to reinflate it. On the contrary, deep architectures and SGD training both *favor* collapsing dimensions and finding efficient, low-dimensional encodings ([](https://proceedings.neurips.cc/paper_files/paper/2022/file/d5cd70b708f726737e2ebace18c3f71b-Paper-Conference.pdf#:~:text=rank%2C%20focusing%20particularly%20on%20the,MLPs%2C%20and%20Transformers%20on%20ImageNet)) ([](https://cbmm.mit.edu/sites/default/files/publications/Implicit%20Rank%20Minimization.pdf#:~:text=We%20analyze%20deep%20ReLU%20neural,Our%20analysis)). This is why we observe phenomena like neural collapse and rank-deficient features in practice. In conclusion, hidden-layer rank deficiency in neural networks is backed by several semi-theoretical and theoretical insights: **deep compositions naturally restrict rank, gradient descent implicitly seeks low-rank solutions (especially in overparameterized setups), common loss functions do not oppose (and in fact often drive) rank collapse, nonlinear activations (ReLU or sigmoid) still end up compressing information, and SGD’s stochastic nature further encourages simplicity**. All these factors together explain **when/why a network’s middle or penultimate layers might maintain a low, stable rank despite ongoing training** ([](https://proceedings.neurips.cc/paper_files/paper/2022/file/d5cd70b708f726737e2ebace18c3f71b-Paper-Conference.pdf#:~:text=rank%2C%20focusing%20particularly%20on%20the,MLPs%2C%20and%20Transformers%20on%20ImageNet)) ([Are All Losses Created Equal: A Neural Collapse Perspective | OpenReview](https://openreview.net/forum?id=8rfYWE3nyXl#:~:text=a%20recent%20line%20work%20showing,produce%20equivalent)) ([](https://cbmm.mit.edu/sites/default/files/publications/Implicit%20Rank%20Minimization.pdf#:~:text=We%20analyze%20deep%20ReLU%20neural,Our%20analysis)).

**References:**

- R. Feng *et al.*, *“Rank Diminishing in Deep Neural Networks,”* NeurIPS 2022 – establishes that network rank **monotonically decreases** with depth and analyzes rank deficiencies per layer ([](https://proceedings.neurips.cc/paper_files/paper/2022/file/d5cd70b708f726737e2ebace18c3f71b-Paper-Conference.pdf#:~:text=rank%2C%20focusing%20particularly%20on%20the,MLPs%2C%20and%20Transformers%20on%20ImageNet)) ([](https://proceedings.neurips.cc/paper_files/paper/2022/file/d5cd70b708f726737e2ebace18c3f71b-Paper-Conference.pdf#:~:text=decreasing%20comes%20from%20the%20structural,layer%20can%20loose%20ranks%20considerably)).  
- S. Gunasekar *et al.*, *“Implicit Regularization in Matrix Factorization,”* NeurIPS 2017 – shows gradient descent on overparametrized linear models converges to **minimum-nuclear-norm (low-rank)** solutions ([](https://minyoungg.github.io/overparam/resources/overparam-v2.pdf#:~:text=Notable%20work%20from%20Arora%20et,%282020%3B%202021%29%20argues)).  
- J. Zhou *et al.*, *“Are All Losses Created Equal? A Neural Collapse Perspective,”* NeurIPS 2022 – proves that both cross-entropy and MSE losses (and others) yield **Neural Collapse** at global optima, implying low-rank penultimate features for sufficiently large networks ([Are All Losses Created Equal: A Neural Collapse Perspective | OpenReview](https://openreview.net/forum?id=8rfYWE3nyXl#:~:text=a%20recent%20line%20work%20showing,produce%20equivalent)).  
- N. Timor *et al.*, *“Implicit Regularization Towards Rank Minimization in ReLU Networks,”* ALT 2023 – finds that while shallow ReLU nets may not always minimize rank, **deeper ReLU nets are biased towards low-rank solutions** under gradient flow ([](https://proceedings.mlr.press/v201/timor23a/timor23a.pdf#:~:text=what%20extent%20this%20gen%02eralizes%20to,solutions%20in%20several%20reasonable%20settings)).  
- T. Galanti & T. Poggio, *“SGD Noise and Implicit Low-Rank Bias in Deep Neural Networks,”* CBMM Memo 2022 – theoretically shows mini-batch **SGD + weight decay imposes a low-rank constraint** on weight matrices; smaller batches and higher weight decay strengthen this effect ([](https://cbmm.mit.edu/sites/default/files/publications/Implicit%20Rank%20Minimization.pdf#:~:text=We%20analyze%20deep%20ReLU%20neural,Our%20analysis)) ([](https://cbmm.mit.edu/sites/default/files/publications/Implicit%20Rank%20Minimization.pdf#:~:text=%E2%80%A2%20In%20Thm,weight%20decay%2C%20optimization%20and%20rank)).  
- A. Rangamani *et al.*, *“Feature Learning in Deep Classifiers through Intermediate Neural Collapse,”* ICML 2023 – empirically demonstrates that **intermediate layers progressively collapse** class-wise: deeper layers have much lower within-class variance (effectively lower rank) relative to between-class variance ([Feature Learning in Deep Classifiers through Intermediate Neural Collapse](https://proceedings.mlr.press/v202/rangamani23a/rangamani23a.pdf#:~:text=these%20properties%20extend%20to%20intermediate,class%20means%20aligns%20with%20the)).  
- M. Rosenfeld *et al.* (OpenReview preprint 2021), *“The Low-Rank Simplicity Bias in Deep Networks,”* – provides empirical evidence that **increasing depth consistently reduces the effective rank** of learned features across various architectures and activations, and that this bias is robust to different optimizers ([](https://minyoungg.github.io/overparam/resources/overparam-v2.pdf#:~:text=gelu%20relu%20leaky%20relu%20tanh,matrix%20on%20a%20variety%20of)) ([](https://minyoungg.github.io/overparam/resources/overparam-v2.pdf#:~:text=simplicity%20bias%20exists%20even%20after,Practically%2C%20we)).  
- Additional references: S. Arora *et al.* 2019; B. Pezeshki *et al.* 2020; H. Papyan *et al.* 2020; and others as cited above, which further support these points on rank collapse and implicit biases in deep learning ([](https://minyoungg.github.io/overparam/resources/overparam-v2.pdf#:~:text=solutions,rank%20solutions.%20Pennington)) ([](https://minyoungg.github.io/overparam/resources/overparam-v2.pdf#:~:text=compression%2C%20Guo%20et%20al,an%20implicit%20regularizer%20during%20training)) ([Are All Losses Created Equal: A Neural Collapse Perspective | OpenReview](https://openreview.net/forum?id=8rfYWE3nyXl#:~:text=a%20recent%20line%20work%20showing,produce%20equivalent)).

================================================================================






================================================================================

--- Processing: ../.ipynb_checkpoints/CL_RS-checkpoint.md ---

# Rich Sutton's Research Contributions to Continual Learning

## 1. Foundational Contributions
Rich Sutton has played a pivotal role in shaping reinforcement learning (RL), laying groundwork that also underpins continual learning. Many of his seminal contributions introduced algorithms and frameworks for agents to learn incrementally from ongoing experience – a core aspect of continual learning. Key foundational contributions include:

- **Temporal-Difference (TD) Learning**: Sutton’s 1988 work on TD learning introduced a method for an agent to update predictions by bootstrapping from newer estimates rather than waiting for final outcomes ([Reinforcement Learning: Temporal Difference (TD) Learning](https://www.lancaster.ac.uk/stor-i-student-sites/jordan-j-hood/2021/04/12/reinforcement-learning-temporal-difference-td-learning/#:~:text=Learning%20www,as%20the%20name%20suggests%2C)). TD learning merged strengths of dynamic programming and Monte Carlo methods, enabling effective incremental learning of value functions. This concept is *“likely the most core concept in Reinforcement Learning”* ([Reinforcement Learning: Temporal Difference (TD) Learning](https://www.lancaster.ac.uk/stor-i-student-sites/jordan-j-hood/2021/04/12/reinforcement-learning-temporal-difference-td-learning/#:~:text=Learning%20www,as%20the%20name%20suggests%2C)) and allowed agents to learn from a continuous stream of data, a prerequisite for continual adaptation.

- **Dyna Architecture**: In 1991, Sutton proposed the Dyna architecture, an integrated approach combining learning, planning, and reacting. In Dyna, an agent learns a world model online and uses it for simulated experience (planning) alongside real experience ([[PDF] Integrated Modeling and Control Based on Reinforcement Learning](https://papers.nips.cc/paper/1990/file/d9fc5b73a8d78fad3d6dffe419384e70-Paper.pdf#:~:text=,Dyna%20is%20based%20on)). *“Dyna architectures are those that learn a world model online while using approximations to [dynamic programming] to learn and plan optimal behavior”* ([[PDF] Integrated Modeling and Control Based on Reinforcement Learning](https://papers.nips.cc/paper/1990/file/d9fc5b73a8d78fad3d6dffe419384e70-Paper.pdf#:~:text=,Dyna%20is%20based%20on)). This framework was foundational for continual learning, as it showed how an agent could keep learning and improving by reusing past knowledge (via the learned model) in an ongoing way.

- **Options Framework (Hierarchical RL)**: Sutton (with Precup and Singh, 1999) introduced the options framework, which defines *“temporally extended ways of behaving”* (options) in RL ([[PDF] Temporal Abstraction in Temporal-difference Networks](http://papers.neurips.cc/paper/2826-temporal-abstraction-in-temporal-difference-networks.pdf#:~:text=%5BPDF%5D%20Temporal%20Abstraction%20in%20Temporal,and%20about%20predictions%20of)). Options are higher-level actions or skills that consist of lower-level primitives, with policies and termination conditions. This framework allows an agent to learn and reuse skills across tasks, effectively providing a form of knowledge transfer and memory over time. *“Generalization of one-step actions to option models… enables an agent to predict and reason at multiple time scales”* ([[PDF] Temporally Abstract Partial Models - OpenReview](https://openreview.net/pdf?id=LGvlCcMgWqb#:~:text=,reason%20at%20multiple%20time%20scales)), which is crucial for continual learning scenarios where the agent must build on prior skills.

- **General Value Functions and the Horde Architecture**: In more recent work, Sutton and colleagues developed the concept of General Value Functions (GVFs) and the Horde architecture (2011) for learning many predictions in parallel. Horde is a framework with a “democracy” of prediction-learning processes (termed “demons”) each learning a GVF about the agent’s sensorimotor stream. Sutton’s team demonstrated that an agent can scale to *“learn multiple pre-defined objectives in parallel”* and accumulate predictive knowledge continuously ([[1206.6262] Scaling Life-long Off-policy Learning - arXiv](https://arxiv.org/abs/1206.6262#:~:text=We%20build%20on%20our%20prior,to%20represent%20a%20wide)). In their words, *“GVFs have been shown able to represent a wide [range of predictions]”* in a lifelong learning setting ([[1206.6262] Scaling Life-long Off-policy Learning - arXiv](https://arxiv.org/abs/1206.6262#:~:text=We%20build%20on%20our%20prior,to%20represent%20a%20wide)). This idea of learning many predictions simultaneously without forgetting earlier ones directly informs continual learning research.

Sutton’s foundational work, including the widely used RL textbook (Sutton & Barto, 1998), established core algorithms and principles (e.g. incremental updates, bootstrapping, and exploration strategies) that enable an agent to learn continually. These contributions introduced formalisms and tools – such as TD error, experience replay (used later in deep RL and continual learning), and function approximation techniques – that remain central in modern continual learning research.

## 2. Relation to Reinforcement Learning
Continual learning and reinforcement learning are deeply intertwined, and Sutton’s work bridges them both conceptually and methodologically. Reinforcement learning deals with agents learning from an ongoing stream of interactions with an environment, which naturally aligns with the idea of *continual* learning (learning that never truly stops). Sutton himself has emphasized the importance of agents that keep learning over time. For example, continual reinforcement learning has been defined as the setting in which an agent *“never stop[s] learning”* ([A Definition of Continual Reinforcement Learning - arXiv](https://arxiv.org/html/2307.11046v2#:~:text=In%20contrast%2C%20continual%20reinforcement%20learning,the%20importance%20of%20continual)), highlighting that the best agents are those that can learn indefinitely. This ethos is a direct reflection of Sutton’s lifelong advocacy for incremental, online learning in RL.

Several RL principles introduced or popularized by Sutton have influenced continual learning algorithms:
- **Online Incremental Updates**: Methods like TD learning and gradient-descent updates allow learning to happen incrementally with each new observation, rather than in large batches. This is essential for continual learning, where data arrives sequentially. Sutton’s algorithms (e.g. TD(λ), SARSA, Q-learning refinements) showed how an agent can update knowledge on the fly and revisit old predictions efficiently, which is also how continual learning systems update without retraining from scratch.
- **Experience Replay and Off-Policy Learning**: While not invented solely by Sutton, the idea of reusing past experiences (experience replay) in RL (pioneered by Lin and later used in DQN) connects to rehearsal strategies in continual learning. Off-policy learning algorithms (such as Q-learning or off-policy TD) studied by Sutton enable learning from older data or from hypothetical trajectories (as in Dyna) ([[PDF] Integrated Modeling and Control Based on Reinforcement Learning](https://papers.nips.cc/paper/1990/file/d9fc5b73a8d78fad3d6dffe419384e70-Paper.pdf#:~:text=,Dyna%20is%20based%20on)), analogous to how rehearsal or memory replay methods mitigate forgetting in continual learning.
- **Exploration and Non-Stationarity**: RL deals with non-stationary data distributions when an agent’s policy changes or the environment changes. Sutton’s work on exploration strategies and non-stationary value functions (e.g. in continuing tasks) provides insight into continual learning, where the data distribution can shift over time (new tasks or contexts). Techniques ensuring stability in RL (like eligibility traces and stable function approximation) help inspire mechanisms to balance stability and plasticity in continual learning.

Importantly, Sutton has argued that the traditional ML focus on training static models (what he calls “non-continual learning”) is limiting. He suggests that solving AI requires agents that learn and adapt continually in the long run. In a recent interview, he *“argues the focus on non-continual learning over the past 40 years is now holding AI back”* ([Rich Sutton's new path for AI - Audacy](https://www.audacy.com/podcast/approximately-correct-an-ai-podcast-from-amii-d6257/episodes/rich-suttons-new-path-for-ai-4a1fa#:~:text=Rich%20Sutton%27s%20new%20path%20for,is%20now%20holding%20AI%20back)). In other words, many successes in ML (e.g. deep learning on fixed datasets) may plateau unless we embrace continual learning principles inherent in the RL paradigm. This perspective has encouraged researchers to apply RL-based thinking (like continual exploration, reward-driven adaptation, and lifelong skill acquisition) to broader continual learning problems.

The influence of Sutton’s RL work is also evident in how continual learning researchers design their algorithms. For example, the formal definition of continual reinforcement learning in recent literature ([A Definition of Continual Reinforcement Learning - arXiv](https://arxiv.org/html/2307.11046v2#:~:text=In%20contrast%2C%20continual%20reinforcement%20learning,the%20importance%20of%20continual)) echoes Sutton’s vision of an *always-learning* agent. Overall, Sutton’s reinforcement learning contributions provide both the theoretical foundation and practical algorithms that continual learning research builds upon, underscoring that an agent’s knowledge should *accumulate and adapt over its entire lifetime* rather than being learned once and for all.

## 3. Recent Advances
Continual learning has seen rapid progress in recent years, spurred in part by the deep learning revolution and by the principles established by Sutton and others. Researchers have proposed various strategies to enable neural networks to learn sequentially without forgetting past knowledge. Many of these advances can be seen as elaborations of ideas present in RL or directly influenced by Sutton’s insights (such as using regularization to protect learned knowledge or replaying experiences). Notable recent developments include:

- **Regularization-Based Methods**: These methods add constraints to the learning process to prevent catastrophic forgetting. A prime example is **Elastic Weight Consolidation (EWC)** by Kirkpatrick et al. (2017), which introduces a penalty term to slow down changes to weights important for old tasks. *“EWC allows knowledge of previous tasks to be protected during new learning, thereby avoiding catastrophic forgetting of old abilities”* ([Overcoming catastrophic forgetting in neural networks - ar5iv - arXiv](https://ar5iv.labs.arxiv.org/html/1612.00796#:~:text=arXiv%20ar5iv,It%20does%20so%20by)). This idea of selectively preserving important parameters connects to Sutton’s notion of valuing previously learned predictions – effectively treating certain learned weights as valuable predictions that shouldn’t be overwritten without penalty.

- **Replay and Rehearsal Methods**: Inspired by the replay buffers in RL (which themselves echo Sutton’s Dyna idea of learning from stored experiences), replay-based continual learning stores samples (or generative models of past data) to intermix old and new experiences. For instance, experience replay and **generative replay** (Shin et al., 2017) train the model on both new data and pseudo-data from previous tasks to refresh its memory. These methods operationalize the idea that reusing past experience (as in off-policy RL) can mitigate forgetting.

- **Dynamic Architectures and Expansion**: Some approaches dynamically grow or adjust the model’s architecture to accommodate new tasks, rather than forcing a single static network to handle everything. **Progressive Neural Networks** (Rusu et al., 2016) grow new columns for new tasks and leverage lateral connections to old knowledge, while other methods add neurons or modules on demand. The concept of **transferable features** and **soft gating** in these models resonates with hierarchical RL (options) – retaining modules (skills) learned before and choosing when to use or adapt them. Although Sutton’s work did not explicitly add neurons over time, his options framework and skill reuse ideas provide conceptual support for building systems that accumulate modules of knowledge.

- **Meta-Learning and Few-Shot Adaptation**: Another trend is applying meta-learning so that models can *learn how to learn* continually. Techniques like continual meta-learning adjust a model’s initialization or learning rules such that it can adapt quickly to new tasks without forgetting old ones. These approaches often draw on optimization-based meta-learning, which can be traced back to ideas in RL about tuning learning processes (for example, Sutton’s work on meta-gradient RL for adjusting parameters). The integration of meta-learning with continual learning reflects a convergence of ideas: using past experience to improve future learning efficiency – a principle that is central in reinforcement learning as well.

In addition to these methods, **recent work by Sutton’s own group has directly tackled continual learning challenges in deep networks**. Notably, Hernandez-Garcia, Sutton, and colleagues (2023) identified the “loss of plasticity” phenomenon: deep networks can become resistant to learning new information after prolonged training. They demonstrated this effect on image recognition and RL tasks and underscored its importance. The abstract of their work states that a learning system *“must continue to learn indefinitely. Unfortunately, our most advanced deep-learning networks gradually lose their ability to learn”* ([Maintaining Plasticity in Deep Continual Learning - Rich Sutton](https://www.youtube.com/watch?v=p_zknyfV9fY#:~:text=Abstract%3A%20Any%20learning%20system%20worthy,learning)). By highlighting this issue, they have spurred research into methods to maintain plasticity, such as resetting certain optimizer states, using regularizers to reinvigorate learning, or architectural tweaks (e.g. LayerNorm) to prevent saturation. This is a cutting-edge area building explicitly on Sutton’s legacy – ensuring agents remain adaptable over time.

The state-of-the-art in continual learning is a vibrant mix of these strategies. No single method has completely solved continual learning, but the community has made strides by combining ideas (for example, using both replay and regularization, or meta-learning with dynamic architectures). Researchers like James Kirkpatrick, David Lopez-Paz, Sylvain Lescouz, Joelle Pineau, and many others (often in collaboration with deep learning pioneers like Geoffrey Hinton or Yoshua Bengio) are actively contributing to the field. Ongoing research trends include applying continual learning to large-scale models (e.g., keeping large language models up-to-date), exploring unsupervised continual learning, and improving benchmarks and evaluation protocols for more realistic scenarios. The influence of Sutton’s foundational work is evident throughout these advances – from the incremental learning algorithms at their core to the broader vision of agents that accumulate knowledge over a lifetime.

## 4. Theoretical and Practical Challenges
Despite significant progress, continual learning still faces major theoretical and practical challenges. A foremost issue is **catastrophic forgetting**, the tendency of neural networks to forget previously learned tasks when trained on new ones. This problem was recognized in the 1980s and *“remains a core challenge in continual learning (CL), where models struggle to retain previous knowledge”* ([Mitigating Catastrophic Forgetting in Online Continual Learning by...](https://openreview.net/forum?id=olbTrkWo1D&referrer=%5Bthe%20profile%20of%20Peilin%20Zhao%5D(%2Fprofile%3Fid%3D~Peilin_Zhao2)#:~:text=Mitigating%20Catastrophic%20Forgetting%20in%20Online,to%20retain%20previous%20knowledge)). In other words, even with methods like EWC or replay, completely eliminating forgetting is an open problem ([A Study on Catastrophic Forgetting in Deep LSTM Networks](https://www.researchgate.net/publication/335698970_A_Study_on_Catastrophic_Forgetting_in_Deep_LSTM_Networks#:~:text=Networks%20www,forgetting%20remains%20an%20open%20problem)). Each class of solution so far comes with trade-offs – for example, regularization methods can constrain learning of new tasks, while replay methods require storage or generative models. *“Despite these advances, the problem of catastrophic forgetting remains unresolved. Each proposed solution comes with trade-offs”* ([Catastrophic Forgetting // is FT isn't the answer/solution? - sbagency](https://sbagency.medium.com/catastrophic-forgetting-is-ft-isnt-the-answer-84d251edd726#:~:text=sbagency%20sbagency,offs)).

One underlying difficulty is the **stability–plasticity dilemma**: a learning system must remain stable enough to preserve old knowledge (stability) yet plastic enough to integrate new knowledge (plasticity). Balancing this trade-off is non-trivial ([[PDF] New Insights for the Stability-Plasticity Dilemma in Online Continual ...](https://iclr.cc/media/iclr-2023/Slides/11634.pdf#:~:text=%E2%80%A2%20Stability,%E2%80%A2%20The)). Too much stability and the model becomes rigid (unable to learn new tasks); too much plasticity and it quickly overwrites old knowledge. Sutton’s observation of deep networks losing plasticity ([Maintaining Plasticity in Deep Continual Learning - Rich Sutton](https://www.youtube.com/watch?v=p_zknyfV9fY#:~:text=Abstract%3A%20Any%20learning%20system%20worthy,learning)) is one side of this coin – methods are needed to restore plasticity without causing forgetting. From a theoretical standpoint, there is not yet a unifying framework that explains how to optimally navigate this stability-plasticity balance in continually learning systems.

Another challenge is the **lack of formal theoretical guarantees** in continual learning. Unlike classical machine learning, which has well-developed theories for convergence and generalization (e.g., PAC learning or online learning regret bounds), continual learning scenarios (especially with non-i.i.d. data streams and task switching) are less understood. Researchers are beginning to address this by precisely defining the continual learning problem and its objectives. For instance, recent work has attempted to *“carefully defin[e] the continual reinforcement learning problem”* and formalize agents that learn indefinitely ([A Definition of Continual Reinforcement Learning - arXiv](https://arxiv.org/html/2307.11046v2#:~:text=In%20contrast%2C%20continual%20reinforcement%20learning,the%20importance%20of%20continual)). Such definitions are a first step toward theoretical analysis, but much remains to be done to derive performance guarantees or convergence proofs for continual learning algorithms.

On the practical side, **scalability and real-world deployment** pose challenges. Many continual learning methods are evaluated on relatively small-scale benchmarks or simplified tasks. There is concern about whether these methods will scale to more complex, real-world situations (e.g. robotics, continual learning in autonomous driving, or lifelong learning in interactive agents). A recent study noted a *“misalignment between the actual challenges of continual learning and the evaluation protocols in use”* ([Is Continual Learning Ready for Real-world Challenges? - arXiv](https://arxiv.org/abs/2402.10130#:~:text=This%20paper%20contends%20that%20this,the%20evaluation%20protocols%20in%20use)) – meaning that current benchmarks might not capture real-world complexity (such as continuous task blending, ambiguous task boundaries, or need for open-world learning where new classes emerge). Bridging this gap is essential for practical impact.

Additional practical challenges include:
- **Memory and Compute Constraints**: Some algorithms require storing data from all past tasks or training separate models for each task, which is impractical as tasks accumulate. Continual learners in the wild might be embedded in edge devices with limited resources, so efficiency is key.
- **Task Recognition and Transfer**: In many settings, the boundaries between tasks are not clearly given. The agent must detect distribution shifts or new tasks on its own (the **task-agnostic continual learning** scenario). The agent should also leverage commonalities between tasks (positive transfer) without interference. Achieving strong forward transfer (learning new tasks faster because of prior knowledge) while avoiding negative backward transfer (forgetting or degrading old task performance) is an open research frontier.
- **Theoretical Understanding of Neural Mechanisms**: Catastrophic forgetting is closely linked to how connectionist models distribute knowledge. A deeper theoretical understanding of why neural networks forget (e.g., weight interference, representational overlap) would inform better solutions. Similarly, understanding the “loss of plasticity” in optimization terms (such as plateaus in the loss landscape or saturation of activations) is an ongoing theoretical quest.

Looking forward, researchers identify several **future directions** to address these challenges. Developing a *unified theory of continual learning* is one such direction – possibly extending frameworks like Markov Decision Processes (MDPs) or online learning theory to encompass multiple tasks and non-stationary data. There is also interest in biologically inspired solutions: for example, taking inspiration from how humans and animals consolidate memories during sleep or through complementary learning systems (hippocampus and cortex). Such mechanisms could inform algorithms like experience rehearsal, generative replay, or dynamic reorganization of networks to protect important memories.

In summary, continual learning must overcome enduring challenges of forgetting and stability-plasticity, scale up to realistic problems, and gain stronger theoretical underpinnings. These challenges define an exciting research agenda: each limitation of current approaches points to an opportunity for innovation, where insights from reinforcement learning, neuroscience, and other fields can converge to advance our understanding and capabilities of lifelong learning systems.

## 5. Pathways for Contribution
For a researcher new to the field, there are rich opportunities to contribute to continual learning, especially on the theoretical side. Given the nascent state of a unifying theory, one promising pathway is to work on **formal frameworks and definitions** for continual learning. Clear definitions (such as recent attempts to formally define continual RL ([A Definition of Continual Reinforcement Learning - arXiv](https://arxiv.org/html/2307.11046v2#:~:text=In%20contrast%2C%20continual%20reinforcement%20learning,the%20importance%20of%20continual))) help in deriving analysis and comparing algorithms fairly. A newcomer could contribute by refining these definitions or proposing new metrics to evaluate continual learning (e.g., better measures of forgetting and knowledge transfer, or establishing theoretical bounds on performance degradation). Aligning evaluation protocols with real-world requirements is another impact area – for instance, defining benchmarks or challenge environments that capture the complexities of continual learning (as suggested by the misalignment noted in evaluations ([Is Continual Learning Ready for Real-world Challenges? - arXiv](https://arxiv.org/abs/2402.10130#:~:text=This%20paper%20contends%20that%20this,the%20evaluation%20protocols%20in%20use))).

On the theoretical front, one could delve into **analysis of learning dynamics** in neural networks under continual learning. This might involve studying the mathematical properties of loss landscapes when tasks change, or analyzing simplified models to understand catastrophic forgetting. For example, researching why certain regularization methods succeed or fail could lead to more principled algorithms. There is also room for developing **new algorithms with provable guarantees** – perhaps borrowing techniques from online convex optimization, game theory, or control theory to ensure stability. Bridging reinforcement learning theory (which deals with non-i.i.d. data and long-term credit assignment) with continual learning is fertile ground; ideas like regret minimization in non-stationary bandits or meta-learning guarantees could inspire continual learning theory.

Interdisciplinary intersections are especially promising. A new researcher might explore **neuroscience-inspired mechanisms** in a mathematically rigorous way. For instance, mechanisms of memory consolidation, neurogenesis (growing new neurons), or synaptic gating in the brain could translate into novel neural network architectures that dynamically grow or compartmentalize knowledge. Investigating such biologically motivated approaches could address the stability-plasticity dilemma in new ways (e.g., by creating separate fast and slow learning components, analogous to hippocampus and cortex). Collaboration with cognitive scientists or neuroscientists can provide insights into how natural systems achieve lifelong learning, which in turn can spark theoretical models for artificial systems.

Another pathway is to connect continual learning with **other areas of AI** that are currently booming. For example, continual learning for **large-scale models and lifelong knowledge** is a timely topic – how can we update large language models or vision models with new information continuously, without retraining from scratch or forgetting? This intersects with transfer learning and domain adaptation. A researcher could contribute by devising methods that allow pretrained models to absorb new data over time (important for keeping AI systems up-to-date in dynamic environments). There is also an intersection with **meta-learning and automated curriculum learning**: one can study how an agent might automatically generate or select experiences to maximally retain old knowledge while learning new things (essentially, self-curation of its training data stream).

From an applications standpoint, identifying real-world problems that benefit from continual learning and demonstrating solutions there can be highly impactful. Robotics is a clear example – an autonomous robot should learn from each experience throughout its life. A newcomer might work on a specific application (say, a household robot that learns new chores incrementally, or a recommendation system that adapts to user preference shifts) and contribute algorithms tailored to that context. Such applied work often reveals new theoretical challenges too, closing the loop between practice and theory.

In terms of community and resources, the continual learning field is very open and collaborative. Engaging with workshops and conferences dedicated to lifelong learning is a great way to contribute and get feedback. Notably, the **Conference on Lifelong Learning Agents (CoLLAs)** was launched in 2022 to bring together researchers focusing on continual learning agents ([Conference on Lifelong Learning Agents (CoLLAs)](https://lifelong-ml.cc/#:~:text=The%20Conference%20on%20Lifelong%20Learning,ideas%20on%20advancing%20machine%20learning)). Top machine learning venues (NeurIPS, ICML, ICLR) regularly feature continual learning papers, and journals like *IEEE TPAMI* and *JMLR* have published surveys and special issues on the topic ([A Comprehensive Survey of Continual Learning: Theory, Method ...](https://ieeexplore.ieee.org/document/10444954/#:~:text=A%20Comprehensive%20Survey%20of%20Continual,representative%20methods%2C%20and%20practical)). For a new researcher, contributing could mean publishing innovative findings at these venues, or even simply collaborating on open-source projects (the **ContinualAI** community, for instance, curates repositories and benchmarks for continual learning).

To summarize, a newcomer can contribute to continual learning by:
- **Developing Theory**: Work on formal definitions, stability-plasticity analysis, and deriving guarantees for algorithms.
- **Innovating Algorithms**: Propose new methods (regularization techniques, memory systems, meta-learning strategies) that address current limitations.
- **Cross-Pollination**: Bring ideas from other domains (neuroscience, RL, meta-learning, even evolutionary algorithms or federated learning) to continual learning.
- **Applications and Benchmarks**: Demonstrate continual learning in new applications or create more realistic benchmarks, guiding the field toward practical relevance.
- **Community Engagement**: Participate in continual learning workshops, share findings, and build upon the work of Sutton and others by keeping the conversation between theory and practice active.

Continual learning remains a frontier with many open questions. Rich Sutton’s contributions provide a strong foundation and inspiration – emphasizing that truly intelligent systems must learn *continually*. By building on this foundation and exploring the open problems, new researchers have the opportunity to make significant theoretical and practical advances in the quest for lifelong learning AI systems. 


Below is a concise summary of the key ideas that emerge from the talks and the two papers:

1. **Standard Deep Learning and Continual Learning:**
   - **One‐Time vs. Continual Learning:** Traditional deep‐learning methods (using backpropagation with gradient descent or variants such as Adam) are designed for “one‐time” training on a fixed dataset. In many real‐world applications—such as robotics, streaming data, or online reinforcement learning—the data distribution changes over time, requiring the network to continually learn.
   - **Loss of Plasticity:** Over time, as standard training continues in a nonstationary (continual) learning setting, deep networks lose their “plasticity” (i.e. the ability to quickly adapt to new data). This loss is manifested in several ways:
     - The weights tend to grow larger.
     - A growing fraction of neurons become “dead” (or saturated), meaning that they rarely change their output.
     - The internal representations (the “feature diversity”) become less rich, as measured by a decrease in the effective rank of the hidden layers.
   - This degradation means that—even if early performance on new tasks is good—the network eventually learns no better than a shallow (or even a linear) system when faced with many successive tasks.

2. **Empirical Demonstrations:**
   - Extensive experiments were conducted on supervised tasks (e.g., variations of ImageNet, class-incremental CIFAR‑100, Online Permuted MNIST, and a “Slowly Changing Regression” problem) and reinforcement learning tasks (such as controlling an “Ant” robot with changing friction).
   - In all these settings, standard backpropagation methods initially learn well but then gradually “forget how to learn” (i.e. they lose plasticity) over hundreds or thousands of tasks.

3. **Maintaining Plasticity by Injecting Randomness:**
   - The initial random weight initialization provides many advantages (diverse features, small weights, non-saturation) that enable rapid learning early on. However, because standard backprop only applies this “randomness” at the start, these beneficial properties fade with continued training.
   - The key idea is that **continual learning requires a sustained injection of randomness or variability** to maintain plasticity.

4. **Continual Backpropagation (CBP):**
   - To counteract the decay of plasticity, the authors propose an algorithm called **Continual Backpropagation**. CBP is almost identical to standard backpropagation except that, on every update, it selectively reinitializes a very small fraction of the network’s units.
   - **Selective Reinitialization:** Using a “utility measure” that assesses how useful a neuron (or feature) is for the current task (based on factors such as its activation, its outgoing weight magnitudes, and how much it is changing), the algorithm identifies neurons that are “underused” or “dead.” These neurons are then reinitialized (with the initial small random values), thereby reintroducing diversity and the benefits of a fresh start.
   - This process—sometimes called a “generate-and-test” mechanism—allows the network to continually inject new random features without having to completely reset or lose past learning.

5. **Comparison with Other Methods:**
   - Other techniques such as L2 regularization, Shrink and Perturb (which combines weight shrinkage with noise injection), dropout, and normalization were examined.
   - Although L2 regularization and Shrink and Perturb help slow the growth of weights and partially mitigate the loss of plasticity, they are generally less robust than CBP. In some experiments (both in supervised and reinforcement learning settings), popular methods like Adam (with standard parameters), dropout, and even batch normalization actually worsened the loss of plasticity over time.

6. **Implications for Continual and Reinforcement Learning:**
   - The findings imply that if deep neural networks are to be deployed in environments where continual adaptation is necessary, the training algorithms must be modified to continuously “refresh” the network’s ability to learn.
   - In reinforcement learning, where both the environment and the agent’s behavior can change over time, the loss of plasticity is especially problematic. The continual backpropagation approach (sometimes combined with a small amount of L2 regularization) was shown to significantly improve performance in nonstationary RL tasks (for example, in controlling an ant robot in environments with changing friction).

7. **Broader Perspective:**
   - The work challenges the assumption that gradient descent alone is sufficient for deep learning in dynamic, nonstationary settings.
   - It suggests that “sustained deep learning” (learning that continues to adapt over time) may require algorithms that combine traditional gradient-based methods with mechanisms for continual variability—in effect, a built-in “refresh” mechanism similar to how biological systems continually reorganize and adapt their neural circuitry.

In summary, the key idea is that standard deep learning methods gradually lose their ability to adapt (loss of plasticity) when faced with a continual stream of new tasks. The proposed solution is to modify backpropagation so that it continuously injects new random features (through selective reinitialization of low-utility units), thereby maintaining the network’s plasticity and enabling it to learn indefinitely in nonstationary environments.


Below is a mathematical‐level explanation of the key ideas behind loss of plasticity in continual learning and the “Continual Backpropagation” (CBP) solution.

---

### 1. Standard Backpropagation and Its Limitations

A deep neural network is parameterized by weights
$$
\mathbf{w} = \{w_{l,i,k}\}
$$
where
- $l$ indexes layers,
- $i$ indexes neurons (or “features”) in layer $l$,
- $k$ indexes neurons in layer $l+1$.

**Initialization:**  
Weights are initially drawn from a “small‐random” distribution, e.g.,
$$
w_{l,i,k}(0) \sim d \quad \text{with} \quad d = \mathcal{U}(-b,b),
$$
where $b$ is chosen (e.g., via Kaiming initialization) so that the activations do not saturate.

**Gradient Descent Update:**  
For each training example (or mini‐batch), the standard update is
$$
w_{l,i,k}(t+1) = w_{l,i,k}(t) - \alpha\, \nabla_{w_{l,i,k}} L(t),
$$
where
- $\alpha$ is the learning rate,
- $L(t)$ is the loss at time $t$.

**Loss of Plasticity:**  
When training continually on a nonstationary stream of data (or a long sequence of tasks), several phenomena occur:
- **Weight Growth:** The weights tend to grow larger over time.
- **Feature Saturation / “Dead” Units:** For activations like ReLU, if a neuron’s output $h_{l,i}(x)$ is zero (or nearly so) for almost all inputs, then
  $$
  \mathbb{P}\bigl[h_{l,i}(x)=0\bigr] \approx 1,
  $$
  the neuron is “dead” and its gradient becomes zero.
- **Representation Collapse (Low Effective Rank):**  
  For a given hidden layer, let $\Phi$ be the matrix of activations across examples. The *effective rank* of $\Phi$ is defined as
  $$
  \operatorname{erank}(\Phi) = \exp\left(-\sum_{k=1}^{q} p_k \log p_k\right),\quad p_k = \frac{\sigma_k}{\sum_{j=1}^{q}\sigma_j},
  $$
  where $\sigma_1,\dots,\sigma_q$ are the singular values of $\Phi$ (with $q = \max\{n, m\}$). A decrease in $\operatorname{erank}(\Phi)$ indicates that the network’s internal representation has lost diversity.

In continual learning, it is observed that after many tasks the network’s performance (say, measured by the error $E(t)$) deteriorates—often approaching or even falling below the performance of a shallow or linear model. In symbols, one finds for standard backpropagation that
$$
\lim_{T\to\infty} E_{\text{BP}}(T) \gtrsim E_{\text{linear}},
$$
indicating a loss of the “plasticity” needed to learn new tasks.

---

### 2. A Utility Measure for Neurons

The intuition is that the “good” properties of the network—diverse, non‐saturated features with small weights—arise from the initial random distribution. To maintain these advantages over time, one can track the “utility” of each neuron and selectively refresh those that are under‐used.

**Contribution Utility:**  
For neuron $i$ in layer $l$ at time $t$, define an instantaneous measure of its contribution as:
$$
c_{l,i}(t) = \; |h_{l,i}(t)| \; \sum_{k=1}^{n_{l+1}} |w_{l,i,k}(t)|,
$$
where
- $h_{l,i}(t)$ is the neuron's output,
- $\sum_{k}|w_{l,i,k}(t)|$ measures the total “influence” of neuron $i$ on the next layer.

To smooth this over time, one can maintain a running average:
$$
c_{l,i,t} = (1-\eta)\, c_{l,i}(t) + \eta\, c_{l,i,t-1},
$$
with decay rate $\eta \in (0,1)$.

**Adaptation Utility:**  
Because the speed at which a neuron can change is also important, one may consider an “adaptation utility” inversely related to the magnitude of its incoming weights:
$$
a_{l,i}(t) = \frac{1}{\sum_{j=1}^{n_{l-1}} |w_{l-1,j,i}(t)|},
$$
or a running average thereof.

**Overall Utility:**  
A combined measure might then be given by (after bias‐correction)
$$
y_{l,i}(t) = \frac{|h_{l,i}(t) - \hat{f}_{l,i}(t)|\;\sum_{k=1}^{n_{l+1}} |w_{l,i,k}(t)|}{\sum_{j=1}^{n_{l-1}} |w_{l-1,j,i}(t)|},
$$
and its running average
$$
u_{l,i,t} = (1-\eta)\, y_{l,i}(t) + \eta\, u_{l,i,t-1}.
$$
Finally, a bias-corrected utility can be computed as
$$
\hat{u}_{l,i,t} = \frac{u_{l,i,t}}{1-\eta^{a_{l,i}}},
$$
where $a_{l,i}$ may also serve as the “age” of the neuron (i.e. the number of updates since its last reinitialization).

Low values of $\hat{u}_{l,i,t}$ indicate that the neuron is “underperforming” or has become “stale.”

---

### 3. Continual Backpropagation (CBP) Algorithm

The CBP algorithm augments standard gradient descent by periodically “refreshing” low-utility neurons. Mathematically, the CBP procedure for each layer $l$ is as follows:

1. **Standard Update:**  
   For each weight, perform the gradient update:
   $$
   w_{l,i,k}(t+1) = w_{l,i,k}(t) - \alpha\, \nabla_{w_{l,i,k}} L(t).
   $$

2. **Age Update:**  
   For each neuron $i$ in layer $l$, update its age:
   $$
   a_{l,i} \leftarrow a_{l,i} + 1.
   $$

3. **Utility Update:**  
   Update the running utility $u_{l,i,t}$ as described above.

4. **Selective Reinitialization:**  
   For each layer $l$, define a replacement fraction $\rho$ (a small number, e.g. such that on average one neuron is reinitialized every few hundred updates). Then for neurons $i$ that satisfy:
   - $a_{l,i} \ge m$ (i.e. they are “mature” enough), and
   - $\hat{u}_{l,i,t}$ is among the lowest $\rho n_l$ values,
   
   perform the following:
   - **Reset Incoming Weights:**  
     $$
     w_{l-1}[:, i] \sim d,
     $$
     i.e. re-sample the incoming weights from the original distribution.
   - **Reset Outgoing Weights:**  
     $$
     w_{l}[i, :] = 0,
     $$
     so that the new neuron does not perturb the current function.
   - **Reset Utility and Age:**  
     $$
     u_{l,i,t} \leftarrow 0,\quad a_{l,i} \leftarrow 0.
     $$

This additional “generate-and-test” step keeps a small fraction of neurons “fresh” so that the benefits of the initial randomness (small weights, diverse activations) persist indefinitely.

---

### 4. Mathematical Effects and Comparisons

**Under Standard Backpropagation:**  
- The weight magnitudes $W(t) = \frac{1}{N}\sum_{l,i,k} |w_{l,i,k}(t)|$ tend to increase with time.
- The effective rank $\operatorname{erank}(\Phi(t))$ of hidden layer activations decreases.
- The fraction of dead neurons (those for which $h_{l,i}(x)=0$ for almost all $x$) increases.

As a consequence, if $E_{\text{BP}}(T)$ denotes the error after $T$ tasks, then
$$
\lim_{T\to\infty} E_{\text{BP}}(T) \approx E_{\text{linear}},
$$
meaning the network’s performance degrades to that of a shallow model.

**Under Continual Backpropagation (CBP):**  
The periodic reinitialization maintains:
- **Bounded Weight Magnitudes:** $W(t)$ remains low.
- **High Effective Rank:** $\operatorname{erank}(\Phi(t))$ stays high, indicating diverse representations.
- **Low Fraction of Dead Units:** Most neurons remain active.
  
Thus, the error $E_{\text{CBP}}(T)$ remains low (and often improves) over many tasks:
$$
\lim_{T\to\infty} E_{\text{CBP}}(T) \ll E_{\text{BP}}(T).
$$

In reinforcement learning, where both the environment and the agent’s actions continually change the data distribution, similar mathematical effects are observed. For example, the agent’s cumulative reward $R(t)$ under standard methods may plateau or even decrease, whereas with CBP (often combined with a modest amount of L2 regularization), the reward remains high.

---

### 5. Summary

Mathematically, the key innovations are:
- Recognizing that the properties of the initial random weight distribution—small magnitude, diversity, non-saturation—are crucial for rapid adaptation.
- Defining a utility measure $u_{l,i,t}$ for each neuron that combines its contribution (via activations and outgoing weights) and its capacity to adapt (via the inverse of the incoming weights).
- Implementing a selective reinitialization rule that, when $a_{l,i} \ge m$ and $u_{l,i,t}$ is low (specifically, among the lowest $\rho$ fraction), resets the neuron's weights to reintroduce the beneficial properties of the initial state.
- This procedure mathematically maintains a low overall weight magnitude, high effective rank, and low incidence of dead neurons, thereby preserving the network’s plasticity and ensuring continued learning in nonstationary or continual learning settings.

This approach—combining standard gradient descent with a continual, selective “refresh” of low‐utility neurons—provides a mathematically grounded mechanism to overcome the loss of plasticity that plagues standard deep learning when faced with a long sequence of tasks.


================================================================================

--- Processing: ../.ipynb_checkpoints/rank_deficiency-checkpoint.md ---
Thanks for the clarification! I will look into theoretical and semi-theoretical results on conditions that lead to stable rank deficiencies in neural networks, particularly in the middle or penultimate layers. The focus will be on:

- Rank collapse or stable low-rank representations under training.
- Loss functions like cross-entropy and MSE.
- The role of nonlinear activations (especially ReLU, sigmoid).
- The effect of standard stochastic gradient descent (SGD).
- Scenarios relevant to overparameterized MLPs.

I will get back to you with findings on theoretical guarantees or conditions that enforce such stable rank deficiencies.

# Stable Rank Deficiencies in Hidden Layers of Neural Networks

**Overview:** Deep neural networks often learn **low-rank representations** in their hidden layers, especially in later layers. Theory and empirical evidence suggest that as training progresses (and with sufficient overparameterization), the **effective rank** of hidden-layer activations or weight matrices can remain low or even *collapse*, rather than utilizing the full potential rank. Below, we outline key factors and findings that explain **when and why hidden layers maintain a stable, low rank** during training, focusing on theoretical conditions, loss functions, activation types, SGD dynamics, and overparameterization. 

## Monotonic Rank Reduction in Deep Networks (Theoretical Conditions)

**Composition Limits Rank:** A fundamental theoretical insight is that the **rank of a composed function (multiple layers)** cannot increase with depth – it tends to *monotonically decrease* or stay the same ([](https://proceedings.neurips.cc/paper_files/paper/2022/file/d5cd70b708f726737e2ebace18c3f71b-Paper-Conference.pdf#:~:text=rank%2C%20focusing%20particularly%20on%20the,MLPs%2C%20and%20Transformers%20on%20ImageNet)). In other words, as inputs propagate through layers, the dimension of the learned feature manifold is non-increasing. Any layer’s output is constrained by the information in previous layers, so once a representation loses rank (becomes low-dimensional), subsequent layers cannot recover the lost dimensions. Feng *et al.* (2022) formally prove a **“universal monotonic decreasing property”** of network rank based on differential and algebraic composition rules ([](https://proceedings.neurips.cc/paper_files/paper/2022/file/d5cd70b708f726737e2ebace18c3f71b-Paper-Conference.pdf#:~:text=rank%2C%20focusing%20particularly%20on%20the,MLPs%2C%20and%20Transformers%20on%20ImageNet)). This means **if a hidden layer becomes rank-deficient, that deficiency tends to persist or worsen in deeper layers**. For example, common operations like pooling, downsampling, or even standard fully-connected layers can significantly **drop the rank** of their outputs ([](https://proceedings.neurips.cc/paper_files/paper/2022/file/d5cd70b708f726737e2ebace18c3f71b-Paper-Conference.pdf#:~:text=decreasing%20comes%20from%20the%20structural,layer%20can%20loose%20ranks%20considerably)). This structural constraint helps explain why certain middle or penultimate layers might **lock in a low rank** — once features collapse onto a low-dimensional subspace, later layers can only use those dimensions (barring injection of new information).

**Intrinsic Data Constraints:** Theoretical conditions for low rank also relate to the data and function being learned. If the target function or data manifold is inherently low-dimensional, a network might **find a low-rank representation that suffices**. In fact, the rank of a layer’s output (viewed as a function) measures the “volume of independent information” it carries ([](https://proceedings.neurips.cc/paper_files/paper/2022/file/d5cd70b708f726737e2ebace18c3f71b-Paper-Conference.pdf#:~:text=In%20mathematics%2C%20the%20rank%20of,learning%20that%20underlies%20many%20tasks)). Under mild assumptions, an optimal network will not inflate rank beyond what’s needed to represent the underlying data structure. This connects to ideas like the **Information Bottleneck**, where networks compress intermediate representations. Patel & Shwartz-Ziv (2024) define a “local rank” measure of feature-manifold dimensionality and show that hidden-layer rank tends to **decrease in later training, forming an emergent bottleneck** ([Learning to Compress: Local Rank and Information Compression in Deep Neural Networks](https://arxiv.org/html/2410.07687v1#:~:text=Deep%20neural%20networks%20tend%20to,This%20work%20bridges%20the%20gap)) ([Learning to Compress: Local Rank and Information Compression in Deep Neural Networks](https://arxiv.org/html/2410.07687v1#:~:text=particularly%20multilayer%20perceptrons%20,2023a)). In summary, **deep networks naturally compress information**, and theory indicates that **once a layer’s rank becomes low (e.g. due to a bottleneck or saturating behavior), it remains stably low or further collapses as training continues** ([](https://proceedings.neurips.cc/paper_files/paper/2022/file/d5cd70b708f726737e2ebace18c3f71b-Paper-Conference.pdf#:~:text=rank%2C%20focusing%20particularly%20on%20the,MLPs%2C%20and%20Transformers%20on%20ImageNet)) ([](https://proceedings.neurips.cc/paper_files/paper/2022/file/d5cd70b708f726737e2ebace18c3f71b-Paper-Conference.pdf#:~:text=decreasing%20comes%20from%20the%20structural,layer%20can%20loose%20ranks%20considerably)).

## Implicit Low-Rank Bias in Overparameterized Models

**Gradient Descent Favors Low Rank:** When neural networks are *overparameterized* (more parameters than data or than strictly needed), there are infinitely many solutions that fit the training data. Theory shows that standard training algorithms have an **implicit bias toward “simpler” (lower-complexity) solutions**, often reflected in low-rank structure. In particular, for deep *linear* networks (a simplified case with no nonlinear activations), it’s proven that gradient descent (or flow) on the squared loss converges to solutions with minimal **effective rank**. Gunasekar *et al.* (2017) demonstrated that a depth-2 linear network trained on a matrix factorization task converges to the minimum nuclear-norm solution ([](https://minyoungg.github.io/overparam/resources/overparam-v2.pdf#:~:text=Notable%20work%20from%20Arora%20et,%282020%3B%202021%29%20argues)) – essentially the **lowest-rank weight matrix** that fits the data. More generally, later works showed that gradient descent tends to act as a *greedy rank minimizer* in linear matrix factorization problems ([](https://minyoungg.github.io/overparam/resources/overparam-v2.pdf#:~:text=Notable%20work%20from%20Arora%20et,%282020%3B%202021%29%20argues)). In deep linear networks, this means if the training data can be fit by a low-rank mapping, gradient-based optimization will converge to that low-rank solution, leaving the weight matrices **rank-deficient** (many singular values driven to zero). This implicit bias arises *without* any explicit rank regularization – it’s a property of the dynamics of overparameterized models.

**Depth Amplifies Rank Bias:** Depth itself contributes to the bias. Adding more hidden layers (even linear ones) can strengthen the tendency toward low-rank solutions. Arora *et al.* (2019) found that in deep linear networks, **singular values of the effective mapping decay faster with increased depth**, indicating that deeper architectures induce a stronger preference for concentrating information in a few directions ([](https://minyoungg.github.io/overparam/resources/overparam-v2.pdf#:~:text=solutions,rank%20solutions.%20Pennington)). In practice, **deeper networks often end up using only a subset of their neurons or directions effectively**, yielding low-rank feature matrices in intermediate and penultimate layers. Empirically, researchers observed that simply increasing depth (without changing the training objective) biases the network toward learning embeddings with lower rank ([](https://minyoungg.github.io/overparam/resources/overparam-v2.pdf#:~:text=work%20on%20over%02parameterization%20and%20highlights,matrix%20has%20a%20low%201)). In fact, random initialized deep models already tend to map data into a low-rank feature space (as measured by the Gram matrix of features) and this bias remains after training ([](https://minyoungg.github.io/overparam/resources/overparam-v2.pdf#:~:text=work%20on%20over%02parameterization%20and%20highlights,matrix%20has%20a%20low%201)). This “**low-rank simplicity bias**” means that among the many possible solutions in an overparameterized setting, deep networks prefer ones where features live in a smaller subspace ([](https://minyoungg.github.io/overparam/resources/overparam-v2.pdf#:~:text=hypothesis%20that%20deeper%20networks%20are,wide%20variety%20of%20commonly%20used)) ([](https://minyoungg.github.io/overparam/resources/overparam-v2.pdf#:~:text=gelu%20relu%20leaky%20relu%20tanh,matrix%20on%20a%20variety%20of)). Notably, this bias appears *robust*: it occurs across different initializations, hyperparameters, and even different optimizers ([](https://minyoungg.github.io/overparam/resources/overparam-v2.pdf#:~:text=simplicity%20bias%20exists%20even%20after,Practically%2C%20we)). In short, **overparameterization (especially increased depth) combined with gradient-based training often leads to persistent rank deficiencies** in hidden layers – the network finds a solution that uses far fewer independent dimensions than the layer width, and stays there.

**Permanent Rank Collapse:** Once a network converges to such a low-rank solution, those rank deficiencies tend to be “permanent” in the sense that continued training doesn’t reintroduce dropped dimensions. Instead, extended training can further reinforce the collapsed structure. For example, in overparameterized classifiers, one often observes **Neural Collapse** at the penultimate layer: features for each class collapse to their class mean, and those means become maximally spaced in a $C$-dimensional simplex (for $C$ classes). This implies the penultimate layer’s output has rank at most $C$ (much lower than its potential dimension). Importantly, this collapsed configuration emerges *in the late stage of training* and then remains stable ([Are All Losses Created Equal: A Neural Collapse Perspective | OpenReview](https://openreview.net/forum?id=8rfYWE3nyXl#:~:text=a%20recent%20line%20work%20showing,produce%20equivalent)). Any extra degrees of freedom (e.g. additional neurons) simply align redundantly with this low-dimensional structure rather than expanding it. Thus, in highly overparameterized Multi-Layer Perceptrons (MLPs), it’s common to see entire directions in weight space or neuron activations effectively unused – **the network has more capacity than it needs, and gradient descent naturally finds a solution that leaves a “rank gap”**. The theoretical and empirical works above provide conditions for this: **if a low-rank solution exists (e.g. data lies on a low-dimensional manifold, or fewer than $N$ independent features are needed to classify $N$ classes), an overparameterized network will often converge to that solution, making the hidden-layer rank deficit permanent** ([](https://minyoungg.github.io/overparam/resources/overparam-v2.pdf#:~:text=Notable%20work%20from%20Arora%20et,%282020%3B%202021%29%20argues)) ([Are All Losses Created Equal: A Neural Collapse Perspective | OpenReview](https://openreview.net/forum?id=8rfYWE3nyXl#:~:text=a%20recent%20line%20work%20showing,produce%20equivalent)). In summary, **overparameterization doesn’t increase the utilized rank – paradoxically, it often ensures some neurons or directions are redundant, locking in a low-rank representation** for middle and penultimate layers.

## Loss Functions (Cross-Entropy vs. MSE) and Rank Stability

**Cross-Entropy and Neural Collapse:** The choice of loss function can influence training dynamics and feature geometry, but common losses in classification (cross-entropy) and regression (mean squared error) ultimately can lead to similar low-rank feature outcomes. Cross-entropy (with softmax output) is known to drive networks into the **“neural collapse”** regime during the terminal phase of training, especially in classification tasks. Under cross-entropy loss, as training error approaches zero, the network continues to sharpen the separations between classes: penultimate layer features for each class become nearly identical (collapsed to their mean), and different class means maximize their mutual distances in feature space. This is a highly symmetric, low-rank configuration (features span roughly a $C-1$ dimensional subspace for $C$ classes). Papyan *et al.* (2020) observed this phenomenon empirically, and it has since been analyzed theoretically. In particular, for sufficiently large networks trained to convergence on cross-entropy, **the only global minimizers are those exhibiting neural collapse** ([Are All Losses Created Equal: A Neural Collapse Perspective | OpenReview](https://openreview.net/forum?id=8rfYWE3nyXl#:~:text=a%20recent%20line%20work%20showing,produce%20equivalent)). That means the **optimal solution inherently has low-rank feature structure** (the rank of the class-feature matrix equals the number of classes, not the number of features). Any features beyond that subspace are essentially unused.

**MSE Loss and Other Losses:** One might suspect that mean squared error (MSE) loss, which doesn’t push outputs to extremes the way cross-entropy does, might behave differently. However, recent theoretical work shows that **MSE loss can also lead to neural collapse at optimality** for overparameterized models. Zhou *et al.* (2022) compare cross-entropy vs. MSE and find that **both losses (and even variants like label smoothing or focal loss) yield the same neural collapse structure in the learned features**, given a large enough network trained to minimal loss ([Are All Losses Created Equal: A Neural Collapse Perspective | OpenReview](https://openreview.net/forum?id=8rfYWE3nyXl#:~:text=a%20recent%20line%20work%20showing,produce%20equivalent)) ([Are All Losses Created Equal: A Neural Collapse Perspective | OpenReview](https://openreview.net/forum?id=8rfYWE3nyXl#:~:text=global%20solution%20and%20landscape%20analyses,FL%20loss%20near%20the%20optimal)). In other words, the global optimum features under MSE classification loss still have all samples of a class coincident at the class mean, and class means maximally separated (simplex configuration). This result implies that **the low-rank collapse of penultimate-layer features is not specific to cross-entropy** – it is a property of the *classification problem* itself at the optimum, rather than the particular loss formula ([Are All Losses Created Equal: A Neural Collapse Perspective | OpenReview](https://openreview.net/forum?id=8rfYWE3nyXl#:~:text=a%20recent%20line%20work%20showing,produce%20equivalent)). Empirically, networks trained with either cross-entropy or MSE (on the same classification task) tend to end up with very similar penultimate-layer geometry and test performance, as long as they are sufficiently overparameterized and trained long enough ([Are All Losses Created Equal: A Neural Collapse Perspective | OpenReview](https://openreview.net/forum?id=8rfYWE3nyXl#:~:text=model%20assumption%2C%20we%20provide%20either,large%20and%20trained%20until%20convergence)).

**Dynamics Differences:** While the *final outcomes* under cross-entropy and MSE can be similar (both can collapse features to a low-rank configuration), the **training dynamics** may differ. Cross-entropy is an **exponential-type loss** that continues to penalize even tiny classification errors, which often drives weights to grow in norm and features to become extremely pure (one-hot like probabilities). This can encourage faster or more pronounced collapse of features during the later stages of training. In contrast, MSE (for classification) treats the problem more like regression to one-hot targets; once the network fits the training points, there’s no additional push to exaggerate the features. As a result, some studies noted that **neural collapse emerges more clearly or earlier with cross-entropy** (which implicitly maximizes class separation margin), whereas with MSE, collapse may still occur but perhaps requires more epochs or stronger overparameterization to mirror the same effect ([Are All Losses Created Equal: A Neural Collapse Perspective | OpenReview](https://openreview.net/forum?id=8rfYWE3nyXl#:~:text=a%20recent%20line%20work%20showing,produce%20equivalent)). Nonetheless, in either case the penultimate layer ends up **low-rank (approximately rank = number of classes)** at convergence. For regression tasks, MSE can also lead to low-rank internal representations if the target function is low-dimensional. Overall, common loss functions like cross-entropy and MSE *do not prevent* rank deficiency in hidden layers; at optimum they often **demand it**, by driving the network toward a simplified, structured solution (neural collapse being a prime example) ([Are All Losses Created Equal: A Neural Collapse Perspective | OpenReview](https://openreview.net/forum?id=8rfYWE3nyXl#:~:text=a%20recent%20line%20work%20showing,produce%20equivalent)).

## Activation Functions (ReLU vs. Sigmoid) and Rank Collapse

**ReLU Networks:** The choice of activation affects how information flows and whether it’s preserved or squashed, which in turn influences rank. **ReLU (Rectified Linear Unit)** activations are piecewise-linear and can introduce **dead neurons or linear dependencies** that reduce rank. For example, if a ReLU neuron’s input is negative for all training samples, that neuron outputs all zeros – effectively removing one dimension from that layer’s output (a rank drop). Even when active, ReLUs output either a linear scaling of the input or zero, so large groups of neurons can end up encoding redundant directions (especially if their weight vectors are correlated). From a theoretical perspective, understanding rank in nonlinear networks is harder than in linear ones. Recent work by Timor *et al.* (2023) shows that in contrast to linear networks, **gradient flow on ReLU networks doesn’t always minimize rank** – in fact, they construct scenarios where a shallow ReLU network does *not* find the lowest-rank solution ([](https://proceedings.mlr.press/v201/timor23a/timor23a.pdf#:~:text=what%20extent%20this%20gen%02eralizes%20to,solutions%20in%20several%20reasonable%20settings)). This is a “negative result” indicating that ReLU’s piecewise linearity can sometimes preserve or create just enough variation to avoid trivial rank collapse in small cases. **However, on the positive side, deeper ReLU networks **are** biased toward low-rank solutions in many settings ([](https://proceedings.mlr.press/v201/timor23a/timor23a.pdf#:~:text=datasets%20of%20size%202%29,solutions%20in%20several%20reasonable%20settings)).** In other words, depth appears to restore the implicit rank minimization tendency even with ReLUs. Sufficiently deep ReLU MLPs have been proven to favor low-rank function mappings under certain assumptions ([](https://proceedings.mlr.press/v201/timor23a/timor23a.pdf#:~:text=datasets%20of%20size%202%29,solutions%20in%20several%20reasonable%20settings)), aligning with the empirical findings discussed earlier (depth-driven low-rank bias). Moreover, experiments show that the phenomenon of rank reduction with depth holds **across a variety of activation functions** – ReLU included. Rosenfeld *et al.* (2021) observed that increasing the number of layers **decreases the effective rank of the feature matrix for ReLU networks**, just as it does for smooth activations ([](https://minyoungg.github.io/overparam/resources/overparam-v2.pdf#:~:text=gelu%20relu%20leaky%20relu%20tanh,matrix%20on%20a%20variety%20of)). In summary, ReLU’s on-off behavior can cause **rank collapse by deactivating neurons or duplicating features**, and while a shallow ReLU net might not automatically minimize rank, a deep ReLU net trained with SGD still tends to learn a compressed (low-rank) representation in later layers.

**Sigmoid and Saturating Activations:** **Sigmoid or tanh activations** (smooth, squashing non-linearities) have their own influence on rank. These functions saturate at extreme values (output approaching 0 or 1 for sigmoid, -1 or 1 for tanh), which can effectively **flatten variations** in input. During training, it is often observed that later in training (or in early layers of very deep networks), many sigmoid/tanh neurons enter saturation for a wide range of inputs. A neuron stuck near 0 or 1 for all inputs contributes almost no meaningful variability – it’s nearly a constant output, reducing the rank of that layer’s activation matrix (similar to a dead or saturated unit). This behavior ties in with the **Information Bottleneck (IB) theory**: Tishby and colleagues (2017) reported that networks with saturating activations show phases of training where **mutual information between the layer and the input drops**, implying the layer is discarding information and compressing its representation. This compression often corresponds to many neurons saturating, hence fewer effective degrees of freedom (a lower-dimensional manifold of activations). In practice, sigmoid networks were found to undergo an initial fitting phase followed by a **compression phase**, where the hidden-layer information (and empirically, the variance or entropy of activations) collapses significantly. This suggests that **sigmoid/tanh networks may exhibit an even stronger rank-collapse tendency in later layers** compared to ReLU, since saturation can make large portions of the layer’s output almost constant. Indeed, Patel & Shwartz-Ziv (2024) note that networks compress their feature manifolds in later training, and this was originally observed in saturating networks consistent with IB predictions ([Learning to Compress: Local Rank and Information Compression in Deep Neural Networks](https://arxiv.org/html/2410.07687v1#:~:text=Deep%20neural%20networks%20tend%20to,This%20work%20bridges%20the%20gap)).

**Activation-Type Comparisons:** Empirical studies directly comparing activation functions find that **the trend of low-rank representations is quite general**. For instance, one study computed the “effective rank” of the feature Gram matrix for networks with ReLU, leaky ReLU, tanh, GELU, and even sinusoidal activations, across various depths. The result was universal: **adding more layers consistently lowered the effective rank of the penultimate-layer features for all activation types tested** ([](https://minyoungg.github.io/overparam/resources/overparam-v2.pdf#:~:text=gelu%20relu%20leaky%20relu%20tanh,matrix%20on%20a%20variety%20of)). While the absolute level of rank may differ (e.g. some activations might retain slightly more information in shallow layers), the *qualitative behavior* – a bias toward low-rank, structured representations – appears across the board ([](https://minyoungg.github.io/overparam/resources/overparam-v2.pdf#:~:text=gelu%20relu%20leaky%20relu%20tanh,matrix%20on%20a%20variety%20of)) ([](https://minyoungg.github.io/overparam/resources/overparam-v2.pdf#:~:text=depth%20of%20the%20model,By%20hierarchically)). Therefore, common nonlinearities like ReLU and sigmoid both permit rank-deficient solutions. ReLU may allow networks to retain more piecewise-linear information in some cases (not *always* minimizing rank), but in deep and overparameterized regimes it still converges to low-rank feature mappings. Sigmoidal activations naturally encourage compression via saturation, often leading to **“rank collapse” as training goes on** (neurons saturate and outputs cluster). In both cases, once neurons either saturate or become redundant, those dimensions effectively drop out of the model’s representation. No matter the activation, a deep network that has more capacity than needed will tend to **only use a few dominant directions** in each layer’s output – yielding a stable, low-rank representation by the penultimate layer.

## Role of SGD and Training Dynamics in Rank Deficiency

**SGD as an Implicit Regularizer:** Interestingly, the stochastic nature of training (stochastic gradient descent, SGD) itself plays a role in enforcing low-rank structures. Recent theoretical analyses point out that **mini-batch SGD injects noise that biases the solution toward low-complexity (low-rank) weights**. Galanti & Poggio (2022) proved that when training deep networks with small-batch SGD (especially with common tweaks like weight decay), the noise in the gradients creates an *implicit constraint* that favors low-rank solutions ([](https://cbmm.mit.edu/sites/default/files/publications/Implicit%20Rank%20Minimization.pdf#:~:text=We%20analyze%20deep%20ReLU%20neural,Our%20analysis)). In fact, they show the very source of SGD noise can be viewed as a form of *rank regularization*: all else equal, SGD tends to drive the weight matrices to smaller effective rank over training ([](https://cbmm.mit.edu/sites/default/files/publications/Implicit%20Rank%20Minimization.pdf#:~:text=We%20analyze%20deep%20ReLU%20neural,Our%20analysis)). Their theory connects batch size, learning rate, and weight decay to the rank of the learned weight matrices ([](https://cbmm.mit.edu/sites/default/files/publications/Implicit%20Rank%20Minimization.pdf#:~:text=%E2%80%A2%20In%20Thm,weight%20decay%2C%20optimization%20and%20rank)). Specifically, **smaller batch sizes and the use of weight decay act as strong low-rank regularizers** ([](https://cbmm.mit.edu/sites/default/files/publications/Implicit%20Rank%20Minimization.pdf#:~:text=%E2%80%A2%20In%20Thm,weight%20decay%2C%20optimization%20and%20rank)). Intuitively, gradient noise continually perturbs the solution within the space of zero training error solutions, and it preferentially guides the weights toward configurations that are “simpler” (analogous to how noise in linear regression can favor solutions with smaller norm). In deep networks, that simplicity manifests as weight matrices with many singular values effectively zero – i.e. **SGD implicitly pushes toward rank deficiency** in each layer’s weights ([](https://cbmm.mit.edu/sites/default/files/publications/Implicit%20Rank%20Minimization.pdf#:~:text=We%20analyze%20deep%20ReLU%20neural,Our%20analysis)). This complements the implicit bias of gradient descent itself; even without noise, gradient descent finds a low-rank solution in many cases, and **with noise (SGD) this bias is even more pronounced**.

**Empirical Evidence in SGD Dynamics:** Empirically, one finds that SGD-trained networks often learn features in a stage-wise fashion, capturing the most significant structure first. Early in training, networks latch onto the largest variance or easiest features in the data (sometimes called *spectral bias* or *dominant feature first* learning). As Pezeshki *et al.* (2020) observed, **SGD tends to learn statistically dominant features first, which leads to learning low-rank solutions** for the data ([](https://minyoungg.github.io/overparam/resources/overparam-v2.pdf#:~:text=compression%2C%20Guo%20et%20al,an%20implicit%20regularizer%20during%20training)). In practical terms, this means the network might initially increase the rank of representations to fit the data variation, but once it has fit the major patterns, additional training **compresses the representation**, aligning with those dominant patterns and ignoring minor ones. This is consistent with a reduction in rank as training continues. Some works also note that different optimization algorithms (SGD vs. adaptive methods like Adam) yield similar low-rank phenomena in deep models ([](https://minyoungg.github.io/overparam/resources/overparam-v2.pdf#:~:text=simplicity%20bias%20exists%20even%20after,Practically%2C%20we)), suggesting the effect is not solely due to SGD’s noise but also due to the parameterization. Nevertheless, **SGD’s stochastic noise reinforces symmetry-breaking and flat minima selection** that often coincides with low-rank weight configurations. The *theorem* by Galanti & Poggio even implies that as long as there is some SGD noise (e.g. mini-batches) and weight decay, the training will *never exactly converge* but instead keep hovering around a solution, effectively preventing the network from utilizing extra degrees of freedom that aren’t needed ([](https://cbmm.mit.edu/sites/default/files/publications/Implicit%20Rank%20Minimization.pdf#:~:text=only%20assumed%20to%20be%20differentiable,rank)) ([](https://cbmm.mit.edu/sites/default/files/publications/Implicit%20Rank%20Minimization.pdf#:~:text=%E2%80%A2%20In%20Sec,4)). In other words, the noise keeps the model near a minimal-rank solution. 

**Breaking vs. Enforcing Rank Deficiency:** One might wonder if SGD noise could ever *break* a rank deficiency (for example, jostle the network out of a bad symmetric solution where two neurons are identical, thereby increasing rank). In practice, SGD **does break exact symmetric degeneracies** (it’s rare for two neurons to remain perfectly identical during SGD training because tiny gradient differences will separate them). However, the net effect of SGD’s randomness is not to maximize rank, but rather to explore solutions of similar performance and favor the ones with smoother or simpler structure. If a certain low-rank configuration suffices to fit the data, SGD is unlikely to kick the network into a higher-rank regime without a clear benefit. In fact, small-batch SGD will add noise that, on average, drives the weights toward the flat region of the loss landscape that often corresponds to compressive solutions ([](https://cbmm.mit.edu/sites/default/files/publications/Implicit%20Rank%20Minimization.pdf#:~:text=%E2%80%A2%20In%20Thm,weight%20decay%2C%20optimization%20and%20rank)). Thus, **SGD more often enforces rank deficiency than alleviates it**. The overall training dynamic typically sees the rank of hidden-layer activations **initially high (when learning diverse features), then stabilizing or decreasing as training converges**, especially under SGD. This aligns with the “compression phase” idea and has been measured directly: for example, Patel (2024) found that the *local rank* of features in each layer dropped during the final phase of training, indicating SGD was fine-tuning the model by compressing representations further ([Learning to Compress: Local Rank and Information Compression in Deep Neural Networks](https://arxiv.org/html/2410.07687v1#:~:text=Deep%20neural%20networks%20tend%20to,This%20work%20bridges%20the%20gap)) ([Learning to Compress: Local Rank and Information Compression in Deep Neural Networks](https://arxiv.org/html/2410.07687v1#:~:text=dimensionality%20and%20demonstrate%2C%20both%20theoretically,information%20bottlenecks%20and%20representation%20learning)). In summary, **SGD (with typical settings) implicitly regularizes the network toward low-rank solutions**, making rank-deficient hidden layers a persistent outcome of the training process ([](https://cbmm.mit.edu/sites/default/files/publications/Implicit%20Rank%20Minimization.pdf#:~:text=We%20analyze%20deep%20ReLU%20neural,Our%20analysis)) ([](https://minyoungg.github.io/overparam/resources/overparam-v2.pdf#:~:text=compression%2C%20Guo%20et%20al,an%20implicit%20regularizer%20during%20training)).

## Overparameterized MLPs and Conditions for Rank Collapse

**Excess Capacity Leads to Redundancy:** In multi-layer perceptrons with more neurons or layers than necessary, certain conditions practically guarantee rank deficiencies. If an MLP has layers much wider than the intrinsic dimension of the problem, it has the freedom to realize a solution where many neurons are simply not needed. **Gradient descent will often utilize only a subset of neurons (or a subset of independent directions in those neurons)** to solve the task, leaving the rest effectively redundant. For instance, if an MLP could solve a task with an internal representation of dimension $d$, giving it $d \times 2$ neurons in that layer doesn’t force it to use all $2d$ directions – it may well use only $d$ of them (making the layer’s output rank $d$). The other neurons might end up as linear combinations of the first $d$ or stuck at zero weights. **This is a common scenario for “permanent” rank deficiency**: the network finds a low-rank configuration early (or by midpoint of training) and never needs to activate the extra capacity. Once those extra neurons settle into redundancy (e.g., duplicating another neuron’s behavior or outputting near-constant), the rank of that layer stays low. Any slight perturbation (like SGD noise) doesn’t overcome the bias to keep them redundant, because deviating would not improve the loss.

**Bottleneck Layers and Architecture:** Some networks explicitly include bottleneck layers (fewer neurons) in the middle by design; those obviously enforce low rank at that point. But even in **uniformly wide networks, an “effective bottleneck” can emerge**. Rangamani *et al.* (2023) empirically found that as one goes deeper into a trained classifier, the **within-class variability of features shrinks and class means become the dominant components** ([Feature Learning in Deep Classifiers through Intermediate Neural Collapse](https://proceedings.mlr.press/v202/rangamani23a/rangamani23a.pdf#:~:text=these%20properties%20extend%20to%20intermediate,class%20means%20aligns%20with%20the)). In effect, by the penultimate layer, most of the variation in features is between classes rather than within – which means the representation has roughly one degree of freedom per class (a very low rank structure) ([Feature Learning in Deep Classifiers through Intermediate Neural Collapse](https://proceedings.mlr.press/v202/rangamani23a/rangamani23a.pdf#:~:text=these%20properties%20extend%20to%20intermediate,class%20means%20aligns%20with%20the)). This happens even if every hidden layer had the same width; the *network learned* to create an information bottleneck near the end. Such **self-induced bottlenecks** are a hallmark of overparameterized models: they have enough layers/neurons to first separate the classes and then compress each class cluster tightly. The penultimate layer in these cases is severely rank-deficient (often close to rank $C$ for $C$ classes). Notably, this low-rank state is maintained – it doesn’t revert or expand – even if training continues longer (the clusters just tighten further). Conditions that encourage this include having **much more model capacity than the minimum required**, and optimizing to near-zero training loss (so the network can afford to project data onto a structured low-dimensional subspace that cleanly separates classes).

**Explicit Regularization and Rank:** It’s worth mentioning that some explicit regularization techniques can also encourage low-rank solutions (e.g. weight decay, which is commonly used, biases toward smaller weights that often imply fewer independent components). However, what we’ve discussed are *implicit* phenomena – even in the absence of explicit rank penalties, overparameterized MLPs tend to end up with stable rank deficiencies. In fact, adding too strong an explicit rank penalty is often unnecessary or even harmful; simply relying on the network’s inductive biases and SGD tends to find a good low-rank solution on its own ([](https://minyoungg.github.io/overparam/resources/overparam-v2.pdf#:~:text=maps%20to%20low,parameters%2C%20and)) ([](https://minyoungg.github.io/overparam/resources/overparam-v2.pdf#:~:text=shows%20that%20linear%20models%20with,demonstrate%20the%20practical%20applications%20of)). Researchers have demonstrated that **deliberately overparameterizing a model (especially in depth) can improve generalization by leveraging this implicit low-rank bias**, rather than explicitly constraining rank ([](https://minyoungg.github.io/overparam/resources/overparam-v2.pdf#:~:text=optimizers,parameterizing%20standard%20models%20at%20training)). For example, inserting extra linear layers (which add depth but no new nonlinearity) during training was shown to yield lower-rank features and better generalization, even though the model’s theoretical capacity didn’t change ([](https://minyoungg.github.io/overparam/resources/overparam-v2.pdf#:~:text=optimizers,parameterizing%20standard%20models%20at%20training)). This underscores that **overparameterization itself (when coupled with SGD) is a form of regularization** that often realizes as low-rank internal representations.

**Summary of Conditions:** In an overparameterized MLP, you are likely to see **permanent rank collapse** in hidden layers when: (1) the model has significantly more parameters than needed to fit the data, (2) the training is run to near convergence (zero or negligible training error), and (3) standard losses (like cross-entropy or MSE) and optimizers (SGD or similar) are used. Under these conditions, theoretical and empirical studies indicate the network will converge to a solution where hidden layers (especially the penultimate layer) have **stable, low rank** – often determined by the problem’s inherent dimensionality (such as number of classes or principal components of the data). The **rank remains low throughout the end of training** because neither the architecture nor the training dynamics provide an incentive to reinflate it. On the contrary, deep architectures and SGD training both *favor* collapsing dimensions and finding efficient, low-dimensional encodings ([](https://proceedings.neurips.cc/paper_files/paper/2022/file/d5cd70b708f726737e2ebace18c3f71b-Paper-Conference.pdf#:~:text=rank%2C%20focusing%20particularly%20on%20the,MLPs%2C%20and%20Transformers%20on%20ImageNet)) ([](https://cbmm.mit.edu/sites/default/files/publications/Implicit%20Rank%20Minimization.pdf#:~:text=We%20analyze%20deep%20ReLU%20neural,Our%20analysis)). This is why we observe phenomena like neural collapse and rank-deficient features in practice. In conclusion, hidden-layer rank deficiency in neural networks is backed by several semi-theoretical and theoretical insights: **deep compositions naturally restrict rank, gradient descent implicitly seeks low-rank solutions (especially in overparameterized setups), common loss functions do not oppose (and in fact often drive) rank collapse, nonlinear activations (ReLU or sigmoid) still end up compressing information, and SGD’s stochastic nature further encourages simplicity**. All these factors together explain **when/why a network’s middle or penultimate layers might maintain a low, stable rank despite ongoing training** ([](https://proceedings.neurips.cc/paper_files/paper/2022/file/d5cd70b708f726737e2ebace18c3f71b-Paper-Conference.pdf#:~:text=rank%2C%20focusing%20particularly%20on%20the,MLPs%2C%20and%20Transformers%20on%20ImageNet)) ([Are All Losses Created Equal: A Neural Collapse Perspective | OpenReview](https://openreview.net/forum?id=8rfYWE3nyXl#:~:text=a%20recent%20line%20work%20showing,produce%20equivalent)) ([](https://cbmm.mit.edu/sites/default/files/publications/Implicit%20Rank%20Minimization.pdf#:~:text=We%20analyze%20deep%20ReLU%20neural,Our%20analysis)).

**References:**

- R. Feng *et al.*, *“Rank Diminishing in Deep Neural Networks,”* NeurIPS 2022 – establishes that network rank **monotonically decreases** with depth and analyzes rank deficiencies per layer ([](https://proceedings.neurips.cc/paper_files/paper/2022/file/d5cd70b708f726737e2ebace18c3f71b-Paper-Conference.pdf#:~:text=rank%2C%20focusing%20particularly%20on%20the,MLPs%2C%20and%20Transformers%20on%20ImageNet)) ([](https://proceedings.neurips.cc/paper_files/paper/2022/file/d5cd70b708f726737e2ebace18c3f71b-Paper-Conference.pdf#:~:text=decreasing%20comes%20from%20the%20structural,layer%20can%20loose%20ranks%20considerably)).  
- S. Gunasekar *et al.*, *“Implicit Regularization in Matrix Factorization,”* NeurIPS 2017 – shows gradient descent on overparametrized linear models converges to **minimum-nuclear-norm (low-rank)** solutions ([](https://minyoungg.github.io/overparam/resources/overparam-v2.pdf#:~:text=Notable%20work%20from%20Arora%20et,%282020%3B%202021%29%20argues)).  
- J. Zhou *et al.*, *“Are All Losses Created Equal? A Neural Collapse Perspective,”* NeurIPS 2022 – proves that both cross-entropy and MSE losses (and others) yield **Neural Collapse** at global optima, implying low-rank penultimate features for sufficiently large networks ([Are All Losses Created Equal: A Neural Collapse Perspective | OpenReview](https://openreview.net/forum?id=8rfYWE3nyXl#:~:text=a%20recent%20line%20work%20showing,produce%20equivalent)).  
- N. Timor *et al.*, *“Implicit Regularization Towards Rank Minimization in ReLU Networks,”* ALT 2023 – finds that while shallow ReLU nets may not always minimize rank, **deeper ReLU nets are biased towards low-rank solutions** under gradient flow ([](https://proceedings.mlr.press/v201/timor23a/timor23a.pdf#:~:text=what%20extent%20this%20gen%02eralizes%20to,solutions%20in%20several%20reasonable%20settings)).  
- T. Galanti & T. Poggio, *“SGD Noise and Implicit Low-Rank Bias in Deep Neural Networks,”* CBMM Memo 2022 – theoretically shows mini-batch **SGD + weight decay imposes a low-rank constraint** on weight matrices; smaller batches and higher weight decay strengthen this effect ([](https://cbmm.mit.edu/sites/default/files/publications/Implicit%20Rank%20Minimization.pdf#:~:text=We%20analyze%20deep%20ReLU%20neural,Our%20analysis)) ([](https://cbmm.mit.edu/sites/default/files/publications/Implicit%20Rank%20Minimization.pdf#:~:text=%E2%80%A2%20In%20Thm,weight%20decay%2C%20optimization%20and%20rank)).  
- A. Rangamani *et al.*, *“Feature Learning in Deep Classifiers through Intermediate Neural Collapse,”* ICML 2023 – empirically demonstrates that **intermediate layers progressively collapse** class-wise: deeper layers have much lower within-class variance (effectively lower rank) relative to between-class variance ([Feature Learning in Deep Classifiers through Intermediate Neural Collapse](https://proceedings.mlr.press/v202/rangamani23a/rangamani23a.pdf#:~:text=these%20properties%20extend%20to%20intermediate,class%20means%20aligns%20with%20the)).  
- M. Rosenfeld *et al.* (OpenReview preprint 2021), *“The Low-Rank Simplicity Bias in Deep Networks,”* – provides empirical evidence that **increasing depth consistently reduces the effective rank** of learned features across various architectures and activations, and that this bias is robust to different optimizers ([](https://minyoungg.github.io/overparam/resources/overparam-v2.pdf#:~:text=gelu%20relu%20leaky%20relu%20tanh,matrix%20on%20a%20variety%20of)) ([](https://minyoungg.github.io/overparam/resources/overparam-v2.pdf#:~:text=simplicity%20bias%20exists%20even%20after,Practically%2C%20we)).  
- Additional references: S. Arora *et al.* 2019; B. Pezeshki *et al.* 2020; H. Papyan *et al.* 2020; and others as cited above, which further support these points on rank collapse and implicit biases in deep learning ([](https://minyoungg.github.io/overparam/resources/overparam-v2.pdf#:~:text=solutions,rank%20solutions.%20Pennington)) ([](https://minyoungg.github.io/overparam/resources/overparam-v2.pdf#:~:text=compression%2C%20Guo%20et%20al,an%20implicit%20regularizer%20during%20training)) ([Are All Losses Created Equal: A Neural Collapse Perspective | OpenReview](https://openreview.net/forum?id=8rfYWE3nyXl#:~:text=a%20recent%20line%20work%20showing,produce%20equivalent)).

================================================================================

--- Processing: ../.ipynb_checkpoints/main-checkpoint.md ---
# Recap and background
Here we recap the findings we have and review some background on  continual learning from the point of view of Ricahrd Sutton. 


*Broad objectives*: Through a series of works on Richard Sutton, he proposes that one of the most fundamental problems facing AI today is that we cannot let a model continually learn different tasks. This problem has two facets: 1) the catastrophic forgetting, which is forgetting what network has learned in the past, and 2) the issue of loss of plasticity, which is the network losing its ability to learn new concepts. We focus on the later issue of CL her, as it clearly relates to the theoretical result proven later. While there are numerous valuable observations and remedies in these works, there are numerous gaps in our understanding of CL, which makes us unable to build models and algorithms that are able to achieve CL in the same fashion as humans or animals in the real world achieve. Our attempt is to give the key ideas and notions in continual learning a theoretical footing, thereby consolidating the key root causes of loss plasticity. 

We first start by a review of loss of plasticity from Richard Sutton’s point of view, a review of hyper cloning. Then, we elaborate on our theories regarding the noise-less and noisy cases. Finally, we draw on parallels between the key results of our theory and loss of plasticity, where we highlight where it aligns and where it might offer a different perspective on fundamental causes of loss of plasticity.  Based on these theoretical results, we make several hypotheses on how these root causes of loss of plasticity emerge during prolonged training. We would like to understand the architectural and training dynamics that contribute to loss of plasticity in order to both avoid the loss of plasticity during training, and be able to recover from it if it is not catastrophic. Finally, we will also rely on empirical evidence to evaluate the hypotheses that were inspired by our theory. 

## Richard Sutton's view on Continual Learning (CL):
 Here we very briefly review the key notions and ideas that are proposed by Richard Sutton to diagnose, quantify and alleviate barriers to continual learning. Through a series of works, he observes that after having being trained one some tasks, the model’s ability to learn new tasks drops irrecoverably the level of a shallow or even linear model.  Here’s a brief recap of his key contributions to understanding and remedies for loss of plasticity. 

1. Standard Deep Learning and Continual Learning:
    * One‐Time vs. Continual Learning: Traditional deep‐learning methods (using backpropagation with gradient descent or variants such as Adam) are designed for “one‐time” training on a fixed dataset. In many real‐world applications—such as robotics, streaming data, or online reinforcement learning—the data distribution changes over time, requiring the network to continually learn.
    * Loss of Plasticity: Over time, as standard training continues in a non-stationary (continual) learning setting, deep networks lose their “plasticity” (i.e. the ability to quickly adapt to new data). This loss is manifested in several ways:
        * The weights tend to grow larger.
        * A growing fraction of neurons become “dead” (or saturated), meaning that they rarely change their output.
        * The internal representations (the “feature diversity”) become less rich, as measured by a decrease in the effective rank of the hidden layers.
    * This degradation means that—even if early performance on new tasks is good—the network eventually learns no better than a shallow (or even a linear) system when faced with many successive tasks.
2. Empirical Demonstrations:
    * Extensive experiments were conducted on supervised tasks (e.g., variations of ImageNet, class-incremental CIFAR‑100, Online Permuted MNIST, and a “Slowly Changing Regression” problem) and reinforcement learning tasks (such as controlling an “Ant” robot with changing friction).
    * In all these settings, standard backpropagation methods initially learn well but then gradually “forget how to learn” (i.e. they lose plasticity) over hundreds or thousands of tasks.
3. Maintaining Plasticity by Injecting Randomness:
    * The initial random weight initialization provides many advantages (diverse features, small weights, non-saturation) that enable rapid learning early on. However, because standard backprop only applies this “randomness” at the start, these beneficial properties fade with continued training.
    * The key idea is that continual learning requires a sustained injection of randomness or variability to maintain plasticity.
4. Continual Backpropagation (CBP):
    * To counteract the decay of plasticity, the authors propose an algorithm called Continual Backpropagation. CBP is almost identical to standard backpropagation except that, on every update, it selectively reinitializes a very small fraction of the network’s units.
    * Selective Reinitialization: Using a “utility measure” that assesses how useful a neuron (or feature) is for the current task (based on factors such as its activation, its outgoing weight magnitudes, and how much it is changing), the algorithm identifies neurons that are “underused” or “dead.” These neurons are then reinitialized (with the initial small random values), thereby reintroducing diversity and the benefits of a fresh start.
    * This process—sometimes called a “generate-and-test” mechanism—allows the network to continually inject new random features without having to completely reset or lose past learning.
5. Comparison with Other Methods:
    * Other techniques such as L2 regularization, Shrink and Perturb (which combines weight shrinkage with noise injection), dropout, and normalization were examined.
    * Although L2 regularization and Shrink and Perturb help slow the growth of weights and partially mitigate the loss of plasticity, they are generally less robust than CBP. In some experiments (both in supervised and reinforcement learning settings), popular methods like Adam (with standard parameters), dropout, and even batch normalization actually worsened the loss of plasticity over time.
6. Implications for Continual and Reinforcement Learning:
    * The findings imply that if deep neural networks are to be deployed in environments where continual adaptation is necessary, the training algorithms must be modified to continuously “refresh” the network’s ability to learn.
    * In reinforcement learning, where both the environment and the agent’s behavior can change over time, the loss of plasticity is especially problematic. The continual backpropagation approach (sometimes combined with a small amount of L2 regularization) was shown to significantly improve performance in nonstationary RL tasks (for example, in controlling an ant robot in environments with changing friction).
7. Broader Perspective:
    * The work challenges the assumption that gradient descent alone is sufficient for deep learning in dynamic, nonstationary settings.
    * It suggests that “sustained deep learning” (learning that continues to adapt over time) may require algorithms that combine traditional gradient-based methods with mechanisms for continual variability—in effect, a built-in “refresh” mechanism similar to how biological systems continually reorganize and adapt their neural circuitry.


## Hyper cloning recap and connection to loss of plasticity 

 In hyper cloning, authors showed various methods to “enlarge” a trained smaller model, such that its forward pass is perfectly preserved during cloning process, and showing that it can be trained much faster than a large model from scratch while achieving a similar accuracy. However, authors also observe that in the cloning strategies that were “noiseless” the training was not as effective, and had to introduce some types of noise to ensure the training diverged from a simple setting. 

*The theoretical result on cloning:* In this work, we first prove a series of result that are directly stated in terms of the 
We proved in the draft that if we duplicate a hyper-clone a model in a super-symmetric way (more accurately, forward and backward symmetry hold), then the forward and backward vectors of the network are cloned. More concretely, the forward and backward of the model are essentially the cloned (duplicated) versions of a smaller model from which they are cloned. This situation has a very dramatic consequence that, we can perfectly predict the training dynamics of the larger model with a smaller model, with the only caveat that the learning rate for different layers are set in a layer and module-dependent manner.  On the other hand, we show that the noisy cloning strategies can be modeled as the noiseless close plus some additive noise to the backprop gradients, where norm of the gradients per each layer may depend on their depth and network parameters in general. 

## Connection to loss of plasticity?

From a high level point of view, we can view the noiseless cloning strategies as analogues of normal backprop, in that this particular way of cloning may catastrophically limit the model's ability to learn, because it will always be as good as the smaller model, which highly resembles  notion of loss of plasticity phenomenon. 
More concretely, the fact that we can prove that the forward representation remain cloned throughout training, also implies indirectly that the rank of forward representations the larger model will always be equal to rank of the smaller model. This is arguably a very strong case of loss of rank (hard rank as opposed to empirical rank in Richard Sutton’s paper), which is provably unrecoverable! 


*Recovery of loss of plasticity*: Furthermore, our theory shows that  noisy cloning strategies as analogues of continual backprop in that they can be viewed as a normal backprop plus some injected noise.  Therefore, the ability to recover the large model capacity by having some type of noise is an analogue of resetting certain weights and units in continual backprop. 

*Key differences and remaining questions* Despite the similarity of our results on key aspects, there remains some important differences and open questions. 

Firstly, if we assume that in the smaller model all neurons are active and the weights are not saturated, the larger model will have similar properties. Therefore, we construct examples of a network that have catastrophic loss of plasticity, while having no dead neurons or overly large weights. In other words, our theory suggests that while dead and saturated neurons may be a sufficient condition for loss o plasticity, they are not necessary conditions. Conversely, the examples observed by Richard Sutton and other empirical evidence suggests that a network can experience loss of plasticity without having explicitly cloned neurons. In yet another words, while cloning or dead or saturated neurons may occur in some cases of loss of plasticity, but not all cases. In our quest to understand loss of plasticity, we will be searching for a set of “universal” conditions that are necessary and sufficient. Another key aspect of loss of plasticity is that without direct intervention, they will “persist” throughout training. Thus, we must be able to demonstrate  that once these conditions  are met, they will persist to be true throughout training, unless  something like a noise injection breaks these properties explicitly. We can refer to these properties as “canonical properties” of loss of plasticity. 

Secondly, even if we show that these conditions are universal and persistent, it is unclear how these conditions “emerge” during training? In other words, even if we have correctly identified these universal causes and can prove they are persistent, it’s still highly non-trivial why and how these conditions emerge as a result of prolonged training.  For example, while dead neurons or saturated neurons or cloned neurons , it is not guaranteed at all (theoretically or empirically) at all if such properties will emerge as a result of prolonged training. 
Since we know that certain choices of models and training will improve or deteriorate the emergence of these conditions, one can argue that that answer to this “emergence” question will be architecture and configuration-dependent  and not universal. Thus, once we can establish some clear links between training dynamics and model architecture, and loss of plasticity, we can also suggest ways to avoid loss of plasticity, or suggest methods to recover from it. 

Thus, these are the following key questions regarding loss of plasticity we are trying address in this work:
* *Canonical conditions*: What are they underlying universal conditions  that are necessary and sufficient, and they will persist throughout training?  
* *Emergence with training*: How do these conditions emerge as a result of prolonged training? “

### Canonical conditions
Let us try to capitalize on our earlier theories about cloning to arrive at a formalization of the question we are trying to address: 

*Definition*: In a neural network with parameters $\Theta$ , we define:
* Universal: we say $C(\Theta)$ a universal condition for loss of plasticity for network, if there some smaller network  with parameters $\theta$ ($|Theta| > |\theta|$), such that the larger model hidden units are exactly a copy of the smaller model up to some duplications. 
* Persistent:  having the we say that a $C(\Theta)$ is persistent
Let’s start by reviewing the key aspect of our theory on cloning and Richard Sutton’s views on loss of plasticity. Most notably, both views suggest that  a collapse of forward representations is a key indicator of loss of plasticity. However, as mentioned earlier, neither dead, saturated, nor neurons nor duplicated are necessary for loss of plasticity to occur. 






================================================================================

--- Processing: ../.ipynb_checkpoints/nn_interpret-checkpoint.md ---
Thanks for the detailed clarifications! I will conduct research on existing academic and open-source efforts related to analyzing neural network activations, weights, and interpretability, particularly with an emphasis on:
- Tracking activations and weight changes both during and after training.
- Extensible architectures (starting from MLP but scalable to CNNs, Transformers, etc.).
- Methods for displaying raw activations and weight matrices while considering cases where the data is too large.
- Interactive vs. static approaches for understanding neural network behavior.
- Potential for an "agentic" framework, where the tool actively interacts with the model, making decisions on what to analyze next based on intermediate results.

I'll return with findings on relevant existing projects, techniques, and potential inspiration for refining the conceptual design.

# Analyzing Neural Network Activations, Weights, and Interpretability

## Tools for Tracking Activations and Weights During Training
- **TensorBoard & Logging Libraries:** Basic tools like TensorBoard can log metrics, weights, and activations at each training step, providing simple visualizations (e.g. histograms of weight magnitudes) over time. However, these often focus on summary statistics and may miss instance-specific behaviors ([Comgra: A Tool for Analyzing and Debugging Neural Networks](https://arxiv.org/html/2407.21656v1#:~:text=What%20Parts%20of%20the%20Network,these%20correlate%20with%20other%20inputs)). More advanced logging frameworks (Weights & Biases, Visdom, etc.) offer custom dashboards for tracking arbitrary values during training ([Comgra: A Tool for Analyzing and Debugging Neural Networks](https://arxiv.org/html/2407.21656v1#:~:text=Visdom,for%20different%20types%20of%20data)), but still require deciding **a priori** what to record.  
- **Comgra (2024):** An open-source library designed to flexibly inspect network parameters and intermediate tensors throughout training ([Comgra: A Tool for Analyzing and Debugging Neural Networks](https://arxiv.org/html/2407.21656v1#:~:text=Comgra.%20Comgra%20,details%20for%20a%20specific%20input)). Comgra addresses the “combinatorial explosion” of possible things to monitor by letting users interactively choose which activations or weights to explore after training, avoiding the need to rerun experiments ([Comgra: A Tool for Analyzing and Debugging Neural Networks](https://arxiv.org/html/2407.21656v1#:~:text=The%20Need%20for%20Flexibility,At)). It records selected tensors during training and provides a GUI to navigate them, balancing aggregate statistics with per-example details in one interface ([Comgra: A Tool for Analyzing and Debugging Neural Networks](https://arxiv.org/html/2407.21656v1#:~:text=layer,a%20noticeable%20loss%20in%20performance)). This ensures one can examine both *overall trends* and *specific cases* without a huge logging overhead. Usage is analogous to TensorBoard: instrument the training loop to save data, then open a browser UI for interactive exploration ([Comgra: A Tool for Analyzing and Debugging Neural Networks](https://arxiv.org/html/2407.21656v1#:~:text=Usage,as%20well%20as%20code%20examples)).  
- **TorchLens (2023):** A Python package for extracting and visualizing hidden-layer activations in PyTorch models ([TorchLens: A Python package for extracting and visualizing hidden ...](https://www.biorxiv.org/content/10.1101/2023.03.16.532916v1#:~:text=,layer%20activations%20in%20PyTorch%20models)). It makes it easy to tap into any layer of an MLP, CNN, or Transformer to record activations and even the computational graph. TorchLens provides detailed visualizations of model architectures (like an improved computation graph over TensorBoard’s basic graph) ([Comgra: A Tool for Analyzing and Debugging Neural Networks](https://arxiv.org/html/2407.21656v1#:~:text=Visualizing%20the%20Computation%20Graph,a%20toolkit%20for%20visualizing%20models)), and lets researchers **probe activations** without writing boilerplate code. This facilitates comparisons of activation patterns across layers or between two training epochs.  
- **Tracking Weight Dynamics:** Beyond activations, some tools focus on how **weight matrices** evolve. For example, WeightWatcher is a research-inspired tool that analyzes trained weight matrices using heavy-tailed spectrum metrics ([WeightWatcher: Data-Free Diagnostics for Deep Learning](https://weightwatcher.ai/#:~:text=WeightWatcher%20%28w%7Cw%29%20is%20an%20open,JMLR%2C%20Nature%20Communications%2C%20and%20NeurIPS2023)) ([WeightWatcher: Data-Free Diagnostics for Deep Learning](https://weightwatcher.ai/#:~:text=%23%20Weightwatcher%20computes%20unique%20layer,quality%20metrics)). It can flag layers that are *under-trained* or *over-regularized* via an alpha metric (ideal range ~2–6 for well-trained layers) ([WeightWatcher: Data-Free Diagnostics for Deep Learning](https://weightwatcher.ai/#:~:text=weightwatcher%20,be%20between%202%20and%206)). Such analysis can be done at checkpoints during training to see if certain layers have converged or not. Academic studies of weight dynamics (e.g. how initialization or optimization affects weight trajectories) provide theoretical insight ([New insights into training dynamics of deep classifiers | MIT News | Massachusetts Institute of Technology](https://news.mit.edu/2023/training-dynamics-deep-classifiers-0308#:~:text=The%20paper%2C%20%E2%80%9CDynamics%20in%20Deep,of%20the%20layers%20are%20intertwined)) ([New insights into training dynamics of deep classifiers | MIT News | Massachusetts Institute of Technology](https://news.mit.edu/2023/training-dynamics-deep-classifiers-0308#:~:text=fit%20a%20training%20dataset%20will,to%20accurately%20classify%20new%20examples)), but practical tools for live weight tracking tend to reduce information to summaries (means, variances, spectral norms, etc.). A well-designed framework might log weight distribution histograms over time or compute metrics like WeightWatcher’s *alpha* at each epoch to observe training progress per layer.

## Interpretability Techniques from MLPs to CNNs to Transformers
- **Attribution Libraries:** For trained models, general interpretability frameworks like **Captum** (PyTorch’s interpretability library) provide a suite of attribution methods (saliency maps, integrated gradients, etc.) to explain predictions ([Comgra: A Tool for Analyzing and Debugging Neural Networks](https://arxiv.org/html/2407.21656v1#:~:text=Attribution%20and%20Interpretability,to%20interpret%20your%20model%E2%80%99s%20behavior)). Such tools apply to MLPs, CNNs, or Transformers in a model-agnostic way, treating the network as a black box to attribute importance to inputs or neurons. This helps answer “why did this input yield that output?” by tracking influence through the network.  
- **Mechanistic Interpretability Toolkits:** A growing set of libraries focus on opening the black box of **internal mechanisms**, often starting with simple models and scaling up. For example, **TransformerLens** (Nanda & Bloom 2022) allows loading pretrained transformers and inspecting or even modifying their internals (attention patterns, layer outputs) easily ([Comgra: A Tool for Analyzing and Debugging Neural Networks](https://arxiv.org/html/2407.21656v1#:~:text=Mechanistic%20Interpretability,toolkit%20for%20sequence%20generation%20models)). Similarly, **Pyrene** and **NNSight** provide interfaces to intervene on activations or weights during runs ([Comgra: A Tool for Analyzing and Debugging Neural Networks](https://arxiv.org/html/2407.21656v1#:~:text=Mechanistic%20Interpretability,toolkit%20for%20sequence%20generation%20models)). These emerged from research in **mechanistic interpretability**, where investigators often begin with small MLPs or toy models to understand learned algorithms, then extend methods to larger CNNs and Transformers. Techniques like **network surgery** (ablating or re-weighting neurons) and **counterfactual inputs** (designed to target specific neurons) are supported in such frameworks to test hypotheses about model behavior.  
- **Architecture-General vs. Specialized Approaches:** Simpler multi-layer perceptrons (MLPs) can be analyzed with generic approaches (e.g. recording activation values, visualizing weights as heatmaps) that carry over to deeper architectures. CNNs introduce structured weights (filters) that can be visualized as images, and activations that can be seen as feature maps; accordingly, tools like **ActiVis** and **Deep Visualization Toolbox** were created to explore CNN internals in real time by showing each layer’s feature maps for a given input ([Understanding Neural Networks Through Deep Visualization](https://anhnguyen.me/project/understanding-neural-networks-through-deep-visualization/#:~:text=Understanding%20Neural%20Networks%20Through%20Deep,files%20or%20read%20video)) ([Deep Visualization Toolbox Open-source software...](https://prostheticknowledge.tumblr.com/post/123726938701/deep-visualization-toolbox-open-source-software#:~:text=software,the%20reaction%20of%20every%20neuron)). Transformers have specialized components (self-attention matrices, multi-head attention, etc.), spurring custom visualization tools like **BertViz** (for attention patterns) and libraries like **Inseq** for sequence-to-sequence models ([Comgra: A Tool for Analyzing and Debugging Neural Networks](https://arxiv.org/html/2407.21656v1#:~:text=Mechanistic%20Interpretability,toolkit%20for%20sequence%20generation%20models)). The key trend is that methods are increasingly *extensible*: an insight gained from a simple MLP (say, tracking neuron activation distributions) can be scaled to thousands of neurons in a CNN, or millions in a Transformer, with the aid of automation and visualization techniques.

## Visualizing and Summarizing Large Activations and Weights
Interpreting a network often means dealing with **high-dimensional data** – e.g. millions of activations across a dataset, or weight matrices with thousands of parameters. Researchers have developed strategies to summarize and visualize this information:

- **Activation Atlases:** Google Brain’s *Activation Atlas* technique (2019) is a prime example of summarizing large activation spaces. By applying dimensionality reduction and feature visualization to millions of intermediate activations, they create an *“explorable activation atlas”* that maps out the prominent features a network has learned ([Activation Atlas](https://distill.pub/2019/activation-atlas#:~:text=By%20using%20feature%20inversion%20to,network%20typically%20represents%20some%20concepts)). Instead of examining one input at a time, an atlas provides a global view of concepts (for instance, clusters of neurons responding to “electronics” or “animal faces” appear as regions in the map) ([Activation Atlas](https://distill.pub/2019/activation-atlas#:~:text=Above%2C%20an%20activation%20atlas%20of,screen%20version)). This kind of visualization helps show *which concepts are represented and where*, giving a big-picture understanding of a CNN’s feature space. It directly addresses the limitation that inspecting single inputs “doesn’t give us a big picture view… when what we want is a map of an entire forest, inspecting one tree at a time will not suffice” ([Activation Atlas](https://distill.pub/2019/activation-atlas#:~:text=Unfortunately%2C%20visualizing%20activations%20has%20a,a%20time%20will%20not%20suffice)).
- **Network Dissection:** *Network Dissection* (Bau et al. 2017) offers a way to compress a complex CNN’s behavior into human-readable summaries by automatically labeling neurons with semantic concepts ([[1704.05796] Network Dissection: Quantifying Interpretability of Deep Visual Representations](https://arxiv.org/abs/1704.05796#:~:text=,our%20method%20to%20compare%20the)). It uses a broad set of visual concepts (like textures, objects, parts) and checks which neurons strongly respond to those concepts in a dataset. The result is a dictionary of neurons and their likely semantic roles (e.g. “neuron 123 = detects cats”). Importantly, this framework quantifies interpretability (what fraction of neurons have clear meanings) and can be applied at different training stages ([[1704.05796] Network Dissection: Quantifying Interpretability of Deep Visual Representations](https://arxiv.org/abs/1704.05796#:~:text=latent%20representations%20of%20various%20networks,interpretability%20of%20deep%20visual%20representations)). For instance, one can see neurons gradually specialize as training proceeds ([[1704.05796] Network Dissection: Quantifying Interpretability of Deep Visual Representations](https://arxiv.org/abs/1704.05796#:~:text=latent%20representations%20of%20various%20networks,interpretability%20of%20deep%20visual%20representations)), helping track *how representational structure emerges*. This is a form of summarization: instead of showing all weights, it highlights a few salient ones with descriptions.
- **Weight Visualizations & Metrics:** For weight matrices, straightforward visualizations include heatmaps of weight values or their distributions. For CNNs, visualizing the first-layer filters gives an intuition of learned edges or color detectors; for deeper layers, tools like **Netron** and **Penzai** focus on visualizing model structures (shapes of weight tensors, connectivity) to manage complexity ([Comgra: A Tool for Analyzing and Debugging Neural Networks](https://arxiv.org/html/2407.21656v1#:~:text=Visualizing%20the%20Computation%20Graph,a%20toolkit%20for%20visualizing%20models)). Summarization metrics can distill large weight sets into numbers: e.g., the **WeightWatcher** tool computes metrics like the *alpha* exponent of layer weight spectra, condensing each layer’s quality into a single number ([WeightWatcher: Data-Free Diagnostics for Deep Learning](https://weightwatcher.ai/#:~:text=weightwatcher%20,be%20between%202%20and%206)). These numbers can be tracked across layers or over training time to find outliers (e.g. a layer whose weights are degenerate or poorly trained). Such quantitative summaries are easier to visualize (as bar charts or trend lines) than the raw weight matrices themselves.
- **Dimensionality Reduction and Embeddings:** Another approach is to embed high-dimensional activations or neurons into lower dimensions for visualization. Techniques like t-SNE or UMAP can project activation vectors (for many data points) into 2D, revealing clustering of neurons or data by similarity. For example, plotting the activations of a certain layer for thousands of inputs might show distinct clusters corresponding to classes. Similarly, one can treat each neuron as a point in a high-dimensional space (defined by its responses across many inputs) and use embedding to find groups of neurons that behave similarly. These visualizations, while not as directly interpretable as Activation Atlases, help **spot structure** in otherwise unwieldy tensors – an important step before feeding results to an interpretability assistant like ChatGPT for summarization.

## Interactive vs. Static Approaches
Interpretability tools vary in how users engage with them:

- **Interactive Tools and Frameworks:** Interactive systems allow users to pose new queries, adjust inputs, and immediately see results, which is invaluable for exploratory analysis. **ActiVis** (Facebook, 2017) is an early example: an interactive visualization system for large-scale models that integrates multiple coordinated views, such as an architecture graph and a neuron activation heatmap, enabling pattern discovery at both instance-level and subset-level ([[1704.01942] ActiVis: Visual Exploration of Industry-Scale Deep Neural Network Models](https://arxiv.org/abs/1704.01942#:~:text=participatory%20design%20sessions%20with%20over,of%20how%20ActiVis%20may%20work)). Users could select subsets of data (say, all images of “cats”) and see which neurons fire strongly, or click on a neuron to see what inputs activate it – all in a fluid UI. Similarly, Google’s open-source **Language Interpretability Tool (LIT)** provides an interactive dashboard for NLP models ([The Language Interpretability Tool (LIT): Interactive Exploration and Analysis o](https://research.google/blog/the-language-interpretability-tool-lit-interactive-exploration-and-analysis-of-nlp-models/#:~:text=With%20these%20challenges%20in%20mind%2C,extensible%20visualizations%20and%20model%20analysis)). It supports *local explanations* (e.g. salience maps on a specific sentence) and *aggregate analysis* (e.g. embedding projections of an entire dataset) side by side ([The Language Interpretability Tool (LIT): Interactive Exploration and Analysis o](https://research.google/blog/the-language-interpretability-tool-lit-interactive-exploration-and-analysis-of-nlp-models/#:~:text=LIT%20supports%20local%20explanations%2C%20including,in%20our%20system%20demonstration%20paper)). Users can generate counterfactual inputs on the fly and see how the model’s predictions and internal activations change, facilitating a tight human-in-the-loop investigation cycle ([The Language Interpretability Tool (LIT): Interactive Exploration and Analysis o](https://research.google/blog/the-language-interpretability-tool-lit-interactive-exploration-and-analysis-of-nlp-models/#:~:text=LIT%20supports%20local%20explanations%2C%20including,in%20our%20system%20demonstration%20paper)). Interactive tools typically emphasize **flexibility and drill-down**: one can start with a broad overview and then zoom into particular cases, or vice versa, to test hypotheses about model behavior in real time.  
- **Static Analysis and Reports:** In contrast, many interpretability techniques yield static outputs – think of a research paper figure showing a set of maximally activating images for several neurons, or a plot of weight distributions at epoch end. Static approaches include saliency maps or Grad-CAM heatmaps produced for a fixed set of inputs, or feature visualizations of neurons (e.g. the synthesized images that strongly activate a neuron). These are often insightful but are inherently limited to the scenarios the researcher anticipated. They don’t easily allow asking new questions of the model without going back to code. For example, a static *feature visualization* shows what one neuron likes, but if you suddenly wonder how that neuron behaves for a specific real input, you’d need to run an experiment outside of the static report. Static results are great for communication (e.g. illustrating a learned feature in a publication) and for documenting known behaviors, but they lack the ability to **adapt** to the analyst’s curiosity in the moment. Modern tools aim to bridge this gap: even Distill.pub articles often embed interactive widgets so that what starts as a “static” article becomes a playground for the reader. This trend recognizes that interpretability is often an iterative process of discovery, benefitting from interactive exploration rather than one-shot analysis.

## Agentic and Automated Analysis Approaches
A recent and exciting development is the idea of an **“agentic” interpretability tool** – one that actively decides what to inspect next, rather than just passively visualizing predetermined data. Instead of a human manually probing the network step by step, an *AI agent* can leverage intermediate findings to guide further analysis. Two notable examples:

- **MIT’s Multimodal Automated Interpretability Agent (MAIA, 2024):** This system uses a pretrained language model equipped with a suite of tools to conduct interpretability research on neural nets ([A Multimodal Automated Interpretability Agent](https://arxiv.org/html/2404.14394v1#:~:text=This%20paper%20describes%20maia%2C%20a,of%20maia%20to%20computer%20vision)) ([A Multimodal Automated Interpretability Agent](https://arxiv.org/html/2404.14394v1#:~:text=trained%20vision,Across%20several%20trained%20models%20and)). In essence, MAIA behaves like a research scientist: it can synthesize new inputs (for example, generate images or edit text) to test what causes a particular neuron or sub-network to activate, find real dataset examples that maximally activate a component, and then *summarize its observations in natural language*. The agent iteratively forms hypotheses (“Neuron X seems to respond to *cars*”), designs experiments to verify them (e.g. feed images of cars, planes, boats to see if it fires only on cars), and refines its understanding based on results ([AI agents help explain other AI systems | MIT News | Massachusetts Institute of Technology](https://news.mit.edu/2024/ai-agents-help-explain-other-ai-systems-0103#:~:text=For%20example%2C%20FIND%20contains%20synthetic,The)) ([AI agents help explain other AI systems | MIT News | Massachusetts Institute of Technology](https://news.mit.edu/2024/ai-agents-help-explain-other-ai-systems-0103#:~:text=Sarah%20Schwettmann%20PhD%20%2721%2C%20co,%E2%80%9D)). Notably, MAIA was able to produce neuron descriptions for vision models that were comparable to human experts’ descriptions ([A Multimodal Automated Interpretability Agent](https://arxiv.org/html/2404.14394v1#:~:text=experimentation%20on%20subcomponents%20of%20other,truth)) ([A Multimodal Automated Interpretability Agent](https://arxiv.org/html/2404.14394v1#:~:text=Interpretability%20experiments%20proposed%20by%20maia,classified.%E2%80%A1)). It also tackled higher-level tasks like identifying biases or spurious features by actively searching for inputs that trigger those behaviors. This agentic approach demonstrates that language models (with the right tooling) can go beyond static summarization – they can *actively explore* a neural network, which is a promising direction for complex models that are too large for exhaustive manual probing ([AI agents help explain other AI systems | MIT News | Massachusetts Institute of Technology](https://news.mit.edu/2024/ai-agents-help-explain-other-ai-systems-0103#:~:text=Central%20to%20this%20strategy%20is,other%20systems%20in%20real%20time)) ([AI agents help explain other AI systems | MIT News | Massachusetts Institute of Technology](https://news.mit.edu/2024/ai-agents-help-explain-other-ai-systems-0103#:~:text=examples%2C%20the%20AIA%20actively%20participates,other%20systems%20in%20real%20time)).
- **OpenAI’s Automated Neuron Explanations (2023):** OpenAI researchers recently explored using GPT-4 to explain neurons in GPT-2 ([OpenAI is Using GPT-4 to Explain Neurons' Behavior in GPT-2 - InfoQ](https://www.infoq.com/news/2023/05/openai-gpt4-explains-gpt2/#:~:text=A%20recent%20paper%20by%20a,step%20towards%20automating%20DNN%20interpretability)). Their method is a three-step loop: (1) **Explain** – prompt GPT-4 with examples of a neuron’s top activations and ask it to hypothesize in plain English what the neuron looks for; (2) **Simulate** – have GPT-4 (or another model) predict the neuron’s activation on a wide range of inputs based on that hypothesis; (3) **Score** – compare the simulated activations against the actual neuron activations to see how well the explanation holds up ([OpenAI is Using GPT-4 to Explain Neurons' Behavior in GPT-2 - InfoQ](https://www.infoq.com/news/2023/05/openai-gpt4-explains-gpt2/#:~:text=A%20recent%20paper%20by%20a,step%20towards%20automating%20DNN%20interpretability)) ([OpenAI is Using GPT-4 to Explain Neurons' Behavior in GPT-2 - InfoQ](https://www.infoq.com/news/2023/05/openai-gpt4-explains-gpt2/#:~:text=In%20the%20first%20step%2C%20a,find%20phrases%20related%20to%20community)). For example, GPT-4 might guess *“Neuron 245 activates for phrases about community or gatherings”* ([OpenAI is Using GPT-4 to Explain Neurons' Behavior in GPT-2 - InfoQ](https://www.infoq.com/news/2023/05/openai-gpt4-explains-gpt2/#:~:text=In%20the%20first%20step%2C%20a,find%20phrases%20related%20to%20community)). The system then checks this by seeing if Neuron 245 indeed fires on words like *“team, group,”* etc., and not on unrelated words. If the match is good, the explanation is validated; if not, the process can iterate with a refined prompt. This approach effectively uses an LLM as an *analyst* that both proposes and evaluates interpretability hypotheses. It’s “agentic” in the sense that the AI is taking on tasks a human analyst would do – generating candidate explanations and testing them – all in an automated loop. While currently focused on individual neurons in language models, the method could extend to analyzing entire circuits or interactions between neurons. It highlights how a ChatGPT-like model can be harnessed as a powerful interpretability assistant, leveraging its world knowledge to articulate what a pattern in activations might represent, and its generation capabilities to design experiments.

These agent-driven methods are at the frontier of interpretability research. They marry the strengths of deep learning (pattern recognition and generation) with the investigative process of science. Crucially, they can scale analysis in ways humans alone might struggle with, by quickly sifting through thousands of neurons and zeroing in on the interesting ones with proposed meanings ([AI agents help explain other AI systems | MIT News | Massachusetts Institute of Technology](https://news.mit.edu/2024/ai-agents-help-explain-other-ai-systems-0103#:~:text=For%20example%2C%20FIND%20contains%20synthetic,and%20not%20air%20or%20sea)) ([AI agents help explain other AI systems | MIT News | Massachusetts Institute of Technology](https://news.mit.edu/2024/ai-agents-help-explain-other-ai-systems-0103#:~:text=Sarah%20Schwettmann%20PhD%20%2721%2C%20co,%E2%80%9D)). The “FIND” benchmark introduced alongside the AIA work ([AI agents help explain other AI systems | MIT News | Massachusetts Institute of Technology](https://news.mit.edu/2024/ai-agents-help-explain-other-ai-systems-0103#:~:text=Complementing%20the%20AIA%20method%20is,g)) even provides ground-truth functions to systematically evaluate how well such agents explain known computations – a sign that this approach is maturing.

## Inspirations for a ChatGPT-Based Neural Network Analysis Tool
Bringing these findings together, we can envision a new neural network analysis tool powered by ChatGPT (or similar LLMs) that leverages the best of both worlds: human-friendly dialogue and powerful automated analysis. Key design inspirations include:

- **Logging and UI from Training to Inference:** Like Comgra and TorchLens, the tool should allow tracking of any activation or weight of interest during training and afterward. The **flexibility** emphasized by Comgra – to choose between individual activations vs. summary stats, different time points, and different inputs on the fly ([Comgra: A Tool for Analyzing and Debugging Neural Networks](https://arxiv.org/html/2407.21656v1#:~:text=What%20Parts%20of%20the%20Network,these%20correlate%20with%20other%20inputs)) ([Comgra: A Tool for Analyzing and Debugging Neural Networks](https://arxiv.org/html/2407.21656v1#:~:text=The%20Need%20for%20Flexibility,At)) – suggests our ChatGPT-based tool must be able to fetch *both* granular data (e.g. “What were the layer-2 weights at epoch 5 vs epoch 50?”) and high-level summaries (“How did the weight distribution change?”) on demand. A possible implementation is a back-end logging system that records extensive data (perhaps guided by heuristics to keep it manageable), which ChatGPT can query via an API. The ChatGPT interface can then present insights in natural language, supplemented by small charts or tables, much like a conversational TensorBoard. This marries the interactive exploration of training dynamics with an LLM’s ability to summarize and explain those dynamics in plain English.
- **Model-Agnostic Analysis Modules:** To handle MLPs, CNNs, and Transformers uniformly, the tool can draw on ideas from Captum and mechanistic interpretability libraries. For instance, it could have a **“saliency probe”** that ChatGPT can invoke to compute attributions for a given input and model, or a **“activation extractor”** for any layer. By wrapping these techniques in an API, ChatGPT could say, *“I will compute which features most influenced this output”*, call an attribution method, and then explain the results. The tool’s architecture might thus be an *agent* (ChatGPT) orchestrating various modules (for attribution, activation visualization, etc.), similar to how MAIA uses a library of interpretability tools ([A Multimodal Automated Interpretability Agent](https://arxiv.org/html/2404.14394v1#:~:text=rather%20than%20labeling%20features%20in,sweeps%20over%20entire%20networks%2C%20or)) ([A Multimodal Automated Interpretability Agent](https://arxiv.org/html/2404.14394v1#:~:text=modular%20design%20of%20the%20tool,we%20use%20the%20following%20set)). Starting with simple tests on an MLP (e.g. *“Does neuron 4 fire for positive numbers?”*), the same agent could seamlessly scale up to a ResNet or Transformer, because it can query appropriate modules (e.g. attention pattern analyzer for Transformers, filter visualizer for CNNs). The **extensible design** of LIT – where new components can be added for new model types ([The Language Interpretability Tool (LIT): Interactive Exploration and Analysis o](https://research.google/blog/the-language-interpretability-tool-lit-interactive-exploration-and-analysis-of-nlp-models/#:~:text=Customizability)) – is a good blueprint for keeping the tool relevant as architectures evolve.
- **Handling Large Data via Summarization:** The challenge of large activations and weight matrices can be tackled by combining visualization techniques with ChatGPT’s summarization capabilities. For example, the tool might internally generate an Activation Atlas for a particular layer  ([Activation Atlas](https://distill.pub/2019/activation-atlas#:~:text=By%20using%20feature%20inversion%20to,network%20typically%20represents%20some%20concepts)), then have ChatGPT *interpret the atlas*: *“Layer 5 appears to have clusters for ‘building structures’ and ‘foliage textures’, indicating specialized feature detectors.”* By automating techniques like dimensionality reduction or concept labeling (à la Network Dissection), the tool can feed ChatGPT higher-level descriptors instead of raw numbers. ChatGPT’s strength in language means it could take a set of neuron labels or a graph of neuron clusters and produce a coherent narrative: *“Early layers differentiate mainly simple edges, while later layers in the CNN have neurons grouped into semantically rich concepts (faces, text, etc.) ([Activation Atlas](https://distill.pub/2019/activation-atlas#:~:text=Above%2C%20an%20activation%20atlas%20of,screen%20version)). Many neurons are devoted to texture in layer 3, which might explain why the model is texture-biased.”* Similarly, for weights, ChatGPT could report: *“Layer 10’s weight matrix has a heavy-tailed distribution (alpha ~7), which WeightWatcher suggests is a sign of under-training ([WeightWatcher: Data-Free Diagnostics for Deep Learning](https://weightwatcher.ai/#:~:text=weightwatcher%20,be%20between%202%20and%206)). This might be a weak link in the network’s performance.”* These kinds of interpretations directly draw from research insights and make them accessible.
- **Interactive Conversational Interface:** Inspired by interactive tools like ActiVis and LIT, the new tool’s interface is conversational but could also include rich media. A user might ask, *“Show me how the activations of layer 2 changed during training”*, and ChatGPT could present a small trend plot of activation means or a description: *“Layer 2’s activation variance increased and then stabilized after epoch 10, suggesting it learned a diversified set of features early on.”* The user could then ask, *“Which neurons in layer 2 are most active for class ‘cat’ images?”*, and the agent would fetch that info (perhaps by scanning the dataset) and reply with an answer and possibly an embedded image of those neurons’ feature visualizations. This *interactive Q&A* style makes analysis accessible – you don’t need to write code or dig through logs; you can simply ask questions about the network’s internals. It’s essentially ChatGPT acting as a knowledgeable guide through the model, powered by real data.
- **Automated “Agentic” Investigations:** Taking a cue from MAIA and OpenAI’s neuron explainer, the tool could have an *autonomous mode* where it performs a series of analyses by itself and reports findings. For example, upon loading a new model, the ChatGPT agent could systematically: scan for neurons with high variance, generate hypotheses about their roles (perhaps by retrieving the top activating inputs and asking itself what they have in common), and then present the user with a summary: *“I noticed neuron 87 in layer 5 consistently activates for images with text – it might be an ‘OCR/text detector’ unit. Neuron 21 in layer 7 seems to pick up on dog faces. There are 10 neurons in layer 9 that together respond to different colors, suggesting color-sensitive features.”* Each of these findings would be backed by evidence the agent gathered (which the user could ask to see, e.g. “show me examples”). The agent can also identify problematic behaviors: *“I tested the model on counterfactual inputs and found that flipping gendered words changes the output significantly, indicating a bias.”* By proactively looking for such patterns, the tool can surface critical insights without the user needing to know exactly what to ask. This *exploratory analysis mode* is directly inspired by the success of automated interpretability agents ([AI agents help explain other AI systems | MIT News | Massachusetts Institute of Technology](https://news.mit.edu/2024/ai-agents-help-explain-other-ai-systems-0103#:~:text=Central%20to%20this%20strategy%20is,other%20systems%20in%20real%20time)) ([AI agents help explain other AI systems | MIT News | Massachusetts Institute of Technology](https://news.mit.edu/2024/ai-agents-help-explain-other-ai-systems-0103#:~:text=Sarah%20Schwettmann%20PhD%20%2721%2C%20co,%E2%80%9D)) and would be a standout feature of a ChatGPT-powered framework.

In summary, recent academic research and open-source projects chart a clear path toward a more intelligent neural network analysis tool. From **tracking training dynamics** (Comgra, TorchLens) to **visualizing internals** (ActiVis, LIT) and **automating interpretability** (MAIA, OpenAI’s GPT-4 explainer), each provides pieces of the puzzle. A new tool based on ChatGPT APIs can combine these pieces: it would log and visualize like existing frameworks, but crucially, also **converse and reason** about the network’s behavior. By doing so, it directly leverages the strengths identified in the literature – flexibility in data inspection ([Comgra: A Tool for Analyzing and Debugging Neural Networks](https://arxiv.org/html/2407.21656v1#:~:text=The%20Need%20for%20Flexibility,At)), multi-scale visualization ([[1704.01942] ActiVis: Visual Exploration of Industry-Scale Deep Neural Network Models](https://arxiv.org/abs/1704.01942#:~:text=participatory%20design%20sessions%20with%20over,of%20how%20ActiVis%20may%20work)), conceptual summarization ([[1704.05796] Network Dissection: Quantifying Interpretability of Deep Visual Representations](https://arxiv.org/abs/1704.05796#:~:text=,our%20method%20to%20compare%20the)), and autonomous hypothesis generation ([AI agents help explain other AI systems | MIT News | Massachusetts Institute of Technology](https://news.mit.edu/2024/ai-agents-help-explain-other-ai-systems-0103#:~:text=examples%2C%20the%20AIA%20actively%20participates,other%20systems%20in%20real%20time)) – to help users deeply understand their neural networks in an intuitive, human-centered way. The result would be an AI-powered “copilot” for neural network interpretability, turning state-of-the-art research ideas into practical tooling for model developers. 

**Sources:**

1. Dietz, F. *et al.* (2023). *Comgra: A Tool for Analyzing and Debugging Neural Networks* ([Comgra: A Tool for Analyzing and Debugging Neural Networks](https://arxiv.org/html/2407.21656v1#:~:text=The%20Need%20for%20Flexibility,At)) ([Comgra: A Tool for Analyzing and Debugging Neural Networks](https://arxiv.org/html/2407.21656v1#:~:text=Comgra.%20Comgra%20,details%20for%20a%20specific%20input))  
2. Taylor, J.M. & Kriegeskorte, N. (2023). *TorchLens: A Python package for extracting and visualizing hidden activations* ([Comgra: A Tool for Analyzing and Debugging Neural Networks](https://arxiv.org/html/2407.21656v1#:~:text=Visualizing%20the%20Computation%20Graph,a%20toolkit%20for%20visualizing%20models))  
3. Kokhlikyan, N. *et al.* (2020). *Captum: A unified model interpretability library for PyTorch* ([Comgra: A Tool for Analyzing and Debugging Neural Networks](https://arxiv.org/html/2407.21656v1#:~:text=Attribution%20and%20Interpretability,to%20interpret%20your%20model%E2%80%99s%20behavior))  
4. Carter, S. *et al.* (2019). *Activation Atlas (Distill)* ([Activation Atlas](https://distill.pub/2019/activation-atlas#:~:text=By%20using%20feature%20inversion%20to,network%20typically%20represents%20some%20concepts)) ([Activation Atlas](https://distill.pub/2019/activation-atlas#:~:text=Unfortunately%2C%20visualizing%20activations%20has%20a,a%20time%20will%20not%20suffice))  
5. Bau, D. *et al.* (2017). *Network Dissection: Quantifying Interpretability of Deep Visual Representations* ([[1704.05796] Network Dissection: Quantifying Interpretability of Deep Visual Representations](https://arxiv.org/abs/1704.05796#:~:text=,our%20method%20to%20compare%20the)) ([[1704.05796] Network Dissection: Quantifying Interpretability of Deep Visual Representations](https://arxiv.org/abs/1704.05796#:~:text=latent%20representations%20of%20various%20networks,interpretability%20of%20deep%20visual%20representations))  
6. Kahng, M. *et al.* (2018). *ActiVis: Visual Exploration of Industry-Scale DNNs* ([[1704.01942] ActiVis: Visual Exploration of Industry-Scale Deep Neural Network Models](https://arxiv.org/abs/1704.01942#:~:text=participatory%20design%20sessions%20with%20over,of%20how%20ActiVis%20may%20work))  
7. Tenney, I. *et al.* (2020). *Language Interpretability Tool (LIT)* ([The Language Interpretability Tool (LIT): Interactive Exploration and Analysis o](https://research.google/blog/the-language-interpretability-tool-lit-interactive-exploration-and-analysis-of-nlp-models/#:~:text=LIT%20supports%20local%20explanations%2C%20including,in%20our%20system%20demonstration%20paper))  
8. Rott Shaham, T. *et al.* (2024). *MAIA: A Multimodal Automated Interpretability Agent* ([AI agents help explain other AI systems | MIT News | Massachusetts Institute of Technology](https://news.mit.edu/2024/ai-agents-help-explain-other-ai-systems-0103#:~:text=Central%20to%20this%20strategy%20is,other%20systems%20in%20real%20time)) ([AI agents help explain other AI systems | MIT News | Massachusetts Institute of Technology](https://news.mit.edu/2024/ai-agents-help-explain-other-ai-systems-0103#:~:text=For%20example%2C%20FIND%20contains%20synthetic,and%20not%20air%20or%20sea))  
9. OpenAI (2023). *Language Models can Explain Neurons in Language Models* ([OpenAI is Using GPT-4 to Explain Neurons' Behavior in GPT-2 - InfoQ](https://www.infoq.com/news/2023/05/openai-gpt4-explains-gpt2/#:~:text=A%20recent%20paper%20by%20a,step%20towards%20automating%20DNN%20interpretability)) ([OpenAI is Using GPT-4 to Explain Neurons' Behavior in GPT-2 - InfoQ](https://www.infoq.com/news/2023/05/openai-gpt4-explains-gpt2/#:~:text=In%20the%20first%20step%2C%20a,find%20phrases%20related%20to%20community))  
10. Martin, C. *et al.* (2023). *WeightWatcher: Data-Free Diagnostics for Deep Neural Networks*

================================================================================

